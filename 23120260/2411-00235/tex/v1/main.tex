\documentclass[a4paper, onecolumn, superscriptaddress, aps, pra, 11pt, notitlepage, tightenlines]{revtex4-2}
\pdfoutput=1
\usepackage{preamble}
\usepackage{times}

\usepackage{tikz}


\renewcommand{\braket}[1]{\left\langle #1 \right\rangle}

\def\apropto{%
  \def\p{%
    \setbox0=\vbox{\hbox{$\propto$}}%
    \ht0=0.6ex \box0 }%
  \def\s{%
    \vbox{\hbox{$\sim$}}%
  }%
  \mathrel{\raisebox{0.7ex}{%
      \mbox{$\underset{\s}{\p}$}%
    }}%
}
\def\llangle{\langle\!\langle}
\def\rrangle{\rangle\!\rangle}

\def\sbra#1{\mathinner{\llangle{#1}|}}
\def\sket#1{\mathinner{|{#1}\rrangle}}
\def\sbraket#1{\mathinner{\llangle{#1}\rrangle}}
\def\sketbra#1{\mathinner{|{#1}\rrangle}\!\mathinner{\llangle{#1}|}}


\DeclareMathOperator*{\SumInt}{%
\mathchoice%
  {\ooalign{$\displaystyle\sum$\cr\hidewidth$\displaystyle\int$\hidewidth\cr}}
  {\ooalign{\raisebox{.14\height}{\scalebox{.7}{$\textstyle\sum$}}\cr\hidewidth$\textstyle\int$\hidewidth\cr}}
  {\ooalign{\raisebox{.2\height}{\scalebox{.6}{$\scriptstyle\sum$}}\cr$\scriptstyle\int$\cr}}
  {\ooalign{\raisebox{.2\height}{\scalebox{.6}{$\scriptstyle\sum$}}\cr$\scriptstyle\int$\cr}}
}
\usepackage[OT2,T1]{fontenc}
\DeclareSymbolFont{cyrletters}{OT2}{wncyr}{m}{n}
\DeclareMathSymbol{\Sha}{\mathalpha}{cyrletters}{"58}

\newtcolorbox{mybox}{colback=gray!10!white, colframe=gray}

\begin{document}
\title{Chasing shadows with Gottesman--Kitaev--Preskill codes}
\author{Jonathan Conrad}
\thanks{\url{j.conrad1005@gmail.com}}
\affiliation{Dahlem Center for Complex Quantum Systems, Physics Department, Freie
Universit{\"a}t Berlin, Arnimallee 14, 14195 Berlin, Germany}
\affiliation{Helmholtz-Zentrum Berlin f{\"u}r Materialien und Energie, Hahn-Meitner-Platz 1, 14109
Berlin, Germany}

\author{Jens Eisert}
\affiliation{Dahlem Center for Complex Quantum Systems, Physics Department, Freie
Universit{\"a}t Berlin, Arnimallee 14, 14195 Berlin, Germany}
\affiliation{Helmholtz-Zentrum Berlin f{\"u}r Materialien und Energie, Hahn-Meitner-Platz 1, 14109
Berlin, Germany}

\author{Steven T.\ Flammia}
\affiliation{Department of Computer Science, Virginia Tech, Alexandria, USA}
\affiliation{Phasecraft Inc., Washington DC, USA}

\date{\today}

\begin{abstract}
The infinitude of the continuous variable (CV) phase space is a serious obstacle in designing randomized tomography schemes with provable performance guarantees. 
A typical strategy to circumvent this issue is to impose a regularization, such as a photon-number cutoff, to enable the definition of ensembles of random unitaries on effective subspaces. 
In this work, we consider the task of performing shadow tomography of a logical subsystem defined via the Gottesman--Kitaev--Preskill (GKP) error correcting code. 
In particular, we construct a logical shadow tomography protocol via twirling of CV-POVMs by displacement operators and Gaussian unitaries. 
In the special case of heterodyne measurement, the shadow tomography protocol yields a probabilistic decomposition of any input state into Gaussian states that simulate the encoded logical information of the input relative to a fixed GKP code and we prove bounds on the Gaussian compressibility of states in this setting. 
For photon-parity measurements, logical GKP shadow tomography is equivalent to a Wigner sampling protocol for which we develop the appropriate sampling schemes and finally, using the existence of a Haar measure over symplectic lattices, we derive a Wigner sampling scheme via random GKP codes. 
This protocol establishes, via explicit sample complexity bounds, how Wigner samples of any input state from random points relative to a random GKP codes can be used to estimate any sufficiently bounded observable on CV space.
\end{abstract}
\maketitle


\section{Introduction}

Recent years have seen steady progress in experimental realizations of quantum experiments. 
On the one hand, experiments towards qubit-based quantum error correction have demonstrated impressive control of large quantum systems consisting of hundreds of physical qubits to encode and process encoded logical information~\cite{Bluvstein_2023}, as well the ability to encode quantum information into a single quantum harmonic oscillator beyond break-even via the \emph{Gottesman--Kitaev--Preskill} (GKP) code from the realm of bosonic quantum error correction~\cite{Sivak_2023, GKP, Terhal_2020}. 
On the other hand, quantum experiments are being conducted to understand practical capabilities of present noisy quantum devices, e.g., through variational quantum algorithms \cite{Cerezo_2021}, or to benchmark the readily accessible ``quantumness" \cite{arute2019quantum} through randomized sampling experiments. 
Aside from experimental progress, the design of error mitigation and smart post-processing techniques yields valuable insights into the design of future experiments and has specifically developed into a quest for learning properties of quantum states from randomly accessible snapshots. 

Any such effort makes sense only, however, 
if the anticipated state preparations or protocols are being implemented with high levels of accuracy. 
To ensure this, one usually has to resort to techniques of benchmarking, certification or
tomographic recovery \cite{BenchmarkingReview}. 
An interesting technique in this realm is the so-called classical shadow tomography protocol \cite{aaronson2018shadow, Huang_2020, Toolbox} that demonstrates just how little classical information needs to be extracted from any quantum state to reproduce the expectation values of a bounded number of suitably bounded observables. 
One of the core appealing features of this family of protocols is that the same kind of measurement is appropriate for many observables, so that the choice on what observables to focus on can be made later.

On the technical level, the (classical) shadow tomography protocol combines two ingredients: the fact that a channel-twirl of a POVM projects it onto a channel with the fixed structure of a depolarizing channel and the existence of strong statistical anticoncentration bounds for medians of means estimation. 
The protocol proceeds by implementing a Clifford channel-twirl of a POVM that outputs samples over the reconstructed pointer states -- which are stabilizer states -- such that expectation values of observables over such samples match the expectation value of the depolarized input state and can be classically processed to yield the targeted expectation value. 
The classical post-processing necessary is informed by the structure of the depolarizing channel that resulted from the projection and is efficiently possible due to the stabilizer state structure of the samples. 
Medians of means estimation then yields a process to combine samples and prove bounds on the necessary sample complexity. 
It has been recognized in refs.~\cite{Chen_2021, Koh2022classicalshadows} that the projective nature of the channel twirl allows one to render the protocol robust to errors in the POVM, as any noisy version of the POVM would simply be projected onto depolarizing channel with amended parameters, which can be accounted for in post-processing. 
In fact, the projection is such a strong feature that any informationally complete POVM, possibly combined with a noise channel, can be used to implement the protocol as long as the effective channel is not fully depolarizing. 

Bosonic quantum systems also offer a wide variety of POVMs beyond those exactly expressible in finite systems, such as heterodyne, homodyne, photon-counting, and photon parity measurements. 
Despite the richness of these POVMs, they can be similarly tamed into an effective channel with a simple structure by appropriate twirling.
Recent work \cite{iosue2022continuousvariable,  gandhari2022continuousvariable} has shown that twirling techniques are applicable for continuous variable quantum systems by only focusing only on certain energy-constrained subspaces.  
Prior work has treated this constraint as necessary due to the inaccessibility of random operations on this infinite Hilbert space.  

In this work, we develop shadow tomography protocols by focusing on logical subsystems prescribed by the GKP code. 
This also yields an effective finite subspace of the infinite dimensional CV Hilbert space and we show how effective shadow tomography protocols can be derived that reproduce logical expectation values of operators relative to the chosen GKP-codes. 
On the technical level, this is executed by twirling a CV-POVM over a set of random \textit{logical} Clifford gates, which has the effect that the logical action of the POVM becomes projected onto a depolarizing channel. 
This effect is revealed when the pointer states output by the protocol are evaluated in accordance to a decoder associated to the code. 
This structure reveals an interesting interplay between the physical structure of the system and its logical content. 
We identify different applications of the GKP-shadow tomography toolbox developed here by considering different choices of bosonic POVMs and finally show how a general shadow protocol for CV states can be obtained by combining our GKP-shadow tomography tools with a random choice of GKP codes. 

For example, when the protocol is executed using heterodyne measurement as POVM, the fact that GKP Clifford gates are represented by Gaussian unitary operations implies that the protocol outputs an ensemble of \textit{Gaussian states}, which contain the same logical information (relative to the chosen GKP code) as the given input state. 
Here the key upshot of our combination of techniques is that the statistical methods used in classical shadow tomography allow for the derivation of rigorous bounds on the number of such Gaussian states needed to faithfully retain the logical content of the input. 
While the bounds we derive scale exponentially in the system size, the key point of this result is that they are obtained \textit{without} knowledge of an analytical expression for the input state. 
This protocol yields an experimental \textit{black-box} procedure to convert an arbitrary physical input state into a convex combination of Gaussian states. 
As Gaussian states are easy to simulate classically \cite{Weedbrook_2012}, we expect this technique to be of value in assessing the performance of quantum computation and error correction using real GKP states. 

In the concrete application with photon-parity detectors we consider, our protocol descends to one that samples the Wigner function of a given quantum state at random points according to a well-tailored distribution and we show how the toolbox we have developed allows one to derive sample complexity bounds for this protocol to estimate a number of arbitrary observables within certain bounds, unconditional on properties of the input state. 
This is achieved by combining the logical GKP shadow tomography protocol with a random choice of GKP code. Here the core technical ingredient is the existence of a Haar measure over the space of symplectic lattices and simple expressions for averages of functions of lattices over this measure. 
At the bottom, this is a randomized protocol to sample the Wigner function of an arbitrary CV state where the intricate way in which we choose where to sample allows us to rigorously bound the required sample complexity to estimate given CV observables to high confidence. 
By averaging over GKP codes, this protocol effectively interpolates between tomography of the \textit{logic} encoded in a physical system, and its \textit{physics}. 

The key point of this work, however, is to highlight the intersection between continuous variable physics and randomized tomographic methods originally derived for discrete variable systems. 
As becomes apparent in the course of our presentation, thinking about classical shadow tomography through the lens of GKP codes helps to refine our general understanding of the nuances of the classical shadow protocol while, vice versa, we obtain experimental handles to learn relevant aspects of a physical CV state using methods from random coding theory. 
Next to the presentation of the concrete results our purpose here is hence a pedagogical one: we hope that, through the lens of GKP codes, our explorations help the curious reader to develop a more refined understanding of the interesting intersection between physics, logic and everything in between.

This article is structured as follows.  
We begin by a review of the basics of quantum harmonic oscillators and the structure of GKP codes in sec.\  \ref{sec:prelim}. 
In sec.\  \ref{sec:twirl}, we give a broad overview on twirling and discuss its various incarnations in state purification protocols, dynamical decoupling, noise mitigation and shadow tomography. 
This section is meant to provide a pedagogical ground-up introduction to the utility of twirling and the role of random operations for quantum experiments. 
We discuss extensions of these tools to the realm of continuous variable systems and explain how the infinitude of the associated groups of Gaussian unitaries can be suitably regularized. 
Finally, in sec.\ \ref{sec:GKPShadows} we apply the developed tools to design and prove bounds for logical shadow tomography protocols relative to GKP codes,  where we also examine the behavior of GKP shadow tomography with random GKP codes in sec.~\ref{sec:chasing_shadows}. 
We show this yields a protocol that allows one to estimate \textit{arbitrary} CV observables in a regularized manner. 
The core contribution of this section is the introduction of new techniques for estimating and bounding the performance of full tomography of a CV quantum system.
We close with a brief discussion and open questions for future work. 
For a better flow of presentation, the detailed proofs of most statements made throughout this manuscript are found in the appendix.

\section{Preliminaries}\label{sec:prelim}
\subsection{Quantum harmonic oscillators}
Bosonic quantum error correction studies the robust embedding of discrete quantum information into a system of multiple \emph{quantum harmonic oscillators} (QHO), each of which can be described by an infinite dimensional Hilbert space $\CH=\operatorname{span}\lrc{\ket{n}}_{n=0}^{\infty}$ where $\ket{n}$ denote 
the well known Fock state vectors whose labels correspond to the \emph{eigenvalues} of the number operator $\hat{n}=\hat{a}^{\dagger}\hat{a}$ and $\hat{a}=(\hat{q}+i\hat{p})/\sqrt{2}$ denotes the annihilation operator.  $\hat{q}$ and $\hat{p}$ are the position and momentum operators, the canonical 
coordinates, whose improper eigenstates yield a basis for the underlying Hilbert 
space. The associated phase space inherits a non-trivial geometry from the canonical commutation relations (we will set $\hbar=1$ throughout),
and is most naturally studied in the Heisenberg 
frame,  i.e., in terms of the transformation behaviour of operators on this space. 
On a system of $n$ QHOs -- which we will refer to as having $n$ \textit{modes} --  we define a generalized quadrature operator  $\bs{\hat{x}}=\lr{\hat{q}_1, \hat{q}_2, \dots
, \hat{p}_{n-1} , \hat{p}_n }^T$ such that $\lrq{\hat{x}_i,  \hat{x}_j}=iJ_{i, j}$ where 
\begin{equation}
J_n=\begin{pmatrix}
0 & I_n \\ -I_n & 0
\end{pmatrix}
\end{equation} is the anti-symmetric symplectic form and $I_n$ denotes the $n \times n$ identity matrix. Unless explicitly needed, we will omit the index $n$ from the symplectic form and simply denote it by $J$.


Analogous to the Pauli-operators for qubit-systems, the Heisenberg-Weyl operators for this infinite dimensional Hilbert space are given by displacement operators
\begin{align}
D\lr{\bs{\xi}} = \exp\left\{-i \sqrt{2\pi} \bs{\xi}^T J \bs{\hat{x}}\right\}\label{eq:displConvJ}
\end{align}
for $\bs{\xi} \in \R^{2n}$ being
elements of phase space. 
These displacement operators satisfy
the \emph{Weyl relations}
\begin{align}
D\lr{\bs{\xi}}
D\lr{\bs{\eta}}
= e^{-i\pi \bs{\xi}^TJ\bs{\eta}}
D\lr{\bs{\xi}+ \bs{\eta}}
\end{align}
for $\bs{\xi},\bs{\eta}\in \R^{2n}$
\footnote{The Weyl relations are actually a way of rigorously capturing the canonical commutation relations without having to resort to 
unbounded operators.}.
They form a basis for operators 
and are Hilbert-Schmidt orthogonal as $\Tr\lrq{D^{\dagger}\lr{\bs{\xi}}D\lr{\bs{\eta}}}=\delta^{(2n)}\lr{\bs{\xi}-\bs{\eta}}$, such that states can be represented by their Wigner function
\begin{equation}
W_{\rho}\lr{\bs{x}}
= \int_{\mathbb{R}^{2n}} d\bs{\eta}\, e^{-i2\pi \bs{x}^T J \bs{\eta}} \Tr\lrq{D\lr{\bs{\eta}} \rho}. \label{eq:Wigner}
\end{equation}

Displacement operators represent the unitary time evolution induced by Hamiltonians linear in the quadrature operators that implement the transformation \begin{equation}
D\lr{\bs{\xi}}^{\dagger} \bs{\hat{x}}D\lr{\bs{\xi}}=\bs{\hat{x}} + \sqrt{2\pi}\bs{\xi}
\end{equation}
 and commute and multiply 
 as
\begin{align}
D\lr{\bs{\xi}}D\lr{\bs{\eta}}&=
e^{-i\pi \bs{\xi}^TJ\bs{\eta}}
D\lr{\bs{\xi}+\bs{\eta}}, \\
&=
e^{-i2\pi \bs{\xi}^TJ\bs{\eta}}D\lr{\bs{\eta}}D\lr{\bs{\xi}}.
\nonumber
\end{align}
It is these properties that make them a natural set to choose stabilizer groups from.

Unitary evolution via Hamiltonians strictly quadratic in the quadrature operators, 
also termed \textit{Gaussian} unitary transformations \cite{GaussianQuantumInfo,Continuous}, implement symplectic transformations
\begin{align}
U=e^{-\frac{i}{2}\bs{\hat{x}}^T C \bs{\hat{x}}},\; C=C^T, \\
U^{\dagger}\bs{\hat{x}}U=S\bs{\hat{x}},\; S=e^{CJ}, \label{eq:sympH}
\end{align}
where $S\in \Sp_{2}\lr{\R}=\lrc{S\in \R^{2\times 2} : S^TJS=J}$ is a real symplectic matrix which follows from unitarity of $U$ and we have 
\begin{equation}
D\lr{\bs{\xi}}U_S=U_SD\lr{S^{-1}\bs{\xi}},
\end{equation}
such that it also holds that
\begin{equation}
W_{U_S\rho U_S^{\dagger}}\lr{\bs{x}}=W_{\rho}\lr{S\bs{x}}.
\end{equation}

\subsection{GKP codes and their Cliffords}
\begin{figure}
\includegraphics[width=.7\columnwidth]{Z2A2GKP.pdf}
\caption{The square $\Z^2$ (l.) and hexagonal $A_2$ (r.) GKP codes each encoding a qubit. The logical displacement amplitudes are marked in turqóise and stabilizer displacements are marked in red.}\label{fig:Z2_A2} 
\end{figure}

The GKP code \cite{GKP, Conrad_2022} is a quantum error correcting code defined to embed discrete quantum information into a system of $n$ quantum harmonic oscillators by identifying a code space symmetric under the stabilizer group
\begin{equation}
\CS=\big\langle D\lr{\bs{\xi}_1}\hdots D\lr{\bs{\xi}_{2n}}   \big\rangle=\lrc{e^{i\phi_M\lr{\bs{\xi}}}D\lr{\bs{\xi}},\; \bs{\xi}\in \CL},  \label{eq:GKPdef}
\end{equation}
which is isomorphic to a full rank lattice $\CL= \mathbb{Z}^{2n}M$ with generator matrix
\begin{equation}
M=(\bs{\xi}_1,  \bs{\xi}_2,  \hdots ,  \bs{\xi}_{2n})^T.
\end{equation}
The symplectic dual lattice 
\begin{equation}
\CL^{\perp}=\lrc{\bs{x}\in \R^{2n}:\; \bs{x}^TJ\bs{\xi} \in \Z\, \forall \bs{\xi} \in \CL}
\end{equation}
labels the centralizer of the GKP stabilizer group,  such that the GKP stabilizer group is abelian if and only if it is isomorphic to a \textit{weakly symplectically self-dual} lattice
\begin{equation}
\CL \subseteq \CL^{\perp} \Leftrightarrow M=AM^{\perp},
\end{equation}
where the right hand side describes the sublattice structure by identifying how basis vectors of $\CL$ are described by (integer) linear combinations of basis vectors of $\CL^{\perp}$ as given by the
the symplectic Gram matrix $A=MJM^T$ when the dual basis is chosen via some canonical choice \cite{GKP}.  
The phases $\Phi_M\lr{\bs{\xi}}=\pi \bs{a}^T A_{\lowertriangle}\bs{a},\, \bs{a}=M^{-T}\bs{\xi} $ in eq.\ \eqref{eq:GKPdef} are determined by the basis in which the stabilizer generators are fixed to a $+1$ eigenvalue and are trivial when the symplectic Gram matrix $A$ is even \cite{Conrad_2022}.

A special class of GKP codes, called \textit{scaled} GKP codes, is obtained from rescaling a symplectic self dual lattice $\CL_0=\CL^{\perp}_0 \mapsto \CL=\sqrt{d}\CL_0:\, \CL \subseteq\CL^{\perp}, \;  $ via the square root of the desired local dimension $d\in \N$ and gives rise to the well-known GKP codes that encode a qubit ($d=2$) into a single oscillator via the square- or the hexagonal lattice with bases


\begin{equation}
M_{\Z^2}=\begin{pmatrix}
1 & 0 \\ 0 & 1
\end{pmatrix},\,
M_{A_2}=\frac{1}{\sqrt{2\sqrt{3}}}\begin{pmatrix}
2 & 0 \\ 1 & \sqrt{3}
\end{pmatrix}.
\end{equation}

These GKP codes have been widely explored in the literature: They afford a distance (given by the length of the shortest non-trivial logical
displacement) $\Delta\lr{\sqrt{2}\Z^2}=2^{-\frac{1}{2}}$ and $\Delta\lr{\sqrt{2}A_2}=3^{-\frac{1}{4}}$. We depict their structure in fig. \ref{fig:Z2_A2}, where it can also be seen that the lattices are respectively symmetric under $\pi/2$ and $\pi/3$ rotations $R_{\pi/2}$ and $R_{\pi/3}$ which correspond to the logical Hadamard $\hat{H}$ gate for the square GKP code and the Hadamard-phase gate $\hat{H}\hat{S}^{\dagger}$ for the hexagonal GKP code. 

The identification as logical Clifford gates is made through their property
as symplectic lattice automorphisms, of which the general structure has been explored in refs.~\cite{garden, burchards2024fiberbundlefaulttolerance, Royer_2022, GCB}. Another such symplectic
automorphism is given by the transvection $S=I+\bs{e}_{1}\bs{e}_2^T$
that yields a logical phase gate $\hat{S}$ for the square GKP code. 
In general, for scaled GKP codes all symplectic autmorphisms are given by the symplectic matrices $S=M^TU^TM^{-T}$, where $M$ is the generator for the lattice basis and $U\in \Sp_{2n}\lr{\Z_d}$ labels the logical action of the corresponding non-trivial Clifford gate. We refer to ref.~\cite{garden} for an in-depth discussion. In the appendix, we show that any such $U\in \Sp_{2n}\lr{\Z_d}$  (and consequently, by conjugation with $M^T$, any corresponding real symplectic automorphism $S$) can be generated by a length $O(dn^2)$ sequence of elementary local matrices in $\Sp_{2n}\lr{\Z_d}$ that correspond to the qudit versions of the usual Hadamard, phase  and CNOT gates. 

\section{Twirling theory}\label{sec:twirl}

We now turn to discussing constructions and applications of \textit{random} logical GKP Clifford gates.  While random Clifford gates is a widely and well studied topic for qubit-based systems,  for the GKP code the question of how to define a measure over the -- now infinite -- trivial-  and non-trivial group becomes more nuanced.  

Constructions of random (trivial or non-trivial) Clifford gates find widespread applications from state preparation to error mitigation to benchmarking and most notably in many recent works on shadow tomography.  The common ground of these applications is that the random Clifford gates are used to implement various incarnations of group projectors -- also phrased \textit{twirling} -- which we summarize in fig.~\ref{fig:twirl}. 
As exploited in ref.~\cite{Conrad_2021} for the square GKP code to Floquet-engineer a GKP Hamiltonian, once a suitable measure for one type of twirl is found, it is easy to translate it into the different incarnations.

\begin{figure*}
\center
\includegraphics[width=.75\linewidth]{Twirling.pdf}
\caption{Different notions of twirling. }\label{fig:twirl}
\end{figure*}

\subsection{Twirling states, channels and POVMs} 

In this section, we first review how the different incarnations of twirling work on a qubit level before discussing its logical analogue using the GKP code. 

\paragraph{State twirling.} 
On the (logical) qubit level,  a state twirl over a group $\CG \subseteq \CU\lr{N}$ maps
\begin{equation}
\rho \mapsto \Pi_{\CG}\lr{\rho}=\int_{\CU\lr{N}} d\mu\lr{U}\,  U\rho U^{\dagger},\label{eq:group_proj}
\end{equation}
where the integration is taken over the group
of all unitaries and we specify the measure $\mu^{\CG}\lr{U}=\chi\lr{U\in \CG}/|\CG|$ that is constant  whenever $U=R\lr{g}$ is a unitary representation of an element in $\CG$.
For this uniform measure over $\CG$ it is quickly verified that the group projector indeed is a projector $\Pi_{\CG}^2=\Pi_{\CG}$ and that it maps onto the \textit{commutant} $\lrc{U\in \CU\lr{N}:\; [U,g]=0\,  \forall g \in \CG}$ which is spanned by the \emph{irreducible representations} (irreps) of $\CG$.  Note that in eq.\  \eqref{eq:group_proj} phases attached to each group element $g$ do not matter and we in fact only need that the operators used in \eqref{eq:group_proj} form a projective unitary representation of the group.  The group projector associated to the Pauli group $\CP=\langle \hat{X},  \hat{Z} \rangle$ can hence be written as
\begin{equation}
\Pi_{\CP}=\Pi_{\langle \hat{X}\rangle} \circ \Pi_{\langle \hat{Z}\rangle}
\end{equation}
and projects any input state onto a state that is invariant under both random bit-  and phase-flips.  The only state satisfying this property is completely mixed $\rho=I/2$,  aligned with the fact that the Pauli group forms a unitary $1$-design -- i.e.,  group twirls involving single powers $g,  g^{\dagger}$ as in eq.\ \eqref{eq:group_proj} act on a state in the same way as a twirl over the whole unitary group $\CG=\CU\lr{N}$ which has only the identity $I$ in its commutant. 

The same idea also works for suitable subgroups, giving rise to twirls that project onto suitable symmetric subspaces. 
For example, on the logical qubit level, we can consider state twirls for a group $\CG \subseteq \CU\lr{N}$ that acts as
\begin{equation}
\rho \mapsto \Pi_{\CG}\lr{\rho}=\int_{\CU\lr{N}} d\mu\lr{U}\,  (U\otimes U) \rho (U \otimes U)^{\dagger},
\end{equation}
where the integration is taken over the Haar measure over $\CG$. This is a group projector onto the commutant $\lrc{U\in \CU\lr{N}:\; [U\otimes U,g]=0\,  \forall g \in \CG}$
that maps quantum states onto logical Werner states. On the logical level, the commutant is spanned by the identity and the flip operator $\mathbb{F}$. 
Symmetric quantum states
$\rho$ in this sense are from the intersection of state space and the commutant, and will,
for two logical qubits, hence be of 
the form
\begin{equation}
	\Pi_{W}\lr{\rho}=\lambda \frac{1}{4}I + (1-\lambda) \frac{1}{2}\mathbb{F},
\end{equation}	
for $\lambda\in[0,1]$. This form of the twirl has been of relevance in protocols to distill entangled states \cite{Werner89,Bennett_1996}.

\paragraph{Channel twirling.} 

A channel twirl $\tau_{\CG}$ maps a quantum channel,  written the channel in its $\chi$-matrix representation with $P_{\alpha}$ a Pauli operator with index $\alpha$
\begin{equation}
\CC\lr{\cdot}=\sum_{\alpha, \beta}\chi_{\alpha, \beta}P_{\alpha}\cdot P_{\beta} \label{eq:Lambda_chi}
\end{equation}
onto a ``symmetrized'' version of itself.   Specifically,  it is defined 
as the map
\begin{align}
\tau_{\CG} \circ \CC=\int_{\CU\lr{N}} d\mu\lr{U}\,   U\CC\lr{U^{\dagger}\cdot U}U^{\dagger}.
\end{align}
In vectorized notation where $\widehat{\CC}$ acts on state vectors $\sket{\rho}\mapsto \widehat{\CC}\sket{\rho}$,  this is recast in the recognizable form 
\begin{align}
\tau_{\CG} \circ \widehat{\CC}=\int_{\CU\lr{N}} d\mu\lr{U}\,  \CU \widehat{\CC} \CU^{\dagger}, \;  \CU:=U\otimes U^*
\end{align}
and again for the uniform measure $\mu^{\CG}$ on $\CG$ we can recognize that the channel is projected onto the commutant of the group representation $R\lr{\CG\otimes \CG^*}= \lrc{R\lr{g}\otimes R(g)^*\,  \forall g \in \CG}$.  

A Clifford twirl on a channel can be understood as a combination of twirls  
\begin{equation}
\Pi_{\Cl\lr{N}}=\Pi_{\Cl\lr{N}/P\lr{N}} \circ \Pi_{P\lr{N}}
\end{equation}
over the trivial- and non-trivial Clifford groups \cite{Dankert_2009,  DiVincenzo_2002},
where the channel twirl enforces that every pair of Pauli operators $P_{\alpha} \otimes P_{\beta}^*$ commute with every operator of the form $P\otimes P^*,  P\in \CP\lr{N}$.  Since Pauli operators only commute up to phases,  this twirl effectively enforces that the channel only has ``diagonal'' Pauli elements for which $\alpha=\beta$ in eq.\  \eqref{eq:Lambda_chi}.
Subsequently,  the twirl over the non-trivial Clifford elements enforces permutation symmetry on the non-trivial Pauli elements indexed in $\chi_{\alpha, \alpha}, \alpha \neq 0$.  For a single qubit,  the desired measure over non-trivial Cliffords hence should satisfy (up to phases)
 \begin{equation}
 \mu_{\Cl\lr{2}/\mathcal{P}} ( I\neq P \in \CP \mapsto 
X,  Y,  Z) =\frac{1}{3}, 
\label{eq:Clifftwirl}
 \end{equation}
and by assuming the input channel to be trace preserving,  the twirl produces a channel
\begin{equation}
\Pi_{\Cl\lr{N}}\circ \CC \lr{\rho}=\chi_{0,0}\rho+\frac{1-\chi_{0,0}}{3}\sum_{\alpha>0}P_{\alpha} \rho P_{\alpha}.
\end{equation}
Writing
\begin{equation}
\sum_{\alpha>0}P_{\alpha} \rho P_{\alpha}=\sum_{\alpha}P_{\alpha} \rho P_{\alpha} - \rho = 4^n \Pi_{\CP}\lr{\rho} - \rho, 
\end{equation}
we obtain, knowing that the Pauli state-twirl produces a completely mixed state $\Pi_{\CP}\lr{\rho}=2^{-1} I$,
\begin{equation}
\Pi_{\Cl\lr{N}}\circ \CC \lr{\rho}=\frac{4 \chi_{0,0}-1}{3} \rho + \frac{2(1-\chi_{0,0})}{3}I.\label{eq:CL_twirled_channel}
\end{equation}
A channel twirl over the Clifford group $\Pi_{\Cl\lr{2}}$ that produces an output of such a form  -- i.e., it is in vectorized notation a linear combination of the trivial channel $I\otimes I^*$ and the fully depolarizing channel $\sketbra{I}$ -- is also called a Clifford $2$-design
\cite{Gross_2007}. 
There are several ways to implement a twirl over either the full or non-trivial Clifford group. 
%{\red{Can we avoid the term non-trivial Clifford group?}}\jc{hm, I sort of think its useful to have this distinguisher, even if I agree that is is ugly.}
One straightforward way to define a non-trivial Clifford twirl is to pick one of the $6$ elements in $\Sp_2\lr{2}$ at random  (corresponding to the matrices with $1$ entries on the diagonal and anti-diagonal and one of the four choices of $\Z_2^{2\times 2}$ matrices with one $0$ entry).  
While this strategy is the simplest,  it requires to exhaustively enumerate all elements of the non-trivial Clifford group does not generalize easily to a multi-qubit setting where it becomes desirable to approximate such twirl via a random walk over $\Sp_{2n}\lr{\Z}$ or,  more commonly,  to directly sample random Clifford circuits (involving Pauli gates) gate-by-gate to set up a good approximation to a (Haar-)random Clifford circuit.
On a single qubit,  the first strategy effectively samples from one of the $24$ elements in the set 
 \begin{align}
 S_0 =\{CP,\; C\in \Cl\lr{2}/\CP, \; P\in \CP \} \subseteq \C^{2\times 2}
 \end{align}
which, in fact, 
turns out to be larger than necessary. 
From the discussion above we have learned that the main function of the non-trivial Pauli twirl is to set up a random permutation of the Pauli operators. 
This function is already fulfilled by the $3$ element cyclic subgroup $\langle \hat{H}\hat{S}^{\dagger}\rangle$, so that only a total of $12$ elements are necessary to consider to build a single qubit Clifford twirl.  

In fact, it is easy to verify that the set
\begin{equation}
 S=\langle \hat{H}\hat{S}^{\dagger}\rangle \CP=\{ CP  \vert \; C\in\langle \hat{H}\hat{S}^{\dagger}\rangle, \; P\in \CP \} 
\end{equation}
is a Clifford 2-design. 
This is verified either by checking that it correctly twirls a Pauli-diagonal channel into a depolarizing channel from its transitivity over the Pauli operators or by verifying an equivalent condition proven in ref.~\cite{Gross_2007}, which is that the 
\textit{frame potential} evaluates to a value of $\CF=2$, proving that one indeed encounters an exact unitary 2-design.
In fact, the group stated above has the minimal cardinality of $(d^2-1)d^2$ shown to yield a Clifford 2-design \cite{Gross_2007, Steve_comment}, which, in our construction, comes from the fact that $\langle \hat{H}\hat{S}^{\dagger}\rangle$ is the minimal Pauli transitive subgroup. 
For any dimension, a Pauli transitive subgroup has at least $d^2-1$ elements, such that a Clifford design of the type proposed above has at least $(d^2-1)d^2$ elements, matching the bound conjectured ref.~\cite{Gross_2007} and later proven to hold in ref.~\cite{Steve_comment}. 

\paragraph{POVM twirling.} 

A special application of the channel twirl -- and at the same time a most important one for the purposes of this work --
is found when it is applied to a POVM.  This setting has been recently popularized by showing how it can be used for 
\textit{shadow tomography} \cite{aaronson2018shadow,  Huang_2020, Chen_2021,RBRobustShadows} giving rise to various \emph{noisy-intermediate-scale-quantum} (NISQ) friendly applications.  
On a single mode consider the POVM representing a 
computational basis measurement
\begin{equation}
\CM_Z=\sum_{z\in \Z_2} \sketbra{\Pi_z},\;  \Pi_z=\ketbra{z}.
\end{equation}
The corresponding $\chi$ matrix representation has $\chi_{0,0}=1/2$,  such that we can compute
\begin{equation}
\CM := \Pi_{\Cl\lr{2}}\circ \CM_Z = \lr{\rho + I} /3. 
\end{equation}
The twirled POVM is a depolarizing channel and is (as a matrix,  
not physically as a quantum channel) 
invertible with $\CM^{-1}\lr{X}=3X-I$. Furthermore,  
its action on state vectors $\sket{\rho}$
 \begin{equation}
 \CM\sket{\rho}=\frac{1}{|\Cl\lr{2}|}\sum_{C\in \Cl\lr{2} \\ z \in \Z_2} \CC \sketbra{\Pi_z} \CC^{\dagger} \sket{\rho}
 \end{equation}
can be interpreted as protocol to decompose arbitrary quantum state vectors $\sket{\rho}$ into stabilizer state vectors $\CC \sket{\Pi_z}$. Note that, as we have done earlier as well, we use the calligraphic symbols for unitary channels in place of $\CC=C \otimes C^*$.  An experimental protocol to reconstruct (properties of) the state is hence identified by realizing that measuring in the computational basis after applying random Clifford gates to an input state allows for a reconstruction
\begin{equation}
\sket{\rho}=\CM^{-1}  \mathbb{E} \lrq{ \CC_i \sket{\Pi_{z_i}}}, \label{eq:approx_state}
\end{equation}
where the Cliffords $C_i$ are picked uniformly from the Clifford group and $z$ is determined by the Born rule $ z_i \sim  \sbraket{\Pi_z | \CC^{\dagger} | \rho}$.  Similarly,  expectation values of observables can also be estimated as
\begin{equation}
\sbraket{O|\rho}=  \sbra{\CM^{-1} \lr{O}} \mathbb{E} \lrq{ \CC_i \sket{\Pi_{z_i}}},
\end{equation}
where we we also used that $\CM^{-1}$ is self-adjoint. 
This protocol is particularly appealing in practical NISQ-era questions for two reasons.  1.  the projective nature of the channel twirl projects noisy versions of the measurement $\CM_Z$ onto $\CM$ (possibly with adapted coefficients \cite{Chen_2021}) and 2.  a relatively small numbers of samples from the distribution over $(C,  z)$ for a given $\rho$ allows to estimate expectation values of exponentially many observables to high confidence \cite{Huang_2020},  where the samples $(C, z)$ can be generated in a quantum experiment without yet having decided on the observable $O$.  Due to the design property of the twirl the partial tomographic data obtained this way suffices to estimate selected observables in purely classical post-processing.  

For applications in quantum computation with bosonic quantum error correction,  such as using the GKP code,  the shadow tomography protocol is particularly interesting since,  typically,  measurements of logical observables can only be carried out indirectly using more naturally accessible measurements such as homodyne detection, heterodyne detection, or photon counting and do not admit a simple and direct physical measurement procedure.  
The projection property (1.) of the logical Clifford-twirl allows us to naturally use more accessible measurements that may be badly aligned with the observables of interest and prescribe how to classical post-process the results to estimate the logical observable at hand.  
Furthermore,  the effective decomposition of the state \eqref{eq:approx_state} obtained from the shadow protocol is interesting as it may give rise to new representations of states in phase space that capture core logical information.  After discussing how a Clifford channel-twirl can be set up for the GKP code,  we will see how the shadow protocol gives rise to an approximation of GKP states using Gaussian states by logical Clifford twirling a heterodyne measurement.  We expect this representation to be particularly useful in the development of new simulation methods for GKP error correction.

\subsection{Displacement twirling}

In this subsection, we turn to discussing twirls that can 
be realized by implementing appropriate displacements in the 
physical Hilbert space. In ref.~\cite{Conrad_2021}, approximate twirls over groups of displacement operators distributed over lattices $\CL^{\perp}$ associated to the GKP code have been constructed by approximating the uniform measure over the (infinite) lattice $\CL^{\perp}$ via a random walk over a generating set given by the rows of $M^{\perp}=\lr{\bs{\xi}^{\perp\, T}_1,\hdots, \bs{\xi}^{\perp\,  T}_{2n}}^T$.  
%
Concretely,  we define a random walk from the joint distribution of $N'=2N$ half-steps $\pm \bs{\xi}_i / 2$,  each of which are selected with $1/2$ probability at each step.  Define for $i=1,\hdots, 2n$ the 
associated (discrete) measure \begin{equation}
\mu'_{i}\lr{\bs{x}} =\frac{1}{2}\delta \lr{\bs{x}-\bs{\xi}^{\perp}_i/2}+\frac{1}{2}\delta \lr{\bs{x}+\bs{\xi}^{\perp}_i/2},
\end{equation}
so that we obtain the measure corresponding to $m$ steps of the 
random walk as $\mu_i^{(*m)} :=\mu_i'^{(*2m)} $.

\paragraph{State twirling.} 
Applying this twirl to a state,  we obtain that
\begin{align}
\rho &=\int_{\R^2} d\bs{\alpha}\, \rho\lr{\bs{\alpha}} D\lr{\bs{\alpha}}  
\mapsto \int_{\R^2} d\mu_i^{(*m)}\lr{\bs{\gamma}} \, D\lr{\bs{\gamma}} \rho D^{\dagger}\lr{\bs{\gamma}}
=\int_{\R^2} d\bs{\alpha} \,  \lrq{\nu_i\lr{\bs{\alpha} }}^m \rho\lr{\bs{\alpha}} D\lr{\bs{x}}  \label{eq:state_twirl}
\end{align}
modifies the characteristic function of the state with the $m-$th power of the symplectic Fourier transform of the measure 
\begin{align}
\nu_i\lr{\bs{\alpha}}&= \int_{\R^2} d\mu_i \lr{\bs{\gamma}} e^{-i2\pi \bs{\gamma}^TJ\bs{\alpha}} 
=\cos^2\lr{\pi \lr{\bs{\xi}_i^{\perp}}^TJ\bs{\alpha}}.
\end{align}
In the limit $m\rightarrow \infty $, this suppresses all contributions $\bs{\alpha}$ except for those in the symplectic dual of $\bs{\xi}_i^{\perp}$.  We define the joint measure over all generators in $M^{\perp}$ to be the joint random walk given by the $2n$-fold convolution
\begin{equation}
\mu_{M^{\perp}} = \mu_1 * \mu_2 *\hdots *\mu_{2n}
\end{equation}
which has  the Fourier transform
\begin{equation}
\nu_{M^{\perp}}\lr{\bs{\alpha}}=\prod_{i=1}^{2n} \cos^2\lr{\pi \lr{\bs{\xi}_i^{\perp}}^TJ\bs{\alpha}}.
\end{equation}
The total effect of this twirl is that in the limit $m\rightarrow \infty$ only logically trivial contributions $\bs{\alpha} \in \CL$ will survive while all other contributions are exponentially suppressed and the state becomes logically fully depolarized.  An alternative view is that in this limit eq.\  \eqref{eq:state_twirl} converges to the group projector of the group generated by the displacement in $\CL^{\perp}$.  The commutant of this group is the stabilizer group generated by displacements in $\CL$.  Non-trivial displacements of $\CL^{\perp}$ are not in this commutant such that it cannot carry logical information.  While twirling a state over $\CL^{\perp}$ does not appear to bear any interesting applications outside deliberate logical depolarization of the state~\footnote{Except perhaps for error mitigation methods where tunable noise is desired as in zero-noise extrapolation \cite{BenjaminZNE,  TemmeZNE}} note that the above outlined method of state twirling is not restricted to using generators in $M^{\perp}$. 
A stabilizer twirl using generator $M$ can equally be used to (approximately) project the state onto one one with a characteristic function supported only on $\CL^{\perp}$.  In conjunction with a twirl over a set of Gaussian unitaries representing a set of logical Clifford gates such as the logical $\hat{H}$ for the square GKP code or $\hat{H}\hat{P}^{\dagger}$, we expect this technique to be useful for the measurement-less preparation 
of \emph{magic states} \cite{PhysRevA.71.022316}, 
analogous to previous proposal for entanglement distillation \cite{Bennett_1996} as well as and the preparation of entangled GKP states analogous to the procedure in refs.~\cite{Werner89, Bennett_1996}.

\paragraph{Channel twirling.} 
%
Acting on a channel (see ref.~\cite{Conrad_2021}) 
\begin{equation}
\CC = \int d\bs{\alpha} d\bs{\beta} \,   c\lr{\bs{\alpha},  \bs{\beta}} D\lr{\bs{\alpha}} \otimes  D^*\lr{\bs{\beta}}
\end{equation}
with chi-function $c\lr{\bs{\alpha},  \bs{\beta}}$,  the $m-$fold displacement channel twirl using our measure $\mu_{M}^{\perp}$ implements the action on the chi-function
\begin{equation}
c\lr{\bs{\alpha},  \bs{\beta}} \mapsto \lrq{\nu_{M^{\perp}}\lr{\bs{\alpha}-\bs{\beta}} }^m c\lr{\bs{\alpha},  \bs{\beta}}.
\end{equation}
Similar to the above,  this channel twirl approximately projects the channel onto a channel where non-stabilizer coherences are suppressed,  i.e.,  contributions in the chi-function $c\lr{\bs{\alpha},  \bs{\beta}}$,  for which $\bs{\alpha}-\bs{\beta} \not\in \CL$ become exponentially suppressed as $m\rightarrow \infty$.  We visualize the factor $\nu_{M^{\perp}}$ for the square-  and hexagonal GKP code 
in fig.~\ref{fig:nuMp}.

For finite strength $m$ of the twirl, the error can be bounded as follows. Let 
\begin{equation}
\overline{\nu}_{\CL}\lr{\Delta} :=\lim_{m\rightarrow \infty} \nu_{M^{\perp}}\lr{\Delta}^m,
\end{equation}
in this limit the function is independent of the choice of dual generating set $M^{\perp}$, which is why the index has been replaced by $\CL$.
For any finite $m$, we have
\begin{align}
    \nu_{M^{\perp}}\lr{\Delta}^m 
    &= \overline{\nu}_{\CL^{\perp}}\lr{\bs{\Delta}} + \nu_{M^{\perp}}\lr{\bs{\Delta} }^m\lrq{1-\overline{\nu}_{\CL^{\perp}}\lr{\bs{\Delta}}}.
\end{align}
To bound the error term, observe that the contribution $\lrq{1-\overline{\nu}_{\CL^{\perp}}\lr{\bs{\Delta}}}$ is only non-zero when $\bs{\Delta} \not\in \CL$. Let $\bs{x}={\rm CVP}\lr{\Delta, \CL}$ be the closest vector in $\CL$ to $\bs{\Delta}$ and let $\bs{\delta}=\bs{\Delta}-\bs{x}$ be the corresponding minimal vector between $\bs{\Delta}$ and the lattice. Assuming $\bs{\delta}$ is small, we can bound each cosine term 
\begin{align}
    \cos^2\lr{\pi \lr{\bs{\xi}_i^{\perp}}^TJ\Delta}
    &= e^{-\pi^2\|\lr{\bs{\xi}_i^{\perp}}\|^2\|\bs{\delta}\|^2}+O\lr{\|\bs{\delta}\|^4},
\end{align}
such that, in total, we have for $\bs{\Delta} \not\in \CL$ close to the lattice
\begin{equation}
    \nu_{M^{\perp}}\lr{\Delta}^m = e^{-\pi^2\|M^{\perp}\|_F^2\|\bs{\delta}\|^2 m }+O\lr{\|\bs{\delta}\|^{4m}},
\end{equation}%
where $\|\cdot\|_F$ denotes the Frobenius norm and $\bs{\delta}=\bs{\Delta}- {\rm CVP}\lr{\bs{\Delta}, \CL}$ is the minimal distance between $\bs{\Delta}$ and $\CL$.


Note that in the above construction we have decided to work with the twirl induced by the described random walk because it yielded a particularly nice analytic form for the characteristic function
\begin{equation}
    \nu_{M^{\perp}}\lr{\bs{x}} = \int_{\R^{2n}} d\mu_{M^{\perp}}\lr{\bs{\gamma}} e^{-i2\pi \bs{\gamma}^TJ\bs{x}}.
\end{equation}
There is no other particularly good 
reason to use this parametrization. In general, 
one may 
also choose an arbitrary regularizer $R\lr{\bs{\gamma}}$ to regularize a uniform distribution over the lattice $\CL^{\perp}$ as \begin{equation}
    d\mu\lr{\bs{\gamma}}=d\bs{\gamma}\,R\lr{\bs{\gamma}} \sum_{\bs{\xi}^{\perp} \in \CL^{\perp}} \delta^{(2n)}\lr{\bs{\gamma} - \bs{\xi}^{\perp}}.
\end{equation}
By the properties of the Fourier transform the result will be provided by the convolution of the asymptotic characteristic with the Fourier-transform of the regularization $\hat{R}$, which will 
be more localized the more homogeneous $R$ is,
\begin{equation}
    \nu\lr{\bs{\Delta}}= \lr{\hat{R} * \overline{\nu}_{\CL}\lr{\bs{\Delta}}}.
\end{equation}


\begin{figure}
\center
\begin{tikzpicture}[scale=1.1]
\node at (0,0){\includegraphics[width=4cm]{nu_Sq.pdf}};
\node at (4,0){\includegraphics[width=4cm]{nu_Hex.pdf}};
\node at (7,0){\includegraphics[width=1.6cm]{colorbar_nu.pdf}};
\end{tikzpicture}

\caption{Figure for $\nu_{M^{\perp}}$ for square and hexagonal GKP codes.} \label{fig:nuMp}
\end{figure}


\subsection{Gaussian unitary twirling}

To  twirl a channel diagonal in (equivalence classes of) logical displacement operators into a logical Pauli-diagonal channel with uniform support over the Pauli operators we implement a random logical Clifford operation as a Gaussian unitary twirl parametrized by a symmetric measure over the symplectic matrices $\mu\lr{S}=\mu\lr{S^{-1}}$,  which implements the map on the displacement twirled chi-function
\begin{align}
\lrq{\nu_{M^{\perp}}\lr{\bs{\alpha}-\bs{\beta}} }^m c\lr{\bs{\alpha},  \bs{\beta}} \nonumber 
&\mapsto \int d\mu\lr{S}  \lrq{\nu_{M^{\perp}}\lr{S^{-1}\lr{\bs{\alpha}-\bs{\beta}}} }^m
c\lr{S^{-1}\bs{\alpha},  S^{-1}\bs{\beta}}  \nonumber\\
&= \int d\mu\lr{S}  \lrq{\nu_{M^{\perp}S^T}\lr{\bs{\alpha}-\bs{\beta}}}^m 
\end{align}
Although such random unitary modifications will never truly project a channel onto one that acts solely within code space,  our goal is to modify it such that for any logical input state and any channel,  decoded logical readout will make it appear as if the twirled channel was a full logically depolarizing.  I.e.,  we target that the projection of the channel onto code space to behave like a logically depolarizing channel.

Note that for the explicit examples we consider here $n=1$ is small enough to simply enumerate the corresponding logical groups $\Sp_2\lr{\Z_d}$ and sample directly from those sets. Nevertheless, we briefly outline how the twirl works in larger systems where the size of $\Sp_2\lr{\Z_d}$ grows super-exponentially. The idea is to choose a generating set $G$ for the symplectic automorphism group $\CG=\Aut^S\lr{\CL^{\perp}}$ and generate random group elements by choosing random sequence from this generating set. It is a result from the theory of random matrices \cite{Varju, DiaconisShashashahani} that a close-to uniform distribution over the target group is obtained by performing at least $k$ steps, once $k\geq |G|\mathrm{diam}^2\lr{\CG,  \,G}$, where $|G|$ is the size of the generating set and the group diameter $\mathrm{diam}\lr{\CG,  \,G}$ expresses the minimal number of generators needed to express any element in the group $\CG$.

For $\CG=\Sp_{2n}\lr{\Z_d}$ with generating set given by elementary generalized Hadamard-,  phase- and CNOT gates  (see appendix), which in their real representation are given by symplectic transvections, this evaluates to a bound $k\geq O\lr{d^2n^6}$.   
This bound can be slightly improved to $k\geq O\lr{d n^6}$ when adding the $r=0, \hdots, d-1$ powers of the elementary gates.
We hence obtain that a logical twirl over the full (non-trivial) logical Clifford can be realized by random sequences of $k\geq O\lr{d^2n^6}$ elementary symplectic transvections.

\section{GKP logical shadows}\label{sec:GKPShadows}
In the previous section we have discussed how a general CV channel $L$ can be twirled using displacement operators and Gaussian unitary operations such that it effectively yields a logical depolarizing channel. 
We here turn to showing how these insights and the
established machinery can be turned
into a scheme devising GKP logical shadows for the efficient measurement of expectation values of observables on the logical level.
The key to this analysis is to develop an understanding of how the physical twirl manifests itself on the logical level, where the conversion from a physical representation of the state to its logical information content is facilitated by a decoder.

\subsection{From physical to logical twirls}

For any physical channel $\CC$,  denote by $\CC^{\tau}$ the channel twirled using logical displacements from $\CL^{\perp}$ and Gaussian unitaries represented by $\Aut^S\lr{\CL^{\perp}}$ as discussed earlier.
We define a \textit{decoder} $\Dec$ to be an surjective map from physical states to logical,  error-free states $\Dec\sket{\rho}=\sket{\overline{\rho}}$.  The decoder is required to commute with noiseless logical channels $\Dec\CC=\CC\Dec$,  where $\CC$ are logical Clifford operations (represented by a Gaussian unitary channel) and reduce error channels $\CR$ compactly supported on displacements from the Voronoi cell $\mathcal{V}\lr{\CL^{\perp}}$ 
\begin{equation}
\mathcal{V}\lr{\CL^{\perp}}:=\lrc{\bs{x}\in \R^{2n},  \|\bs{x}\|\leq \|\bs{x}-\bs{\xi}^{\perp}\|\; \forall \bs{\xi}^{\perp} \in \CL^{\perp}\setminus\lrc{0}}
\end{equation}
to the logical identity channel $\Pi=\Pi_{\CL}\otimes\Pi_{\CL}^*$,  $\Pi\CR= \Pi$,  where $\Pi_{\CL}$ is the code space projector. 

Applying the decoder to the twirled channel,  we obtain for any 
input state $\rho$
\begin{align}
 \Dec\CC^{\tau} \sket{\rho}&=\alpha \Dec\sket{\rho} + \beta \sket{\Pi_{\CL} }
 \nonumber
 \\
 &=\alpha\sket{\overline{\rho}}+\beta \sket{\Pi_{\CL} }:=\widetilde{\CM}\sket{\overline{\rho}},\label{eq:DTwirled}
\end{align}
that the decoded twirled channel effectively behaves like a logically depolarizing channel.  The depolarizing channel is invertible as $\widetilde{M}^{-1}\lr{X}=\lr{X-\beta\Pi_{\CL}}/\alpha$ as long as $\alpha\neq 0$.  
Let $\CC=\sum_i\sketbra{\Pi_i}$ be a POVM. Equation~\eqref{eq:DTwirled} takes the form
\begin{equation}
\widetilde{\CM}\sket{\overline{\rho}}= \Dec  \sum_i\int_{\rm GU\lr{n; \CL}} d\mu\lr{U} \,  \CU\sket{\Pi_i} \sbraket{\Pi_i|\CU^{\dagger}|\rho},
\end{equation}
where the unitaries are drawn from a distribution over Gaussian unitary operations representing logical Clifford gates obtained from concatenating the displacement- and non-trivial logical Clifford ${\rm GU\lr{n; \CL}}$, as discussed in the previous sections. Since $\widetilde{\CM}$ commutes as logical channel with $\Dec$,  given a physical state $\sket{\rho}$,  we can obtain the expectation value of any logical observable $O$ via the following shadow protocol.
\begin{enumerate}
\item Sample a set of $\lrc{\CU_i}_{i=1}^N$ from ${\rm GU\lr{n; \CL}}$ and perform the POVM for each sample.  The probabilities of obtaining outcome pointers $\sket{\Pi_i}$ are each given by the Born probability $\sbraket{\Pi_i| \CU^{\dagger}_i | \rho}$.
\item Reconstruct the shadow as empirical expectation value over the output states \begin{equation}\sket{S}=N^{-1}\sum_{i=1}^N  \CU_i \sket{\Pi_i}.  \label{eq:shadowstate}\end{equation}
\item Compute the logical expectation value \begin{equation}\sbraket{O|\widetilde{M}^{-1}|S}=\sbraket{\widetilde{M}^{-1}\lr{O}|S}.\end{equation}
\end{enumerate}
Since 
\begin{equation}
\Dec\mathbb{E}\lrq{ \widetilde{M}^{-1} \CU \sket{\Pi}} = \Dec\sket{\rho},  \label{eq:Drho_shad}
\end{equation}
for sufficiently large sample size the empirical mean state converges to the true mean state and the procedure recovers the expectation value of the logical observable as if it was measured directly after implementing a suitable decoder, that is, 
the estimator 
\begin{equation}
\tilde{o}=\sbraket{\widetilde{M}^{-1}\lr{O} | \CU| \Pi}
\end{equation}
inherits the mean $\langle \tilde{o}\rangle=\sbraket{O|\Dec|\rho}$.

It is key to the protocol, however, that not only the mean is correctly recovered, but that also the variance is small: on this logical level the estimation essentially does not differ from the typical qubit scenario as already detailed in the construction provided by Huang,  Kueng and Preskill \cite{Huang_2020} and their performance guarantees apply,  in that a small number of samples suffices to reproduce the expectation value of many observables with high confidence.

\begin{them}[HKP \cite{Huang_2020}]
A collection of $NK$ samples $\lrc{\lr{\CU_i , \Pi_i }}_{i=1}^{NK}$ produced via the above protocol from a CV state $\rho$ suffice to estimate logical expectation values on observables $O_i,\; i=1, \hdots , M$  via median of means prediction up to $\epsilon$ additive error provided that
\begin{equation}
K=2\log\lr{2M/\delta},\; N=\frac{34}{\epsilon^2} \max_i \biggl\|O_i-\frac{\Tr\lrq{O_i}}{2^n} I \biggr\|^2_{\rm shadow}
\end{equation}
with probability at least $1-\delta$. 
\end{them}
%See 
We refer to ref.~\cite{Huang_2020} for the proof and the definition of the shadow norm, 
for which ref.\ \cite{Huang_2020} 
has also provided the upper bound 
\begin{equation}
\biggl\|O_i-\frac{\Tr\lrq{O_i}}{2^n} I \biggr\|^2_{\rm shadow}\leq 3\Tr\lrq{O_i^2}.
\end{equation}
Median of means prediction 
is carried out by dividing the set of $NK$ samples into $K$ batches of $N$ samples each,  for each of which the arithmetic mean is computed  and taking the median over the batches
\begin{equation}
\tilde{o}^{\rm est} = {\rm median} \lrc{ N^{-1}\sum_i \tilde{o}_{jN+i} }_{j=1,\hdots
, K}.
\end{equation}
In particular,  this also implies that the reconstructed state \eqref{eq:shadowstate} yields a good representation of the decoded state $\Dec\sket{\rho}$ in that it reproduces the expectation value many low-rank observables with only small additive error.  By paying an extra cost in sample overhead,  we can use the shadow to obtain a full representation of the (finite dimensional)
state.

\begin{them}[Full representation of the 
state]\label{them:GKPshadow}
Let $\CL \subset \R^{2n}$ denote the lattice corresponding to a scaled GKP code on $n$ modes that encodes $d^n$ logical dimension and let $\mu$ denote a measure over elements $\Aut^S\lr{\CL^{\perp}}$ forming a logical Clifford $2-$design and let $\CM=\SumInt dz \sketbra{\Pi\lr{z}}$ denote a physical POVM.   Let $\sket{\rho}$ be an arbitrary state vector on the $n$-mode Hilbert space. The state vector
\begin{equation}
\sket{S}=N^{-1} \sum_{i=1}^N \CU_{S,  i}\sket{\Pi\lr{z_i}}
\end{equation}
produced by sampling $N$ Gaussian unitary operations via the measure $\mu$ and measurement outcomes from the Born distribution $z_i \sim \sbraket{\Pi\lr{z}|\CU_{S,  i}^{\dagger}|\rho}$ approximates the logical value of the state in Hilbert-Schmidt distance

\begin{equation}
d_{\rm HS}\lr{\Dec \sket{\rho} ,  \Dec \sket{S}} \leq \delta_{\rm HS}^2
\end{equation}
with probability at least $1-\delta$ for 
\begin{equation}
N\geq \frac{2d^{2n}}{\alpha^2 \delta_{\rm HS}^2} \lrq{\ln\lr{\frac{2}{\delta}} + 2n \ln\lr{d}},
\end{equation}
where $\alpha$ is determined by the commutation of the twirled 
POVM with the decoder
\begin{align}
 \Dec\CM^{\tau} &=\alpha \Dec + \beta \sket{\Pi_{\CL} }.
\end{align}
In particular, 
we also have
\begin{equation}
\|\Dec \sket{\rho}-  \Dec \sket{S}\|_1 \leq d^{\frac{n}{2}}\frac{\delta_{\rm HS}}{2} . %\label{eq:thm3_tr}
\end{equation}
\end{them}%
See appendix \ref{app:GKPshadow_proof} for the proof of this statement. Although the sample overhead derived here contains the extra dimensional factor $O\lr{d^{2n}}$ and is therefore
rather large compared to the usual situations considered in ref.~\cite{Huang_2020}, this statement allows us to apply the shadow tomography toolbox to derive simple representations of bosonic states with the right information content on selected (GKP) subspaces. We exemplify its utility using the heterodyne- and photon-click 
POVM that resolve the presence and absence of photons.

\subsection{Gaussian decomposition of the hexagonal GKP code from heterodyne measurements}

In our convention for displacement operators,  we define generalized coherent state vectors as $\ket{\bs{\alpha}}=D\lr{\bs{\alpha}}\ket{0}$,  where $\ket{0}$ denotes a $n$-mode vacuum state where we have
\begin{align}
\ketbra{\bs{\alpha}}&=\int_{\R^{2n}} d\bs{\beta}\,  \Tr\lrq{D^{\dagger}\lr{\bs{\beta}} \ketbra{\bs{\alpha}}} D\lr{\bs{\beta}} \\
&=\int_{\R^{2n}} d\bs{\beta}\,   e^{-\frac{\pi}{2} \bs{\beta}^T\bs{\beta}-i2\pi\bs{\alpha}^TJ\bs{\beta}}D\lr{\bs{\beta}} ,
\nonumber
\end{align}
such that a resolution of the identity is given by
\begin{equation}
\int_{\R^{2n}}d\bs{\alpha}\, \ketbra{\bs{\alpha}} = I.
\end{equation}
In this sense, the coherent states constitue an overcomplete resolution of the identity.
We also have
\begin{equation}
\braket{\bs{\beta}|\bs{\alpha}}=e^{-\frac{\pi}{2}\lr{\|\bs{\alpha}-\bs{\beta}\|^2+i2\bs{\alpha}^TJ\bs{\beta}}}.
\end{equation}
Generalized heterodyne measurements are interferometric quantum optical measurements
that are known to in effect implement a POVM described by the quantum channel
\begin{align}
\CC&=  \int d\bs{\alpha}  \ketbra{\bs{\alpha}} \otimes  \ketbra{\bs{\alpha}}^* \\
&=\int d\bs{\beta} e^{-\pi \bs{\beta}^T\bs{\beta}} D\lr{\bs{\beta}} \otimes D\lr{\bs{\beta}}^*,
\nonumber
\end{align}
equivalent to a Gaussian displacement channel where displacements of amplitude $\bs{\beta}$ occur with probability density $ e^{-\pi \bs{\beta}^T\bs{\beta}}$. 
In this form it becomes clear why this POVM is special.  1.  it is already diagonal in displacement operators,  such that a displacement twirl has no effect and 2.  the chi-function only depends on the Euclidean length $\|\bs{\beta}\| $ of the corresponding displacement amplitudes,  such that the channel is also invariant under Gaussian unitary twirls realized via orthogonal symplectic transformations of $\bs{\beta}$. 

For the single mode hexagonal GKP code we have found that a Clifford $2$ design is given by the set of rotations $R_{\frac{2\pi}{3}}$,  which are such orthogonal symplectic transformations.  Hence this is an example where the twirls have no effect.  
We can define a sequence of \textit{Voronoi-``shells''}
\begin{align}
\mathcal{V}^0&=\mathcal{V}\lr{\CL^{\perp}},\\
\mathcal{V}^k&=\mathcal{V}\lr{(2k+1)\CL^{\perp}}\setminus \mathcal{V}^{k-1}, \; k\in \N_0,\\
\bigcup_{k=0}^{\infty}\mathcal{V}^k&=\R^{2n},
\end{align}
where each shell with $m=0\, \mod d$ contains logically trivial displacement amplitudes.  
In particular  $ \forall \bs{x}\in \mathcal{V}\lr{\CL^{\perp}}$ our decoder maps
\begin{equation}
\Dec \CD\lr{\bs{x}} = \Dec.
\end{equation}
For $d=2$, 
we can 
compute
\begin{align}
\Dec \CC &=\Dec\sum_{k=0}^{\infty} \int_{\mathcal{V}^k}d\bs{\alpha}\, e^{-\pi\bs{\alpha}^T\bs{\alpha}} \\
\nonumber
&=(p_0-p_1)\Dec + 2p_1\Pi_{\CL}, \\ \hspace{.5cm}
p_i&=\sum_{k=0}^{\infty} \int_{\mathcal{V}^{2k+i}}d\bs{\alpha}\, e^{-\pi\bs{\alpha}^T\bs{\alpha}}
\end{align}
where we have used that each displacement in an even shell is removed by the decoder and each displacement in an odd shell is equally likely attributed to a logical Pauli $X-,\, Y-,$ or $Z-$ displacement.  Here we have 
that the associated logical depolarizing channel $\widetilde{M}=(p_0-p_1)\Dec + 2p_1\Pi_{\CL}$ is invertible as long as $p_0\neq p_1$.  


Since we have $p_0+p_1=1$,  this is can be verified by showing that $p_0\neq 1/2$. 
We have $\mathcal{V}^0\supseteq \CB^2\lr{\rho\lr{\CL^{\perp}}}$ and for $k\geq 1$
\begin{equation}
\mathcal{V}^k \subseteq \CB^2\lr{\lr{2k+1}\mu\lr{\CL^{\perp}}} \setminus \CB^2\lr{\lr{2k-1}\rho\lr{\CL^{\perp}}},
\end{equation}
where $\rho\lr{L}=\lambda_1\lr{L}/2$ denotes the packing radius of the lattice $L$ and $\mu\lr{L}$ denotes the covering radius,  which for the hexagonal lattice is given by $\mu\lr{A_2}=\lambda_1\lr{A_2}/\sqrt{3}$ where $\lambda_1 \lr{A_2}=\sqrt{2/\sqrt{3}}$.
Using $\lambda_1\lr{\CL^{\perp}}=\lambda_1\lr{A_2}/\sqrt{2}$ we can thus bound
\begin{align}
p_0&=\sum_{k=0}^{\infty} \int_{\mathcal{V}^{2k}} e^{-\pi \|\bs{\alpha}\|^2} \leq  2\pi\int_0^{\mu\lr{\CL^{\perp}}} r e^{-\pi r^2}  \\
\nonumber
&\hspace{1cm}+  2\pi \sum_{k=1}^{\infty} \int_{(4k-1)\lambda_1\lr{\CL^{\perp}}/\sqrt{3}}^{(4k+1)\lambda_1\lr{\CL^{\perp}}/2} dr\, r e^{-\pi r^2}  \\
\nonumber
&=\lrq{1-e^{-\frac{\pi}{3\sqrt{3}}}} \\ 
\nonumber
&\hspace{.5cm} + \sum_{k=1}^{\infty} \lr{e^{-\frac{\pi}{3\sqrt{3}}(4k-1)^2\lambda_1^2} - e^{-\frac{\pi}{4\sqrt{3}}(4k+1)^2\lambda_1^2}}\\
&\approx 0.455\hdots \, .
\nonumber
\end{align}
This yields a bound $|\alpha|=|p_0-p_1|=|2p_0-1| \geq 0.09$.
We find that while the effective channel $\widetilde{M}$ is in fact invertible,  the logical information retained in the depolarized state is rather minuscule as indicated by the value $|\alpha|=|p_0-p_1|=|2p_0-1| \geq 0.09$. 


\subsection{Decomposing GKP states from photon click detectors}

Homodyne or 
(generalized) heterodyne measurements are by no means the only common 
and practically feasible measurements in quantum optical measurements.
Possibly even more common are 
photon click detectors such as being realized by avalanche photon detectors \cite{Cova2004}.
Photon click detectors have dichotomic outcomes and distinguish the presence from the absence of photons.
We examine for this reason
the POVM
\begin{align}
\CC&=\Pi_0\otimes \Pi_0^* + \Pi_{1} \otimes \Pi_1^*, \\
\Pi_0&=\ketbra{0},\\
\Pi_{1}&=I-\ketbra{0},
\end{align}
with chi-function 
\begin{align}
c\lr{\bs{\alpha}, \bs{\beta}}&=e^{-\frac{\pi}{2} \|\bs{\alpha}\|^2-\frac{\pi}{2} \|\bs{\beta}\|^2} \\ 
&\hspace{.5cm}
+\lr{\delta\lr{\bs{\alpha}}- e^{-\frac{\pi}{2} \|\bs{\alpha}\|^2}}\lr{\delta\lr{\bs{\beta}}- e^{-\frac{\pi}{2} \|\bs{\beta}\|^2}}.\nonumber
\end{align}
As before, this channel is rotationally invariant, such that the only non-trivial component of the twirl is given by its diagonalizing action on the chi function. Note that the wave function collapse induced by projective measurement of a generating set of GKP stabilizers is equivalent a full stabilizer displacement twirl (with $m\rightarrow\infty$). Ignoring stabilizer coherences, 
the channel thus takes the form
\begin{align}
 \CC^{\tau} \lr{\rho} &=   \int d^2\bs{\alpha} f\lr{\bs{\alpha}} D\lr{\bs{\alpha}} \rho D^{\dagger}\lr{\bs{\alpha}}, \\
 f\lr{\bs{\alpha}}&= \sum_{\bs{\xi} \in \CL} c\lr{\bs{\alpha},\bs{\alpha}+\bs{\xi}}e^{-i\pi\bs{\alpha}^TJ\bs{\xi}}.
\end{align}
Commuted through a decoder, the effective depolarizing channel obtains the coefficients
\begin{align}
    \alpha&=1-2\vartheta\lr{0 \, | \, iG_{A_2}}+I_1
    ,
    \\
\alpha+\beta &= 1-2\vartheta\lr{0 \, | \, iG_{A_2}}+I_2,
\end{align}
where we have defined the Riemann theta function as
\begin{equation}
\vartheta\lr{\bs{z}\, | \, F}=\sum_{\bs{m}\in \Z^{2n}} e^{i2\pi\lr{\frac{1}{2}\bs{m}^TF\bs{m}+\bs{m}^T\bs{z} }}
\end{equation}
and 
\begin{align}
    I_1&=\sqrt{2} \int_{\mathcal{V}\lr{A_2}}d\bs{\alpha'} e^{-\frac{\pi}{2}\|\bs{\alpha'}\|^2} \vartheta \lr{M_{A_2}\lr{iI+J}\bs{\alpha'}| i G_{A_2} }\nonumber\\
&\hspace{2cm}
\times\vartheta \lr{M_{A_2}\lr{iI-J}\bs{\alpha'}| i G_{A_2} }
,\\
    I_2 &= 2\sqrt{2} \int_{\mathcal{V}\lr{A_2}}d\bs{\alpha'} e^{-2\pi\|\bs{\alpha'}\|^2} \vartheta\lr{M_{A_2}\lr{iI+J}\bs{\alpha'}| i G_{A_2} } \nonumber\\
    &\hspace{2cm}
    \times
    \vartheta \lr{M_{A_2}\lr{iI-J}\bs{\alpha'}| i G_{A_2} },
\end{align}
where $G_{A_2}=M_{A_2}M_{A_2}^T$ is the Euclidean Gram matrix for the symplectic basis of the $A_2$ lattice.
By numerical integration and approximating the Riemann theta function with one evaluated using a truncated sum $\bs{m}\in \lrc{-t,\hdots ,t}^{2n}$ with $t$ chosen large enough for convergence of the output values, we evaluate $\vartheta\lr{0 \, | \,i G_{A_2}}=1.1596$, $I_1=1.493$, $I_2=1.64$ to obtain the estimates $\alpha=0.32$ and $\beta=0.15$. Similar to the previous section, this shows that a representation of the logical content of a single mode GKP code relative to the hexagonal GKP code can be obtained from a finite number of samples from random displaced photon click detectors, of which the number is bounded by theorem \ref{them:GKPshadow}. 

In this section, 
we have adapted the (HKP classical) shadow tomography protocol introduced in 
ref.~\cite{Huang_2020} to operate on the logical degrees of freedom of quantum information encoded via a GKP code in a continuous variable system and considered two of the best-behaved POVMs given by heterodyne detection and a photon-click detector and derived the constant $\alpha$ to show the invertibility of the effective depolarizing channel. While the full analysis is more complex for more general POVMs, such as photon counting, our approach is sufficiently general to be generalized to such situations.

\subsection{Applications to classical simulation}

We briefly speculate about applications of the logical shadow tomography schemes developed here. The main development of the preceding sections has been that one can view the GKP logical shadow protocol as a physical black box protocol that converts a given physical state via application of appropriate random Gaussian unitaries into a convex combination of Gaussian states when using heterodyne detection as the underlying POVM. Since Gaussian states are efficiently described by mean $\bs{\overline{x}}\in \R^{2n}$ and variance $V=S^TS\in \R^{2n \times 2n}$, each part of the decomposition offers an efficient classical description and transforms in a simple manner under application of Gaussian channels \cite{Weedbrook_2012}.
Similar decompositions have been utilized, e.g., 
in ref.~\cite{Bourassa_2021} to develop classical simulation methods for CV states by first decomposing them into a linear combination over Gaussian states, analogous to simulation via stabilizer state decompositions found in the qubit-based quantum computing literature, see, e.g., ref.~\cite{Bravyi2019simulationofquantum} and references therein. 
The practical drawback of the previously 
presented methods, however, is that an analytic 
description of the initial state must be known \textit{a priori}, on which basis the decomposition then proceeds and no  bound has been derived on the number of parts necessary to obtain a good approximation of the state. By focusing on reproducing underlying logical properties of the input state and using statistical methods known from classical shadow tomography, the tools developed here yield an experimental method that converts a given physical state into a convex combination of Gaussian states, corresponding to the samples obtained in the protocol and our bounds yield upper bounds on the number of components (samples) necessary to achieve a good approximation to the input state on the logical level. In the development of practical quantum computers based on GKP encoded logic, one may hence apply this protocol to convert experimentally realizable (multi-mode) GKP states into a classical description of Gaussian state components, which can then further be used to assess their performance in algorithms or under noise. While the overhead is not expected to scale favourably in the number of non-Gaussian channels, one may also imagine a recursive version of this procedure to simulate non-Gaussian evolutions while using the logical shadow protocol to repeatedly convert the mid-simulation states into a convex combination of Gaussian states. The protocol outlined here is only one of many possible uses of our scheme and highlights its potential for practical applications.

\subsection{Random Wigner tomography}

Yet another important and practically relevant class of POVMs is given by photon parity measurements. The parity operator $\hat{\pi}=e^{-i\pi \hat{N}}$, with $\hat{N}=\sum_{i=1}^n, \hat{n}_i$ has an interesting decomposition into displacement operators with constant characteristic function
\begin{align}
    \Tr\lrq{D^{\dagger}\lr{\bs{\beta}} \hat{\pi}}&=\int_{\R^{2n}}d\bs{\alpha}\, \braket{\bs{\alpha}| D^{\dagger}\lr{\bs{\beta}}|-\bs{\alpha}} \\ 
    &=\int_{\R^{2n}}d\bs{\alpha}\, e^{-2\pi \bs{\alpha}^T\bs{\alpha}} =2^{-n}.
    \nonumber
\end{align}
The fact that this characteristic function is constant has the effect that expectation values of displaced parity measurements yield
\begin{align}
    \langle D\lr{\bs{x}}\hat{\pi}D^{\dagger}\lr{\bs{x}} \rangle 
    &= 2^{-n} \int_{\R^{2n}} d\bs{\beta}\, e^{-i2\pi \bs{x}^T J \bs{\beta}} \Tr\lrq{D\lr{\bs{\beta}} \rho} \nonumber \\
    &= 2^{-n} W_{\rho}\lr{\bs{x}}. 
\end{align}
Up to the rescaling $2^{-n}$  this is precisely the Wigner function we have encountered in eq.\ \eqref{eq:Wigner}. This basic insight has been used consistently in quantum optics for the purposes of tomographic recovery \cite{PhysRevLett.76.4344}.
The chi-function associated to the measurement channel
\begin{align}
    c\lr{\bs{\alpha}, \bs{\beta}} 
    &=\frac{1}{2}\lrq{\delta\lr{\bs{\alpha}}\delta\lr{\bs{\beta}}+2^{-2n}} \label{eq:c_Wigner}
\end{align}
is invariant under symplectic transformations $S\in \Sp_{2n}\lr{\R}:\; c\lr{S\bs{\alpha}, S\bs{\beta}}=c\lr{\bs{\alpha}, \bs{\beta}}$:  the only non-trivial component a logical Clifford twirl can have is provided by the displacement twirl over displacements in $\CL^{\perp}$.  Due to the connection of this POVM to Wigner tomography, each displacement sampled in the displacement twirl $\mu_{\CL^{\perp}}\lr{  \bs{\gamma}} $ corresponds to choosing to estimate the Wigner function of the input state $\rho$ at the random point $\bs{\gamma}\in \CL^{\perp}$.
Similar as before, a displacement twirled photon parity measurements -- let us call them \textit{Wigner shadows} -- can be understood to behave like a logical depolarizing channel. In contrast to the cases we have considered before, the characteristic function in eq.\ \eqref{eq:c_Wigner} contains a constant that would yield a diverging contribution when summing over all stabilizer-equivalent coherences in the channel, which forces us to make a more refined model of the displacement twirl. 
%
We consider the measure $d\mu\lr{\bs{\gamma}}=p_{\sigma^2}\lr{\bs{\gamma};\CL^{\perp}} d\bs{\gamma}$, with probability density
\begin{align}
    p_{\sigma^2}\lr{\bs{\gamma};\CL^{\perp}}&=N_{\CL^{\perp}}^{-1} \sum_{\bs{\xi}^{\perp} \in \CL^{\perp}} e^{-\frac{\sigma^2}{2}\|\bs{\xi}^{\perp}\|^2-\frac{1}{2\sigma^2} \| \bs{\gamma}- \bs{\xi}^{\perp}\|^2},\\
    N_{\CL^{\perp}}&=(2\pi \sigma^2)^n\, \Theta_{\CL^{\perp}}\lr{i\frac{\sigma^2}{4\pi}},
\end{align}
where we have introduced the lattice theta constant \cite{ConwaySloane, Conrad_2022} $\Theta_{L}\lr{z}=\sum_{\bs{x}\in L} e^{i2\pi z\|\bs{x}\|^2}$ to express the normalization factor.

The characteristic function of this distribution is given by
\begin{align}
    \nu_{\sigma^2}\lr{\bs{\Delta}}&= \int d\mu\lr{\bs{\gamma}}\, e^{-i2\pi \bs{\gamma}^TJ\bs{\Delta}}
    \\
    &=N_{\CL^{\perp}}\sum_{\bs{\xi}^{\perp} \in \CL^{\perp}} e^{-\frac{\sigma^2}{2}\|\bs{\xi}^{\perp}\|^2-i2\pi \lr{\bs{\xi}^{\perp}}^TJ\bs{\Delta} } 
    \int d\bs{\gamma}\, e^{-\frac{1}{2\sigma^2}\bs{\gamma}^T\bs{\gamma}-i2\pi \bs{\gamma}^TJ\bs{\Delta}},
    \nonumber
\end{align}
and we can evaluate this line using the 
Poisson summation formula (see 
ref.~\cite{Conrad_2022} for a formulation 
tailored to symplectic inner 
products and sums over lattices) and Gaussian integration to obtain
\begin{align}
    \nu_{\sigma^2}\lr{\bs{\Delta}}&= c^{-1} \sum_{\bs{\xi} \in \CL} e^{-2\pi^2\sigma^2 \|\bs{\Delta}\|^2}e^{-\frac{2\pi^2}{\sigma^2} \|\bs{\Delta}-\bs{\xi}\|^2}, \\
    c&=\det\lr{\CL^{\perp}} \sigma^{2n} \Theta_{\CL^{\perp}}\lr{i\sigma^2/4\pi}.
\end{align}
This distribution is in particular normalizable, with $\int d\bs{\Delta}\, \nu\lr{\bs{\Delta}} =1$.
This allows us to compute the logical fidelity of the depolarizing channel
\begin{align}
    1-p&\geq\int_{\mathcal{V}\lr{\CL^{\perp}}} d\bs{\alpha}\, \int_{\R^{2n}} d\bs{\Delta} c\lr{\bs{\alpha},  \bs{\alpha}-\bs{\Delta}}\nu_{\sigma^2}\lr{\bs{\Delta}} \\
    &=\frac{1}{2}\lrq{1+\frac{\det\lr{\CL^{\perp}}}{2^{2n}}}.
    \nonumber
\end{align}
If the GKP code is chosen so to encode $k$ qudits each of dimension $d$ in $n$ modes, this computation estimates that the logical information accessible from the samples can be understood to have undergone a logical depolarizing channel with depolarizing probability
\begin{equation}
    p\leq \frac{1}{2}\lrq{1- 2^{-n \lr{2 + \frac{k}{n}\log_2\lr{d}} } }.
\end{equation}
For a fixed number $n$ of modes this estimate is reassuring: it tells us that the parameters can be chosen so that the effective logical error probability is bounded away from $\frac{1}{2}$. Asymptotically for large $n$, however, we are only promised an exponentially small amount of logical information content to survive the procedure. 

\subsubsection{A quasi-logical estimator}

The pointer states output by photon parity measurements are generically relatively unwieldy states and we expect it to be difficult to use them in a similar manner as e.g. the heterodyne pointers discussed earlier. In the present context it is however more suggestive to use samples from the Wigner function of an arbitrary state in the following way. 
Let $G\lr{\bs{x}}=W_G\lr{\bs{x}}$ be the Wigner function of an arbitrary 
\footnote{Hence, this is not necessarily finite dimensional.} trace-class observable on a system of $n$
quantum harmonic oscillators
and let $\CL\subseteq \CL^{\perp}$ describe a GKP code with even symplectic Gram matrix $A$. We define the estimator
\begin{equation}
    \widetilde{G}\lr{\bs{x}}  \coloneqq  W_{\rho}\lr{\bs{x}} G\lr{\bs{x}} \label{eq:estimator}
\end{equation}
such that samples from the Wigner function of an input state according to the distribution $p_{\sigma}\lr{\bs{\gamma};\CL^{\perp}}$can be combined with the observable Wigner function $G\lr{\bs{x}}$ at the same points to produce the expectation value
\begin{align}
    \overline{G}^{\sigma}_{\CL^{\perp}} &=  \int d\bs{x}\, p_{\sigma}\lr{\bs{x};\CL^{\perp}}  W_{\rho}\lr{\bs{x}} G\lr{\bs{x}}  \nonumber\\
    &\xrightarrow{\sigma \rightarrow 0}\sum_{\bs{\xi}^{\perp} \in \CL^{\perp}} W_{\rho}\lr{\bs{\xi}^{\perp}} G\lr{\bs{\xi}^{\perp}} \eqqcolon \overline{G}_{\CL^{\perp}}.\label{eq:sum_lat}
\end{align}
If either the Wigner function of the input state or the observable were solely supported on the lattice $\CL^{\perp}$, we see that in this limit $\sigma \to 0$ the expectation value 
\begin{equation}
    \Tr\lrq{\rho G}=\int d\bs{x}\, W_{\rho}\lr{\bs{x}} G\lr{\bs{x}}
\end{equation}
would be exactly reproduced. Note, however, that code space operators 
(i.e., 
ones that commute with the stabilizer group) are not necessarily restricted in this way and 
can also assume non-zero values for the Wigner function outside of $\CL^{\perp}$. The simplest way to realize this is by interpreting the Wigner function evaluated at point $\bs{x}$ as measure of the degree-$\pi$ rotation symmetry about $\bs{x}$. Naturally, a code state 
\begin{equation}
    \rho=\sum_{\bs{\xi}^{\perp}\in \CL^{\perp}} c\lr{\bs{\xi}^{\perp}} D\lr{\bs{\xi}^{\perp}} 
\end{equation}
can also possess rotation symmetry about phase space points outside of $\CL^{\perp}$, such as at the points right in between lattice points $\bs{x}\in \frac{1}{2}\CL^{\perp}$ and an  interpretation of eq.~\eqref{eq:sum_lat} as code-space expectation value can generally not be claimed. We will see in the subsequent section however how we can still make good use of eq.~\eqref{eq:sum_lat}. To this end, we examine how well the lattice sum is approximated for small $0<\sigma \ll 1$. In order to bound the effective convergence of $p_{\sigma}\lr{\bs{x};\, \CL^{\perp}}\to \Sha_{\CL^{\perp}}\lr{\bs{x}}=\sum_{\bs{\xi}^{\perp}\in \CL^{\perp}} \delta\lr{\bs{x}-\bs{\xi}^{\perp}}$ under the integral, we need to make soft assumptions on the state 
and 
observable.


\begin{lem}[Random 
lattice point sampling]\label{lem:rdn_lat_points}
    Let $\rho, G$ be operators such that the product of their Wigner functions $ \widetilde{G}\lr{\bs{x}}  =  W_{\rho}\lr{\bs{x}} G\lr{\bs{x}}$ is well defined, $\Tr\lrq{G^2}<\infty$ is finite and further assume that 
    $W_{\rho}\lr{\bs{x}}, G\lr{\bs{x}}$ 
    are Lipschitz-continuous, with $\|\nabla W_{\rho}\lr{\bs{x}}\| \leq l_{\rho}$ and $\|\nabla G\lr{\bs{x}}\| \leq l_{G}$. Set $\tilde{l}=l_{\rho}+l_G$.
    Let $\sigma \ll \lambda_1\lr{\CL^{\perp}}$ be a small parameter. It holds that
    \begin{align}
    \overline{G}^{\sigma}_{\CL^{\perp}} &=  \int d\bs{x}\, p_{\sigma}\lr{\bs{x};\CL^{\perp}}  \widetilde{G}\lr{\bs{x}}  \nonumber \\
    &=\sum_{\bs{\xi}^{\perp} \in \CL^{\perp}} \widetilde{G}\lr{\bs{\xi}^{\perp}} +\epsilon\lr{\sigma} \\
    &= \overline{G}_{\CL^{\perp}}+\epsilon\lr{\sigma} \nonumber
    \end{align}
    with 
    \begin{equation}
        |\epsilon\lr{\sigma}| \leq \sqrt{2}\sigma \frac{\sqrt{2} \Gamma\lr{n+\frac{1}{2}}}{  \Gamma\lr{n}} \tilde{l}.
    \end{equation}
    Furthermore,
    \begin{align}
    \overline{G}^{\sigma}_{\CL^{\perp}} &=  \int d\bs{x}\, p_{\sigma}\lr{\bs{x};\CL^{\perp}}  \widetilde{G}^2\lr{\bs{x}}  \nonumber \\
    &=\sum_{\bs{\xi}^{\perp} \in \CL^{\perp}} \widetilde{G}^2\lr{\bs{\xi}^{\perp}} +\epsilon\lr{\sigma} \\
    &= \overline{G}^{(2)}_{\CL^{\perp}}+2\epsilon\lr{\sigma}\nonumber
    \end{align}
    with the same bound for $\epsilon\lr{\sigma}$ and
    $\overline{G}^{(2)}_{\CL^{\perp}}\leq 2^{2n}\Tr\lrq{G^2}$.
\end{lem}


Note that for large $n$, we have that $\Gamma\lr{n+\frac{1}{2}}/\Gamma\lr{n}= \sqrt{n}+O\lr{n^{-1/2}}$ while $\lambda_1\lr{\CL^{\perp}}\propto \sqrt{n}$ is the maximally achievable shortest vector scaling for any family of lattices $\CL^{\perp}=\CL^{\perp}_n$. Picking $\sigma=o\lr{1/\sqrt{n}}$ essentially allows to reduce this error to a negligible amount.
 


\subsection{Chasing shadows}\label{sec:chasing_shadows}

In this section, we investigate how the estimator derived above together with a random choice of lattice allows to construct a shadow tomography protocol for the Hilbert space of a $2n$-dimentional continuous variable system. Crucially, by averaging over lattices, the reference to any particular GKP code (as in previous section \ref{sec:GKPShadows}) is removed. The following  presents a randomized scheme to select phase space points on which the Wigner function of an arbitrary input state are evaluated so to estimate any observable without restriction to any particular GKP code space. 
From here on we will focus on symplectically self-dual lattices $L=L^{\perp}\subseteq \R^{2n}$. Such a lattice is spanned by the rows of a symplectic generator matrix $M\in \Sp_{2n}\lr{\R}$ and the space of all possible symplectic lattices in dimension $2n$ is simply parametrized by the group of symplectic matrices \textit{up to basis transformations} $Y_{n}=\Sp_{2n}\lr{\Z}\backslash \Sp_{2n}\lr{\R}$, where here the ``$\backslash$'' operator is meant to denote a left-modulo operation since basis transformations are implemented by left-multiplication on a given generator matrix.

\subsubsection{Moments of symplectic lattices}

Following work by Siegel and Rogers, it has been realized by Buser and Sarnak in ref.~\cite{Sarnak1994} and by Kelmer and Yu in ref.~\cite{Kelmer_2019} that he space of symplectic lattices possesses a Haar measure $\mu\lr{L}$, relative to which functions of the lattice can be integrated. Define the Siegel transform of a sufficiently fast decaying function $f:\R^{2n}\to \R$ by

\begin{equation}
    F_f\lr{L}=\sum_{\bs{v}\in L_{\rm pr}} f\lr{\bs{v}},
\end{equation}
where $L_{\rm pr}$ denotes the set of primitive vectors of $L$, that is the set of minimal vectors that cannot be obtained by another via ineger multiplication. This (infinite) set is such that every point in the lattice $L=\N \otimes L_{\rm pr}$ is uniquely reproduced a positive integer multiple of a point in $L_{\rm pr}$.
We also define

\begin{equation}
    \widetilde{F}_f\lr{L}=\sum_{\bs{v}\in L-\lrc{0}} f\lr{\bs{v}}=\sum_{k\in \N} \sum_{\bs{v}\in L_{\rm pr}} f\lr{k \bs{v}}.
\end{equation}
The symplectic version of Siegel's mean value theorem, as derived in ref.\ \cite{Sarnak1994}, can thus be formulated as
\begin{equation}
    \int_{Y_n} d\mu\lr{L} F_f\lr{L} = \frac{1}{\zeta\lr{2n}}\int_{\R^{2n}}d\bs{x}\, f\lr{\bs{x}},
\end{equation}
with $\zeta\lr{z}=\sum_{k\in \N }k^{-z}$ 
being 
the Riemann zeta function. Applying this formula once allows to derive its perhaps more standard variant
\begin{align}
    \int_{Y_n} d\mu\lr{L} \widetilde{F}_f\lr{L}&= \frac{1}{\zeta\lr{2n}}\sum_{k\in \N} \int_{\R^{2n}} d\bs{x} f\lr{k\bs{x}} \\ &=  \int_{\R^{2n}} d\bs{x} f\lr{\bs{x}}.
    \nonumber
\end{align}
Analogous to the inner product over $L^2\lr{\R^{2n}}$, \begin{equation}
    \braket{f, g}=\int_{\R^{2n}} d\bs{x}f\lr{\bs{x}}^*g\lr{\bs{x}},
\end{equation}
one can define an inner product over $Y_n$ as
\begin{equation}
    \braket{F, G}_{Y_n}=\int_{Y_n} d\mu\lr{L}F\lr{L}^*G\lr{L}.
\end{equation}
In this notation, the mean value formula can be simply expressed as 
\begin{equation}
    \braket{1, F_f}_{Y_n}=\frac{1}{\zeta\lr{2n}}\braket{1, f}.
\end{equation}
In ref.~\cite{Kelmer_2019}, Kelmer and Yu have derived a formula for second moments over $Y_n$, which for two even and compactly supported functions $f, g$ can be compactly written as 

\begin{equation}
    \braket{F_f, F_g}_{Y_n}=\frac{\braket{f, 1}\braket{1, g}}{\zeta\lr{2n}^2} + \frac{2}{\zeta\lr{2n}}\lr{\braket{f, g}+\braket{\iota \lr{f}, g}},
\end{equation}
where $\iota$ is an isometry such that $\|\iota \lr{f}\|=\|f\|=\sqrt{\braket{f, f}}$.
Note that, since $L_{\rm pr}$ and $L$ are even sets, any non-even function $f$  on the LHS of this formula can simply be replaced by their even projection $\frac{f\lr{\bs{x}}+ f\lr{-\bs{x}}}{2}$ (similar for $g$): the assumption of evenness of the functions is without loss of generality.
Using this expression, we show the following statement in the appendix.

\begin{lem}[Scalar product bound]\label{lem:scalar}
    Let $f, g$ be two even compactly supported functions, it holds that
    \begin{equation} \left|\braket{\widetilde{F}_f, \widetilde{F}_g}_{Y_n}\right| \leq |\braket{f, 1}\braket{1, g}|+\frac{4\zeta\lr{n}^2}{\zeta\lr{2n}}\|f\|\|g\|.
    \end{equation}
\end{lem}

\subsection{Estimating CV observables with random lattices}

The mean value theorem and Lemma~\ref{lem:scalar} are the key 
tools to analyse the behaviour of the estimator $\widetilde{G}$ when in addition to the points $\bs{x}\sim p_{\sigma}\lr{\bs{x}, L}$ being randomly sampled from a lattice-Gaussian-like distribution also the lattice $L=L^{\perp}$ is chosen according to a uniformly random distribution $\mu\lr{L}$ over the space of lattices $L=L^{\perp}$.
Under this procedure, the estimator takes the expectation value 
\begin{align}
    \int d\mu\lr{L}\int d\bs{x}\, p_{\sigma}\lr{\bs{x};\, L} \widetilde{G}\lr{\bs{x}}&= \int d\mu\lr{L}\,\overline{G}_L \nonumber\\ 
    &= \widetilde{G}\lr{0} + \int d\bs{x}\, \widetilde{G}\lr{\bs{x}}, \nonumber\\
    &= \widetilde{G}\lr{0} + \Tr\lrq{\rho G}\nonumber \\
    &\eqqcolon \overline{G}
\end{align}
which is attained up to a small error $\epsilon\lr{\sigma}$ as in Lemma~\ref{lem:rdn_lat_points}. 
Note, that $\widetilde{G}\lr{0}=\Tr\lrq{\hat{\pi}\rho}\Tr\lrq{\hat{\pi}G}$ is simply the total photon parity of the state and observable together and can be offset by an independent estimation of the photon parity of the unknown input state.

We consider the following two scenarios. In the first, we simply sample a point $\bs{x}_{i, k}$ according to the distribution $p_{\sigma}\lr{\bs{x}, L_k}$ for each randomly chosen lattice $L_k$. In this case a second moment bound is obtained as follows.

\begin{lem}
    [Second moment: random lattice sampling]\label{lem:random_lat_1}
    Let $\tilde{G}$ be an estimator as in Lemma~\ref{lem:rdn_lat_points} for an observable on a $n$-mode continuous variable quantum system and assume that the corresponding observable $G$ is such that 
    \begin{equation}
        \|G\|^2_2\coloneqq \int d\bs{x}\, |G\lr{\bs{x}}|^2<\infty 
    \end{equation}
    is finite. It holds that
\begin{equation}
    \braket{\tilde{G}^2}\leq 2^{2n} \lrc{1+\lr{\frac{2\pi}{\sigma^2}}^n} \|G\|_2^2.
\end{equation}

\end{lem}


The second scenario that we consider is such that the ``inner'' average is computed with high precision so that the only contributions to the variance arise from the choice of random lattices. Using similar techniques and Lemma~\ref{lem:scalar}, in appendix~\ref{app:lem4} we show that estimation of $\overline{G}_L^{\sigma}$ yields a variance bound scaling with
\begin{equation}
    \braket{\lr{\overline{G}_L^{\sigma}}^2}_{Y_n}=O\lr{\frac{\zeta\lr{n}^2}{\zeta\lr{2n}}\lr{4\pi\frac{\sigma^{-2}+\sigma^2}{\sigma^4}}^n \|G\|_1^2}.\label{eq:random_lat_2}
\end{equation}

This second moment bound naturally implies a variance bound for the expectation value $\overline{G}$ in the scenario where the inner average is taken with high precision and the only fluctuations around the mean essentially arise from the choice of random lattice. It is interesting to note that in terms of the scaling with $\sigma$ and $n$, this bound appears to scale significantly worse as compared to the simpler one in Lemma~\ref{lem:random_lat_1}, which encourages the strategy to simply produce a single sample point for each individual lattice. Supportive of this observation is that the bound in Lemma~\ref{lem:random_lat_1} scales with the $L^2$-norm of the observable while in eq.~\eqref{eq:random_lat_2} it is the $L^1$-norm, which typically has less favourable convergence properties as compared to the $L^2$-norm.

This sampling procedure results in $N$ points $\lrc{\bs{x}_i}_{i=1}^N\subset \R^{2n}$ at which the Wigner function of an unknown input state is to be evaluated. Combining the values of the Wigner function of the input state with that of the observables at these points and performing an average then yields the desired estimate for the phase space average $\overline{G}$ (which is still offset by the average photon parity of the state and observable).

\begin{them}[Random lattice CV shadows]\label{them:Randomlatshadow}
    Let $\tilde{\epsilon}, \delta, \sigma> 0$ be small parameters, let $G_{m},\, m=1,\hdots, M$ be operators with finite $\|G_m\|_2^2$ and set \begin{align}
        K&:=2\log\lr{2M/\delta}\; \text{and} \\
        B&:=2^{2n}\lrc{1+\lr{\frac{2\pi}{\sigma^2}}^n}\max_m \|G_m\|_2^2 /\tilde{\epsilon}^2.\end{align} 
    Then $N=KB$ samples from the distribution of phase space points $\lrc{\bs{x}_i}_{i=1}^N$ sampled according to protocol in fig.~\ref{prot:CV_shadow} approximate the expectation values 
    \begin{equation}
        \overline{G}_{m}=\Tr\lrq{\rho G_m}+\tilde{G}_m\lr{0}\end{equation} of an arbitrary state on an $n$-mode continuous variable quantum system up to a photon-parity offset $\tilde{G}\lr{0}$ and error $|\epsilon\lr{\sigma}|+\tilde{\epsilon}$ with probability at least $1-\delta$.
\end{them}

\begin{figure*}
    \begin{mybox}
        \begin{prot}\label{protocol1}
            \begin{enumerate}
                \item[]
                \item Sample $N$ points  $\bs{x}_i \xleftarrow{p_{\sigma}\lr{\bs{x}, L_k}} L_k \xleftarrow{\mu\lr{L}}Y_n$. 
                \item Evaluate the Wigner function of an unknown input states at these points $\lrc{\bs{x}_i}_{i=1}^N$ and combine the output with the Wigner function of the observable $G\lr{\bs{x}_i}$ at these points.
                \item For each point, return the estimate  $\widetilde{G}\lr{\bs{x}_i}=W_{\rho}\lr{\bs{x}_i}G\lr{\bs{x}_i}$.
             \end{enumerate}
            \end{prot}
            \caption{Random CV shadow protocol.}\label{prot:CV_shadow}
    \end{mybox}
\end{figure*}

It is a valid question to ask whether such an involved procedure really is necessary to design a shadow protocol for observables on a CV system. Many alternative strategies seem equally valid, in particular through alternative means to regularize the Hilbert space to an effectively finite-dimensional system. This is the approach that has been taken by Iosue et al. and Ghandari et al.\ in refs.~\cite{iosue2022continuousvariable, gandhari2022continuousvariable}. A related technical ingredient to draw random unitaries on a CV Hilbert space is also presented in ref.~\cite{Zhong2023information}. The interesting observation to be taken away from this section is that it is in fact possible to interpolate between \textit{logical} shadow tomography of a discrete quantum system embedded in a CV Hilbert space and \textit{physical} shadow tomography applied to the full CV Hilbert space. The bounds derived in this section explain exactly the cost of this interpolation. This property appears to be special to GKP codes and is implied by the fact that the group algebra of displacement operators defining the GKP stabilizer group together by their real powers form a complete operator basis for the CV Hilbert space and the existence of a Haar measure over the possible codes. These are properties not shared by other bosonic quantum error correcting codes that we know of where such a interpolation does not seem to be possible.

Note that in ref.~\cite{CVLearning} an information theoretic bound for learning CV quantum states 
has been derived, which we recall.

\begin{them}[Obstructions against general quantum state tomography \cite{CVLearning}]\label{them:CVlearning}
Let $\rho$ be an unknown quantum state on $n$ bosonic 
modes satisfying an energy
constraint $\Tr(\hat N\rho)
\leq n \overline{N}$ for some absolute constant $\overline{N}$. Then the number of copies of $\rho$ required to perform quantum state tomography with precision $\varepsilon$ in trace distance has to scale at least as 
$(\Theta (\overline{N}/\varepsilon))^{2n}$.
\end{them}

From the sample overhead scaling in Theorem~\ref{them:Randomlatshadow}, 
we clearly see that our protocol appears consistent with this bound, scaling both 
exponential in the system size as well as the lattice approximation parameter $\sigma^{-1}$. We can model an average energy constraint on an input state by assuming it to be in the support of a Gaussian regularizer $\hat{R}=e^{-\sigma\hat{N}}$, which constrains the typical support of any input state to be within a phase-space radius of 
\begin{equation}
r=\Omega\lr{\sqrt{n \sigma^{-1}}}. 
\end{equation}
In a rough estimate, we relate the typical support radius of such a state in phase space $r^2\propto n\overline{N}$ to its average photon number, which conversely leads us to the estimate that, given a state with average photon number constrained as above, sampling the Wigner function from a probability distribution with Gaussian envelope with variance $\sigma^{-1} \propto \overline{N}$ suffices to cover the essential support of the state. With this estimate the theorem above thus expects a scaling of the sample overhead with $\sigma^{-2n}$, which is indeed the scaling in $\sigma$ of Theorem~\ref{them:Randomlatshadow}. Note that this model for a regularized state is not generic, but tailored to reproduce the results of Theorem~\ref{them:Randomlatshadow} in consistency with the bound above. In practice, the strategy to choose a $\sigma$ (in order to minimize the sample overhead) needs to be tailored to the macroscopic structure of the state one expects and the approximation error $|\epsilon\lr{\sigma}|$ that is desired. To minimize this error, however,  as pointed out in Lemma~\ref{lem:random_lat_1}, it would be desirable to choose $\sigma=o\lr{1/\sqrt{n}}$, such that in general, all things combined, our protocol is not expected to come close to the optimal bound in Theorem~\ref{them:CVlearning}.

\section{Discussion and Outlook}\label{sec:DiscussionOutlook}

In this article, we have developed the toolbox of shadow tomography for a continuous variable system relative to GKP codes, which themselves possess an intrinsic infinite structure and so that the infinitude of the CV 
phase space can be matched.  Our analysis and bounds on logical shadow tomography for GKP codes is particularly useful to design and test logical properties of experimental realizations of GKP code states such as pursued for GKP-\emph{measurement based quantum computations} 
(MBQC) \cite{Bourassa_2021_blue,  Menicucci_2014} or 
 realizations via superconducting architectures,  where a photon parity measurement is inexpensive \cite{Terhal_2020,  lachancequirion2023autonomous}.
 

The logical channel twirl of a bosonic POVM introduced here is a powerful technique in its own right, as the space of bosonic POVMs is plentiful and one often does not have native access to a POVM tailored to the observable one desires to measure.  As we have discussed,  one may, however, tailor a given POVM to ones needs by suitable channel twirling to generate classical snapshots that capture relevant information of a given quantum state.  It would be interesting to investigate how well properties of quantum states can be learned for POVMs strongly different from the measurement of interest by combining our twirling technique with a classical learning strategy and explore where the boundaries of such schemes lie.

The probabilistic state decomposition derived from heterodyne measurements and displacement twirling relative to a GKP code also warrants further investigation. This protocol can be understood as a ``black-box'' scheme that produces relevant Gaussian samples from an arbitrary state in the input such that logical expectation values are aligned.  Since Gaussian states are computationally easy to tract through Gaussian evolutions,  this tool may find application in the simulation of realistically producible states with a priori unknown decomposition into Gaussian states as they are evolved under Gaussian evolutions which is relevant in the design of large scale experiments with the GKP code \cite{Bourassa_2021}. It would be interesting to investigate \textit{lower bounds} for this kind of decomposition: how many Gaussian states need to be mixed in order to emulate the logic of an encoded (GKP) state? This question, here motivated by our protocol and the proposed applications to simulation, motivate the general exploration of effective logical properties of easily accessible physical states which need not be restricted to the representation of encoded quantum states. It would further be interesting to understand what kind of logical dynamics can be effectively generated -- in expectation -- via Gaussian quantum channels. 

Finally, we view the extension of the logical GKP shadow to a full-fledged CV shadow discussed in sec.~\ref{sec:chasing_shadows} as the most interesting contribution presented here. 
It is interesting that random coding techniques are useful for (continuous variable) shadow tomography and, as discussed in sec.~\ref{sec:chasing_shadows}, the presented toolbox also stimulates further research towards the construction of CV state designs, of which we present a GKP incarnation in ref.~\cite{CV_GKP_design}. 
It would be interesting to investigate the extend to which the presented ideas are applicable to discrete variable systems, e.g., whether logical shadow tomography relative to random qubit-QECCs also allows to estimate properties of the full physical system. 
Similarly, it would be interesting to investigate the extractable information from elements of random stabilizer or subsystem codes on a multi-qubit quantum system through the lens of shadow tomography, where we speculate that techniques similar to those presented here may be applicable.
The toolbox developed here allows for a wide range of generalizations and possible applications,  for which we hope this work stimulates curiosity.
\medskip

\acknowledgments
We are grateful to %
V.~Albert, 
C.~Bertoni,
A.~Ciani,
J.~Haferkamp, 
J.~Iosue,
R.~Kueng,
J.~Magdalena de la Fuente,
Y.~Teng
and N.~Walk
for many inspiring and helpful discussions. 
JC thanks in particular C.~Bertoni sharing his toolbox of ideas  on statistical bounds and V.~Albert and J.~Iosue for discussions on the relation between the techniques discussed in sec.~\ref{sec:chasing_shadows} and CV state designs.
%
JC and JE gratefully acknowledge support from the BMBF (RealistiQ, MUNIQC-Atoms, PhoQuant, QPIC-1, and QSolid),  the DFG (CRC 183, project B04, on entangled states of matter), the Munich Quantum Valley (K-8), the ERC (DebuQC), Quantum Berlin
as well as the Einstein Research Unit on quantum devices.

\bibliographystyle{apsrev4-1}

\bibliography{Shadow_bib}% Produces the bibliography via BibTeX.


\begin{widetext}
    
\appendix

    
\section{Generating the symplectic group}


In this section, we discuss schemes for generating all elements in the symplectic 
group \cite{Continuous,GaussianQuantumInfo}.
Block matrices 
\begin{equation}
S=\begin{pmatrix}
A & B \\ C & D
\end{pmatrix} \in \Z_d^{2n \times 2n}
\end{equation}
are symplectic if $S^TJS=J$,  which requires $A^TC=C^TA$,  $B^TD=D^TB$ as well as$A^TD-C^TB=I$.  In particular, 
we have that for $B=C=0$ it is symplectic if $D=A^{-T}$ such that $S=A\oplus A^{-T}$.  If $A=D=I$  and $C=0$ ($B=0$) it becomes necessary that $B=B^T$ is symmetric ($C$ is symmectric). 
Matrices of these constrained types %have particularly 
feature a particularly
simple structure and follow the simple multiplication rules
\begin{align}
\begin{pmatrix}
A_1 & 0 \\ 0 & A_1^{-T}
\end{pmatrix} \begin{pmatrix}
A_2 & 0 \\ 0 & A_2^{-T}
\end{pmatrix}
&=\begin{pmatrix}
A_1A_2 & 0 \\ 0 & \lr{A_1A_2}^{-T}
\end{pmatrix}, \\
\begin{pmatrix}
I & B_1 \\ 0 & I
\end{pmatrix} \begin{pmatrix}
I & B_2 \\ 0 & I
\end{pmatrix}
&=\begin{pmatrix}
I & B_1+B_2 \\ 0 & I
\end{pmatrix}, \\
\begin{pmatrix}
I & 0 \\ C_1 & I
\end{pmatrix} \begin{pmatrix}
I & 0 \\ C_2 & I
\end{pmatrix}
&=\begin{pmatrix}
I & 0 \\ C_1+C_2 & I
\end{pmatrix} .
\end{align}

In this section, we 
show specifically -- building on previous work on qubits \cite{AaronsonGottesman,  PatelMarkovHayes} -- how for prime dimension $q$,  symplectic matrices in $\Sp_{2n}\lr{\Z_d}$ can be synthesized from an elementary gate set $S=\lrc{ J_i ,   P_{i},  C_{i\rightarrow j}}$ of such constrained block matrices consisting of the following matrices in block form,   where $\pi_{i}=\bs{e}_i\bs{e}_i^T$ and $e_{i,j}=\bs{e}_i\bs{e}_j^T$:

\begin{itemize}
\item  The quantum Fourier transform on qudit $i$
\begin{equation} J_i=\begin{pmatrix}
I-\pi_{i} & \pi_i  \\ -\pi_i & I-\pi_i
\end{pmatrix},\, i\in \lrq{1,n}
\end{equation}
with $J_i^2=-I$,  mapping  $X_i \mapsto Z_i^{-1}, \, Z_i \mapsto X_i,$

\item the phase gate
\begin{equation}
P_i=\begin{pmatrix}
I & 0  \\ \pi_i& I
\end{pmatrix},\,  i\in \lrq{1,n},
\end{equation}
 mapping $X_i\mapsto X_i Z_i$ and 

\item the  CNOT gate
\begin{equation}
C_{i\rightarrow j}=\begin{pmatrix}
I+e_{j,i} &  0 \\ 0& I-e_{i,j}
\end{pmatrix},\,  i \neq j \in \lrq{1,n}
\end{equation}
 that maps $X_i \mapsto X_iX_j$.

\item The CNOT gate is of block diagonal form,  and it can be shown by performing the matrix multiplication that the upper triangular elementary block matrix
\begin{equation}
B_{i,j}=\begin{pmatrix}
I &  e_{i,j}+e_{i,j} \\ 0& I
\end{pmatrix} = J_j^{-1} C_{j\rightarrow i} J_j
\end{equation}
 mapping $Z_i\mapsto X_j Z_i$ and $Z_j\mapsto X_i Z_j$ can be obtained by conjugating the CNOT with a Hadamard type gate.  This generating set has $|S|=2n+n(n-1)$ elements,  where the contribution $n(n-1)$ comes from the fact that we assume all-to-all connectivity for the CNOTs in use.  This set can be reduced down to a set of $3n-1$ generators with CNOTs only between a linear number of pairs analogous to the Lickorish generators for the Dehn-twists mentioned in the main text,  which however would come at the cost of needing to mediate CNOTs not included in the set via a $O(n)$ number of those that are. 
 \end{itemize}

Denote sequences generated by a finite product from $S$ as $S^k:=\lrc{g_1 ,g_2,\hdots ,g_k,, \, g_i \,  \in S}$.
Similar to previous work on generating $\Sp_{2n}\lr{\Z_2}$ we show here a result for prime dimension.
%that for

\begin{lem}[Universality for prime dimensions]
Let $d$ be prime.  For the generating set $S={ J_i ,   P_{i},  C_{i\rightarrow j}}$ defined above,  we have
\begin{equation}
\Sp_{2n}\lr{\Z_d} \subseteq S^k
\end{equation}
for $k=O\lr{d n^2}$.
\end{lem}
That is,  sequences of $O(dn^2)$ of gates from $S$ suffice to generate all elements in $\Sp_{2n}\lr{ d}$.

\proof It has been shown in 
ref.~\cite{DopicoJohnson} that every symplectic matrix  $S\in \Sp_{2n}\lr{\Z_d}$,  $d$ prime admits a decomposition into symplectic matrices
\begin{equation}
S=Q\begin{pmatrix}
I & 0 \\ C & I
\end{pmatrix}
\begin{pmatrix}
A & 0 \\ 0& A^{-T}
\end{pmatrix}
\begin{pmatrix}
I & B \\ 0 & I
\end{pmatrix},
\end{equation}
where $A\in \mathrm{GL}\lr{n, d}$ is invertible and $C \in \mathrm{GL}_n\lr{d}$ and $B \in \mathrm{GL}_n\lr{d}$ are symmetric and $Q$ is a $O(n)$ length product of the matrices $J_i$ we have defined above. Reference \cite{DopicoJohnson}, in fact, has shown 
this for the field of complex numbers $\C$ but the proof carries over to any number field,  such as $\Z_d=\mathbb{F}_d$ for $d$ prime.  
Using this decomposition,  it suffices to check how each individual block matrix can be compiled from the generating set above.  Using that 
\begin{equation}
J\begin{pmatrix}
I & B \\ 0 & I
\end{pmatrix} J^T=\begin{pmatrix}
I & 0 \\ -B & I
\end{pmatrix}
\end{equation} together with $J=\prod_{i=1}^n J_i$ we have that the every upper block triangular matrix can be converted to a lower block triangular one with $O(n)$ overhead and that every block upper triangular matrix 
\begin{equation}
\begin{pmatrix}
I & B \\ 0 & I
\end{pmatrix}
\end{equation}
with $B=B^T$ can be obtained from a $O(dn^2)$ fold product of matrices of type $J_i P_i J_i^T$ and $B_{i,j}$ for their simple multiplication structure.  It remains to bound the complexity of compiling the block diagonal part $A\oplus A^{-T}$. Note that due to the simple multiplication structure of these matrices this problem is equivalent to bounding the complexity of compiling the blocks $A$ as generated by elements $I+e_{j,i}$.  This is bounded 
using 
the same argument 
as 
in 
ref.\  \cite{AaronsonGottesman}, which has employed 
a result from Patel 
et al.\  \cite{PatelMarkovHayes},  who have shown 
that for the underlying field $\Z_2$, an achievable lower bound is given by $O\lr{n^2/\log_2\lr{n}}$.  As was already noticed in 
ref.~\cite{PatelMarkovHayes},  their technique generalizes for any finite field with order $d$,  where it yields a bound $O\lr{n^2/\log_d\lr{n}}$.  In total,  we hence obtain a bound $O(dn^2)$ for the length of the product from $S$ to generate any element in $\Sp_{2n}\lr{\Z_d}$.
\endproof

\section{Proof of Theorem \ref{them:GKPshadow}}\label{app:GKPshadow_proof}

In this section, we provide a proof of Theorem~\ref{them:GKPshadow} 
of the main text.

\begingroup
\renewcommand\thethem{2}
\begin{them}[Full representation of the 
state]
Let $\CL \subset \R^{2n}$ denote the lattice corresponding to a scaled GKP code on $n$ modes that encodes $d^n$ logical dimension and let $\mu$ denote a measure over elements $\Aut^S\lr{\CL^{\perp}}$ forming a logical Clifford $2-$design and let $\CM=\SumInt dz \sketbra{\Pi\lr{z}}$ denote a physical POVM.   Let $\sket{\rho}$ be an arbitrary state on the $n$-mode Hilbert space. The state
\begin{equation}
\sket{S}=N^{-1} \sum_{i=1}^N \CU_{S,  i}\sket{\Pi\lr{z_i}}
\end{equation}
produced by sampling $N$ Gaussian unitary operations via the measure $\mu$ and measurement outcomes from the Born distribution $z_i \sim \sbraket{\Pi\lr{z}|\CU_{S,  i}^{\dagger}|\rho}$ approximates the logical value of the state in Hilbert-Schmidt distance

\begin{equation}
d_{\rm HS}\lr{\Dec \sket{\rho} ,  \Dec \sket{S}} \leq \delta_{\rm HS}^2
\end{equation}
with probability at least $1-\delta$ for 
\begin{equation}
N\geq \frac{2d^{2n}}{\alpha^2 \delta_{\rm HS}^2} \lrq{\ln\lr{\frac{2}{\delta}} + 2n \ln\lr{d}},
\end{equation}
where $\alpha$ is determined by the commutation of the twirled 
POVM with the decoder
\begin{align}
 \Dec\CM^{\tau} &=\alpha \Dec + \beta \sket{\Pi_{\CL} }.
\end{align}
In particular, 
we also have
\begin{equation}
\|\Dec \sket{\rho}-  \Dec \sket{S}\|_1 \leq d^{\frac{n}{2}}\frac{\delta_{\rm HS}}{2} . \label{eq:thm3_tr}
\end{equation}
\end{them}%
\endgroup

\proof
The random variable $x_i=\frac{1}{\alpha N}\sbraket{P_{\alpha} | \Dec  \CU_{S, i}|\Pi\lr{z_i}}$ lies in a range $\lrq{-1/\alpha N,\, 1/\alpha N}$, such that H{\"o}ffding's inequality implies that for all logical Pauli operators $P_{\alpha},\, \alpha=1,\hdots ,d^{2n}$ the probability 
of the arithmetic mean $\sbraket{P_{\alpha} | \Dec |S}$ to deviate from its expectation value is bounded by

\begin{equation}
    P\lr{|\sbraket{P_{\alpha} | \Dec |S} - \mathbb{E}\lrq{\sbraket{P_{\alpha} | \Dec |S}} | \geq \epsilon } \leq 2^{-N\alpha^2\epsilon^2/2},
\end{equation}
such that\begin{equation}
    N=\frac{2}{\alpha^2\epsilon^2}\lrq{\ln\lr{\frac{2}{\delta}}+2\ln\lr{d}}
\end{equation}
asserts that with probability $1-\delta$, the arithmetic mean recovers the true expectation value up to $\epsilon$ accuracy. In particular, this also bounds the logical Hilbert-Schmidt distance

\begin{equation}
    d_{\rm HS}\lr{\Dec \sket{\rho},\, \Dec \sket{S}} \leq d^{2n} \epsilon^2
\end{equation}
with probability $1-\delta$. Setting $\delta_{\rm HS}^2=d^{2n} \epsilon^2$ then yields the result. Finally, eq.\ \eqref{eq:thm3_tr} is recovered using the Cauchy-Schwarz
inequality to trade off the trace distance with the Hilbert-Schmidt distance.
\endproof

\section{Proof of Lemma~\ref{lem:rdn_lat_points}}

In this section,
we prove Lemma~\ref{lem:rdn_lat_points}.

\begingroup
\renewcommand\thelem{1}
\begin{lem}[Random lattice point sampling]
    Let $\rho, G$ be operators such that the product of their Wigner functions $ \widetilde{G}\lr{\bs{x}}  =  W_{\rho}\lr{\bs{x}} G\lr{\bs{x}}$ is well defined, $\Tr\lrq{G^2}<\infty$ is finite and further assume that 
    $W_{\rho}\lr{\bs{x}}, G\lr{\bs{x}}$ 
    are Lipschitz-continuous, with $\|\nabla W_{\rho}\lr{\bs{x}}\| \leq l_{\rho}$ and $\|\nabla G\lr{\bs{x}}\| \leq l_{G}$. Set $\tilde{l}=l_{\rho}+l_G$.
    Let $\sigma \ll \lambda_1\lr{\CL^{\perp}}$ be a small parameter. It holds that
    \begin{align}
    \overline{G}^{\sigma}_{\CL^{\perp}} &=  \int d\bs{x}\, p_{\sigma}\lr{\bs{x};\CL^{\perp}}  \widetilde{G}\lr{\bs{x}}  \nonumber \\
    &=\sum_{\bs{\xi}^{\perp} \in \CL^{\perp}} \widetilde{G}\lr{\bs{\xi}^{\perp}} +\epsilon\lr{\sigma} \\
    &= \overline{G}_{\CL^{\perp}}+\epsilon\lr{\sigma} \nonumber
    \end{align}
    with 
    \begin{equation}
        |\epsilon\lr{\sigma}| \leq \sqrt{2}\sigma \frac{\sqrt{2} \Gamma\lr{n+\frac{1}{2}}}{  \Gamma\lr{n}} \tilde{l}.
    \end{equation}
    and furthermore
    \begin{align}
    \overline{G}^{\sigma}_{\CL^{\perp}} &=  \int d\bs{x}\, p_{\sigma}\lr{\bs{x};\CL^{\perp}}  \widetilde{G}^2\lr{\bs{x}}  \nonumber \\
    &=\sum_{\bs{\xi}^{\perp} \in \CL^{\perp}} \widetilde{G}^2\lr{\bs{\xi}^{\perp}} +\epsilon\lr{\sigma} \\
    &= \overline{G}^{(2)}_{\CL^{\perp}}+2\epsilon\lr{\sigma}\nonumber
    \end{align}
    with the same bound for $\epsilon\lr{\sigma}$ and
    $\overline{G}^{(2)}_{\CL^{\perp}}\leq 2^{2n}\Tr\lrq{G^2}$.
\end{lem}
\endgroup
   
\proof
We have
\begin{align}
    \overline{G}^{\sigma}_{\CL^{\perp}}=N_{\CL^{\perp}}^{-1} \sum_{\bs{\xi}^{\perp}\in \CL^{\perp}} e^{-\frac{\sigma^2}{2} \|\bs{\xi}^{\perp}\|^2} \int d\bs{x}\, e^{-\frac{1}{2\sigma^2}\|\bs{x}-\bs{\xi}^{\perp}\|^2} \widetilde{G}\lr{\bs{x}}.\label{eq:proof_dev_1}
\end{align}
In each summand, we have to first order
\begin{align}
    \int d\bs{x}\, e^{-\frac{1}{2\sigma^2}\|\bs{x}-\bs{\xi}^{\perp}\|^2} \widetilde{G}\lr{\bs{x}} 
   & = \int d\bs{x}\, e^{-\frac{1}{2\sigma^2}\|\bs{x}-\bs{\xi}^{\perp}\|^2} \widetilde{G}\lr{\bs{\xi}^{\perp}} + \int d\bs{x}\, e^{-\frac{1}{2\sigma^2}\|\bs{x}-\bs{\xi}^{\perp}\|^2} \lr{\bs{x}-\bs{\xi}^{\perp}}^T\nabla \widetilde{G}\lr{\bs{\xi}^{\perp}}.
\end{align}
The first term in this expression simply evaluates to 
\begin{equation}
    \int d\bs{x}\, e^{-\frac{1}{2\sigma^2}\|\bs{x}-\bs{\xi}^{\perp}\|^2} \widetilde{G}\lr{\bs{\xi}^{\perp}} = (2\pi \sigma^2)^n \widetilde{G}\lr{\bs{\xi}^{\perp}}
\end{equation}
and contributes the term $\overline{G}_{\CL^{\perp}}$ to the final expression, where the normalization factor $N_{\CL^{\perp}}$ perfectly cancels out.

Using Lipschitz continuity and 
the boundedness of the Wigner 
functions, it holds that $\|\nabla \widetilde{G}\lr{\bs{x}}\| \leq \tilde{l}$. Together with the Cauchy-Schwartz inequality, we can bound the second term
\begin{align}
\left|
\int d\bs{x}\, e^{-\frac{1}{2\sigma^2}\|\bs{x}-\bs{\xi}^{\perp}\|^2} \lr{\bs{x}-\bs{\xi}^{\perp}}^T\nabla \widetilde{G}\lr{\bs{\xi}^{\perp}} \right| &\leq \tilde{l} \int d\bs{x}\, e^{-\frac{1}{2\sigma^2}\|\bs{x}-\bs{\xi}^{\perp}\|^2} \left\|\bs{x}-\bs{\xi}^{\perp}\right\| \\
&=\sqrt{2}\sigma \lr{2\pi \sigma^2}^n\frac{\Gamma\lr{n+\frac{1}{2}}}{ \Gamma\lr{n}} \tilde{l}.
\nonumber
\end{align}
 Combining this with the normalization and sum yields an expression for the error
\begin{align}
    |\epsilon\lr{\sigma}| \leq \sqrt{2}\sigma \frac{\sqrt{2} \Gamma\lr{n+\frac{1}{2}}}{  \Gamma\lr{n}} \tilde{l}.
\end{align}
Finally, to obtain the second moment, simply replace $\widetilde{G}$ in the previous derivation by $\widetilde{G}^2$ and notice that $\|\nabla \widetilde{G}\lr{\bs{x}}\| \leq \tilde{l}$ implies 
$\|\nabla \widetilde{G}^2\lr{\bs{x}}\| \leq 2\tilde{l}$. The last observation is that since $\bs{x}\mapsto \widetilde{G}^2\lr{\bs{x}}$ is a non-negative real function, it holds that
\begin{equation}
    \sum_{\bs{\xi}^{\perp} \in \CL^{\perp}} \widetilde{G}^2\lr{\bs{\xi}^{\perp}}\leq \int d\bs{x}\, \widetilde{G}^2\lr{\bs{\xi}^{\perp}} \leq 2^{2n}\int d\bs{x}\, G^2\lr{\bs{x}}=2^{2n}\Tr\lrq{G^2}.
\end{equation}

\section{Proof of Lemma~\ref{lem:scalar}}
Here, we prove the scalar product bound in Lemma~\ref{lem:scalar}. The main ingredient to the following derivation is a scalar product formula for functions on the Hilbert space associated to the space of lattices, $Y_n=\Sp_{2n}\lr{\Z}\backslash\Sp_{2n}\lr{\R}$, provided in ref.~\cite{Kelmer_2019}.

\begingroup
\renewcommand\thelem{2}
\begin{lem}[Scalar product bound]
    Let $f, g$ be two even compactly supported functions, it holds that

    \begin{equation}
        \left|\braket{\widetilde{F}_f, \widetilde{F}_g}_{Y_n}\right| \leq |\braket{f, 1}\braket{1, g}|+\frac{4\zeta\lr{n}^2}{\zeta\lr{2n}}\|f\|\|g\|.
    \end{equation}
\end{lem}
\endgroup

\proof
Define $f_k\lr{\bs{x}}=f\lr{k\bs{x}}$, $g_{k'}\lr{\bs{x}}=g\lr{k'\bs{x}}$. 
We have that
\begin{align}
    \left|\braket{\widetilde{F}_f, \widetilde{F}_g}_{Y_n} \right|&=\left|\sum_{k, k'\in \N} \braket{F_{f_k}, F_{g_{k'}}}\right| \\
    \nonumber
    &= \left|\sum_{k, k'\in \N}\frac{\braket{f_k, 1}\braket{1, g_{k'}}}{\zeta\lr{2n}^2} + \frac{2}{\zeta\lr{2n}}\lr{\braket{f_k, g_{k'}}+\braket{\iota \lr{f_k}, g_{k'}}}\right|\\
    \nonumber
    &=\left|\braket{f, 1}\braket{1, g} + \frac{2}{\zeta\lr{2n}}\sum_{k, k'\in \N}\lr{\braket{f_k, g_{k'}}+\braket{\iota \lr{f_k}, g_{k'}}}\right| \\
    \nonumber
    &\leq \braket{f, 1}\braket{1, g} +\frac{4}{\zeta\lr{2n}}\sum_{k, k'\in \N}\|f_k\| \|g_{k'}\|, 
    \nonumber
\end{align}
where we repeatedly use the triangle inequality, the fact that $\braket{f_k, 1}=k^{-2n} \braket{f, 1}, \braket{1, g_{k'}}=k'^{-2n} \braket{1, g} $ and the last line is derived using the Cauchy-Schwartz inequality $\braket{\iota \lr{f_k}, g_{k'}} \leq \|\iota \lr{f_k}\| \|g_{k'}\|=\|f_k\| \|g_{k'}\|$. 
Finally, note that it holds that
\begin{equation}
    \sum_k \|f_k\|=\sum_k \sqrt{\int d\bs{x} |f\lr{k\bs{x}}|^2} = \sum_k k^{-n}\sqrt{\int d\bs{x} |f\lr{\bs{x}}|^2} = \zeta\lr{n}\|f\|,
\end{equation}
which implies the final result. \endproof


\section{Proof of Lemma~\ref{lem:random_lat_1}}

\begingroup
\renewcommand\thelem{3}
\begin{lem}
    [Second moment: random lattice sampling]
    Let $\tilde{G}$ be an estimator as in Lemma~\ref{lem:rdn_lat_points} for an observable on a $n$-mode continuous variable quantum system and assume that the corresponding observable $G$ is such that 
    \begin{equation}
        \|G\|^2_2\coloneqq \int d\bs{x}\, |G\lr{\bs{x}}|^2<\infty 
    \end{equation}
    is finite. 
It holds that
\begin{equation}
\int_{Y_n}d\mu\lr{L}\int d\bs{x}\, p_{\sigma}\lr{\bs{x}; L} \widetilde{G}\lr{\bs{x}}^2 \leq 2^{2n}\lrc{1+\lr{\frac{2\pi}{\sigma^2}}^n} \|G\|_2^2.
\end{equation}
\end{lem}
\endgroup
    

\proof
Note that from $|W_{\rho}\lr{\bs{x}}| \leq 2^n$ we generally have $\widetilde{G}\lr{\bs{x}}^2\leq 2^{2n}G\lr{\bs{x}}^2$. Using $\Theta_L\lr{i\frac{\sigma^2}{4\pi}}\geq 1$ we can thus estimate

\begin{align}(2\pi\sigma^2)^{n}\int_{Y_n}d\mu\lr{L}\int d\bs{x}\, p_{\sigma}\lr{\bs{x}; L} \widetilde{G}\lr{\bs{x}}^2 &\leq 2^{2n}\int_{Y_n}d\mu\lr{L}\int d\bs{x}\, \lrc{e^{-\frac{1}{2\sigma^2}\|\bs{x}\|^2} + \widetilde{F}_{f_{\bs{x}}}} G\lr{\bs{x}}^2 \nonumber\\
&=2^{2n}\int d\bs{x}\, \lrc{e^{-\frac{1}{2\sigma^2}\|\bs{x}\|^2} + \braket{\widetilde{F}_{f_{\bs{x}}}, 1}_{Y_n}}G\lr{\bs{x}}^2
\end{align}
where we defined $f_{\bs{x}}\lr{\bs{v}}=e^{-\frac{\sigma^2}{2}\|\bs{v}\|^2-\frac{1}{2\sigma^2}\|\bs{x}-\bs{v}\|^2}$.
Using the mean value formula, it holds that 
\begin{align}
    \braket{\widetilde{F}_{f_{\bs{x}}}, 1}_{Y_n}&=\int d\bs{v}\, e^{-\frac{\sigma^2}{2}\|\bs{v}\|^2-\frac{1}{2\sigma^2}\|\bs{x}-\bs{v}\|^2}
    = \lr{\frac{2\pi}{\sigma^2+\sigma^{-2}}}^n e^{-\frac{\Sigma}{2} \|\bs{x}\|^2}\eqqcolon c_{\sigma}^n e^{-\frac{\Sigma}{2} \|\bs{x}\|^2},
\end{align}
where we have called the expression in the final bracket $c_{\sigma}=\frac{2\pi}{\sigma^2+\sigma^{-2}}$, which is bounded in the range $c_{\sigma}\in \lrq{0, \pi}$ and have defined $\Sigma:=\sigma^{-2}\lr{1-\lr{1+\sigma^4}^{-1}}=\frac{\sigma^2}{1+\sigma^4} \approx \sigma^2$.
Note that $\forall s>0$, 
it holds that
\begin{align}
    (2\pi s^2)^{-n}\int d\bs{x}\,e^{-\frac{1}{2s^2}} G\lr{\bs{x}}^2 \leq \int d\bs{x}\,G\lr{\bs{x}}^2\leq \int d\bs{x}\,G\lr{\bs{x}}^2\eqqcolon \|G\|_2^2,
\end{align}
such that, in particular, it at the same time is true that
\begin{equation}
    \int d\bs{x}\,e^{-\frac{1}{2s^2}} G\lr{\bs{x}}^2 \leq (2\pi s^2)^n \|G\|_2^2. 
\end{equation}
Inserting this inequality and using that $\Sigma^{-1}=\sigma^2+\sigma^{-2}$ yields the final bound
\begin{align}
    \int_{Y_n}d\mu\lr{L}\int d\bs{x}\, p_{\sigma}\lr{\bs{x}; L} \widetilde{G}\lr{\bs{x}}^2 \leq 2^{2n}\lrc{1+\lr{\frac{2\pi}{\sigma^2}}^n} \|G\|_2^2.
\end{align}


\endproof

\section{Second moment bound for exact inner means}\label{app:lem4}
In this section, we prove the following lemma.

\begin{lem}[Second moment: random lattice sampling for large $N_P$]
    Let $\widetilde{G}$ be an estimator as in Lemma~\ref{lem:rdn_lat_points} for an observable on a $n$-mode continuous variable quantum system and assume that the corresponding observable $G$ is such that 
    \begin{equation}
        \|G\|_1 \coloneqq \int d\bs{x}\, |G\lr{\bs{x}}|<\infty 
    \end{equation}
    is finite. 
   Let  
   \begin{equation}
       \overline{G}_L^{\sigma}= \int d\bs{x}\, p_{\sigma}\lr{\bs{x};\, L} \widetilde{G}\lr{\bs{x}}
   \end{equation}
   be the estimator from Lemma~\ref{lem:rdn_lat_points} and
    let $\sigma>0$ be a small parameter. It holds that
   \begin{equation}
        \left|\int_{Y_n} d\mu\lr{L}\lr{\overline{G}_L^{\sigma}}^2\right|= O\lr{\frac{\zeta\lr{n}^2}{\zeta\lr{2n}}\lr{16\pi\frac{\sigma^{-2}+\sigma^2}{\sigma^4}}^n \|G\|_1^2}.
    \end{equation}
\end{lem}

\proof
We start by computing the second moment
    \begin{align}
    \int d\mu\lr{L}\lrq{\int d\bs{x}\, p_{\sigma}\lr{\bs{x};\, L} \widetilde{G}\lr{\bs{x}}}^2 &= \int d\mu\lr{L}\int d\bs{x}d\bs{y}\, p_{\sigma}\lr{\bs{x};\, L}p_{\sigma}\lr{\bs{y};\, L} \widetilde{G}\lr{\bs{x}}\widetilde{G}\lr{\bs{y}}.
    \end{align}
Swapping the order of integration, we first 
compute
\begin{align}
    (2\pi\sigma^2)^{2n}\int d\mu\lr{L}p_{\sigma}\lr{\bs{x};\, L}p_{\sigma}\lr{\bs{y};\, L} &= \int d\mu\lr{L}\Theta_L^{-2}\lr{i\frac{\sigma^2}{4\pi}} \lrc{e^{-\frac{1}{2\sigma^2}\|\bs{x}\|^2}+\widetilde{F}_{f_{\bs{x}}} }\lrc{e^{-\frac{1}{2\sigma^2}\|\bs{y}\|^2}+\widetilde{F}_{f_{\bs{y}}} }, \nonumber \\
    &\leq \int d\mu\lr{L}\lrc{e^{-\frac{1}{2\sigma^2}\|\bs{x}\|^2}+\widetilde{F}_{f_{\bs{x}}} }\lrc{e^{-\frac{1}{2\sigma^2}\|\bs{y}\|^2}+\widetilde{F}_{f_{\bs{y}}}} \nonumber\\
    &= e^{-\frac{1}{2\sigma^2}\lr{\|\bs{x}\|^2+\|\bs{y}\|^2}}+ \braket{\widetilde{F}_{f_{\bs{x}}}, 1}_{Y_n} e^{-\frac{1}{2\sigma^2}\|\bs{y}\|^2}+\braket{1,\widetilde{F}_{f_{\bs{y}}}}_{Y_n}
    \nonumber\\
    &\times e^{-\frac{1}{2\sigma^2}\|\bs{x}\|^2} + \braket{\widetilde{F}_{f_{\bs{x}}},\widetilde{F}_{f_{\bs{y}}}}_{Y_n}
\end{align}
where  we denote again $f_{\bs{x}}\lr{\bs{v}}=e^{-\frac{\sigma^2}{2}\|\bs{v}\|^2-\frac{1}{2\sigma^2}\|\bs{x}-\bs{v}\|^2}$ and have used that 
$\Theta_L\lr{i\frac{\sigma^2}{4\pi}} \geq 1$. Here, we have
 inserted the relations from the proof of Lemma~\ref{lem:random_lat_1}
%
Using Lemma~\ref{lem:scalar}, we also obtain the bound
\begin{align}
\left|\braket{\widetilde{F}_{f_{\bs{x}}},\widetilde{F}_{f_{\bs{y}}}}_{Y_n}\right| &\leq |\braket{f_{\bs{x}}, 1}\braket{1, f_{\bs{y}}}|+\frac{4\zeta\lr{n}^2}{\zeta\lr{2n}}\|f_{\bs{x}}\|\|f_{\bs{y}}\|.
\end{align}
We have already computed $\braket{f_{\bs{x}}, 1}=\braket{1, f_{\bs{y}}}$. Similarly, we obtain by completing the square
\begin{align}
    \|f_{\bs{x}}\|^2 = \int d\bs{v} e^{-\sigma^2\|\bs{v}\|^2 -\sigma^{-2}\|\bs{x}-\bs{v}\|^2} = \lr{\frac{c_{\sigma}}{2}}^n e^{-\Sigma \|\bs{x}\|^2}.
\end{align}
 Taking everything together and using the triangle inequality gives rise to the bound
 \begin{align}
     \left|(2\pi\sigma^2)^{2n}\int d\mu\lr{L}p_{\sigma}\lr{\bs{x};\, L}p_{\sigma}\lr{\bs{y};\, L}\right| 
     &\leq e^{-\frac{1}{2\sigma^2}\lr{\|\bs{x}\|^2+\|\bs{y}\|^2}}+ c_{\sigma}^n e^{-\frac{\Sigma}{2} \|\bs{x}\|^2} e^{-\frac{1}{2\sigma^2}\|\bs{y}\|^2}+c_{\sigma}^n e^{-\frac{\Sigma}{2} \|\bs{y}\|^2}e^{-\frac{1}{2\sigma^2}\|\bs{x}\|^2} \nonumber\\ &\hspace{1cm} +
     c_{\sigma}^{2n} e^{-\frac{\Sigma}{2} \lr{\|\bs{x}\|^2+\|\bs{y}\|^2}}
     +\frac{4\zeta\lr{n}^2}{\zeta\lr{2n}} 
     \lr{\frac{c_{\sigma}}{2}}^n e^{-\frac{\Sigma}{2}\lr{\|\bs{x}\|^2+\|\bs{y}\|^2}}.
 \end{align}
Again using that $\forall s>0$,s it holds that
\begin{align}
    (2\pi s^2)^{-n}\int d\bs{x}\,e^{-\frac{1}{2s^2}} |\widetilde{G}\lr{\bs{x}}| \leq \int d\bs{x}\,|\widetilde{G}\lr{\bs{x}}|\leq 2^n\int d\bs{x}\,|G\lr{\bs{x}}|\eqqcolon 2^n\|G\|_1,
\end{align}
such that in particular it also holds that
\begin{equation}
    \int d\bs{x}\,e^{-\frac{1}{2s^2}} |\widetilde{G}\lr{\bs{x}}| \leq (4\pi s^2)^n \|G\|_1. 
\end{equation}
We can now combine all the above elements and obtain
\begin{align}
    \left|\int d\mu\lr{L}\lrq{\int d\bs{x}\, p_{\sigma}\lr{\bs{x};\, L} \widetilde{G}\lr{\bs{x}}}^2\right| &\leq 2^{2n}\lrq{1+2\lr{\frac{\Sigma^{-1}}{\sigma^2}}^n+c_{\sigma}^{2n}\lr{\frac{\Sigma^{-1}}{\sigma^2}}^{2n} +\frac{4\zeta\lr{n}^2}{\zeta\lr{2n}}\lr{\frac{c_{\sigma}}{2}}^n  \lr{\frac{\Sigma^{-1}}{\sigma^2}}^{2n}} \|G\|_1^2\\
    &=2^{2n}\lrq{1+2\lr{\sigma^{-4}+1}^n+\lr{\frac{2\pi}{\sigma^2}}^{2n} +\frac{4\zeta\lr{n}^2}{\zeta\lr{2n}}\lr{4\pi\frac{\sigma^{-2}+\sigma^2}{\sigma^4}}^n} \|G\|_1^2,\nonumber
\end{align}
 where the last line we have inserted the definition $\Sigma^{-1}=\sigma^2+\sigma^{-2}$.

 \endproof

\section{Proof of Theorem~\ref{them:Randomlatshadow}}

\begingroup
\renewcommand\thethem{3}
\begin{them}[Random lattice CV shadows]
    Let $\tilde{\epsilon}, \delta, \sigma> 0$ be small parameters, let $G_{m},\, m=1,\hdots, M$ be operators with finite $\|G_m\|_2^2$ and set $K=2\log\lr{2M/\delta}$ and $B=2^{2n}\lrc{1+\lr{\frac{2\pi}{\sigma^2}}^n}\max_m \|G_m\|_2^2 /\tilde{\epsilon}^2$ . Then $N=KB$ samples from the distribution of phase space points $\lrc{\bs{x}_i}_{i=1}^N$ sampled according to protocol in fig.~\ref{prot:CV_shadow} approximate the expectation values $\overline{G}_{m}=\Tr\lrq{\rho G_m}+\tilde{G}_m\lr{0}$ of an arbitrary state on an $n$-mode continuous variable quantum system up to a photon-parity offset $\tilde{G}\lr{0}$ and error $|\epsilon\lr{\sigma}|+\tilde{\epsilon}$ with probability at least $1-\delta$.
\end{them}
\endgroup
\proof

We proceed by a medians of means strategy equivalent that in ref.~\cite{Huang_2020}. 
Block the $N=KB$ estimates into $K$ batches each of size $B$. As per the result of Lemma~\ref{lem:rdn_lat_points} and using Chebychevs inequality the arithmetic mean $\widehat{G}_k$ of each of these batches approximates the final mean $\overline{G}$ up to an error of size $|\epsilon\lr{\sigma}|$ from Lemma~\ref{lem:rdn_lat_points} with failure probability
\begin{equation}
    P\lr{|\widehat{G}_k-\overline{G}| > |\epsilon\lr{\sigma}|+\tilde{\epsilon}} \leq \frac{\lrc{1+\lr{\frac{2\pi}{\sigma^2}}^n} \|G\|_2^2}{B\tilde{\epsilon}^2}.
\end{equation}
Choosing $B\geq2^{2n}\lrc{1+\lr{\frac{2\pi}{\sigma^2}}^n} \|G\|_2^2 /\tilde{\epsilon}^2 $
and using Hoeffing's bound, the probability of deviation of the median of these estimates 
\begin{equation}
    G_{\rm MoM}\coloneqq {\rm median}\lrc{\widehat{G}_1,\hdots, \widehat{G}_K}
\end{equation}
from the real mean is bounded by
\begin{equation}
    P\lr{|G_{\rm MoM}-\overline{G}| \geq |\epsilon\lr{\sigma}|+\tilde{\epsilon}} \leq 2e^{-K/2}
\end{equation}
for all $\tilde{\epsilon}>0$. Choosing $K=2\log\lr{2M/\delta}$ suppresses the failure probability uniformly for $M$ observables.
\endproof


\section{A sketch of a scheme for GKP encoded private quantum channels}\label{app:PQC}

In this appendix we outline yet another application of the CV twirling techniques discussed in this manuscript. 
The idea of realizing channel twirls on the logical level
of GKP encoded logical information
by means of operators implemented on the physical level
is expected to give rise to applications in their own
right, independent from the concrete implementations 
for the use in GKP logical shadows specifically discussed here. For example, this framework also immediately invites the new application of a GKP encoded \emph{private quantum channel}
\cite{PrivateQuantumChannel, conrad2023good}. When secret classical information may be shared between the sender and receiver party of such a scheme, the implementation of a \emph{quantum-one-time-pad} (QOTP) \cite{childs_2005}, where sender and receiver apply the same random unitary operation (its inverse) to a quantum channel running through public space, may be thought of a channel twirl. In terms of qubit-based systems, as discussed in the main text, it is known that the Clifford twirl projects any quantum channel onto a depolarizing channel, which for sufficiently large depolarizing probabilities is entanglement breaking \cite{Lieb1973, Bravyi_2019}, which is a feature that can aid in removing spurious correlations to an unwelcome party attached to the system in the first place and it randomizes the individual-shot information to a sufficient degree that an eavesdropper, unaware of the unitary picked, had no access to the transmitted information. The techniques in the main text readily translate this setup into a GKP encoded version, such that privacy features are natively combined with error correcting capabilities. 
\end{widetext}


\end{document}

