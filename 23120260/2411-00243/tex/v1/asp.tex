\section{Adiabatic State Preparation}
\label{sec:asp}
Adiabatic state preparation (ASP) is a common reference
when dealing with state-preparation algorithms,
because of its relative simplicity and robustness.
The algorithm allows one to initialize a set of qubits
to the ground state of a chosen ``target'' Hamiltonian
without the need for any additional qubits.
What is needed instead is to find a ``simple'' Hamiltonian
that possesses a readily achievable ground state.
The simple Hamiltonian should be related to the full target
one in such a way that a set of coefficients can be used to interpolate between them.

ASP works by applying a set of time evolution
operators each with a different Hamiltonian, which come from a
discretization of the interpolation between the simple and the target Hamiltonians.
This allows the state to evolve towards the desired target ground state
by always remaining close to the ground state of the interpolating Hamiltonians
at all time steps.

In the Schwinger model, we select the initial Hamiltonian as follows:
\begin{equation}
    H_0 = H_{ZZ} + H_Z|_{m\rightarrow m_0, \theta \rightarrow 0}\,,
\end{equation}
whose ground state is a state of alternating spins.
The parameters of the Hamiltonian, $w=1/(2a)$, $\theta$ and $m$,
are rewritten to be adiabatic dependent on an additional
variable called ``adiabatic time'' $t$:
\begin{equation}
        w\rightarrow \frac{t_i}{T} w\quad\theta\rightarrow \frac{t_i}{T} \theta\quad
    m\rightarrow \left(1 - \frac{t_i}{T} \right)m_0 + \frac{t_i}{T} m\,,
    \label{eq:linear}
\end{equation}
for a set of $i=1,\ldots, M$ values of the adiabatic time $t_i = t_{i-1} + \delta t_i$
with time intervals $\delta t_i$. The end time of the adiabatic evolution is
denoted by $T = \sum_{i=1}^{M} \delta t_i$.

The simplest discretization is a linear one,
corresponding to taking all $\delta t_i$ to be the same, $\delta t_i = T/M$,
but other nonuniform values for $\delta t_i$ can be chosen.

To prepare the desired quantum state, we define an adiabatic time-evolution operator $U(t, t+\delta t) = e^{-iH_A(t)\delta t}$,
where $H_A(t)$ is the interpolating adiabatic Hamiltonian at adiabatic time $t$.
We distinguish between two types of linear discretizations
of the adiabatic evolution, labeled ``L1'' and ``L2'' according to
the order of the Trotter product formula used to
evaluate the evolution operator $U(t, t+\delta t)$.

We also consider other discretizations in this paper, given by
\begin{equation}
\label{eq:sin2}
\delta t_i = 2 \frac{T}{M} \sin^2\left(\pi \frac{i}{M}\right)\,,
\end{equation}
\begin{equation}
\label{eq:cos2}
\delta t_i = 2 \frac{T}{M}\cos^2\left(\pi \frac{i}{2M}\right)\,,
\end{equation}
and
%\begin{equation}
%\label{eq:tan2}
%t_i = \frac{T}{M}\frac{\tanh(i/M)}{\tanh(1)},,
%\end{equation}
which we denote respectively by ``S1'' (or ``S2'') and ``C1'' (or ``C2'')
%and T1 (or T2)
depending on whether we use a first-order or a second-order
Trotter product formula for the adiabatic time evolution.

These choices are made in order to achieve discretizations
that are either denser or sparser at different stages of the adiabatic evolution.
The S1, S2 discretizations are denser at the beginning and the end of the
adiabatic evolution, while the C1, C2 choices are sparser at the beginning
of the adiabatic evolution and denser at the end.

It is common practice, in the field of quantum computing, to count the number of two-qubit gates, typically CNOT gates, as a quantity that measures the length of the algorithm as they have a smaller fidelity on current quantum hardware and hence dominate the noise. For the Hamiltonian we are considering, the number of CNOT gates required for a single first-order Trotter step are given by $4(N-1) + (N-1)(N-2)$, where $N$ is the number of sites of the lattice. The first term comes from the hopping term $H{\pm}$; the second comes from the nonlocal term $H_{ZZ}$ and crucially grows quadratically in $N$. The mapping of the terms of the Hamiltonian to quantum gates is explained in~\cite{Chakraborty:2020uhf}.

For reference, in Tab.~\ref{tab:cnot}, we report the average CNOT gate count per qubit for each step of the adiabatic time 
evolution for system with $N=4,8,12$ and both 
first- and second-order Trotter discretizations, which differ by a simple factor of two. The quadratic scaling coming from $H_{ZZ}$ dominates already for small values of $N$, making the average number of gates grow linearly with the size of the problem.
\begin{table}[h]
    \begin{center}
        \begin{tabular}{ccc}
            $N$  & 1st Order & 2nd Order \\\hline 
            $4$  & $4.5$     & $9$       \\
            $8$  & $8.8$     & $17.5$    \\
            $12$ & $12.8$    & $25.7$     
        \end{tabular}
        \caption{Average number of CNOT gates per qubit for each step of the adiabatic time.
        $N$ is the number of points in the spatial direction and the order refers to trotterization adopted.}
        \label{tab:cnot}
    \end{center}
\end{table}


\subsection{Numerical Results}

We investigate the quantum algorithm for ASP
on a classical computer, exploring how the choice of discretization and total
evolution time $T$ influence the quality of the prepared ground state.
To assess our algorithms, we consider the energy and overlap with the
exact ground state obtained from diagonalization. We focus in particular
on small values of $M$ and $T$ for two systems with $N=4,8$, which lead to
short overall algorithms.

We use the Qiskit software package~\cite{qiskit} for our numerical simulations. 
This choice allows to use a gate-based representation of the algorithms and to also 
access the state vector during the simulations.
There are two parameters that significantly impact the final approximation.
The first is the ratio $\delta t / T$, which determines the speed
of the system's evolution between the initial and the target Hamiltonian.
A large ratio reduces the number of steps required,
but at the cost of lower precision in the intermediate states,
as they are further away from their eigenstates.
A small ratio, on the other hand, ensures the state remains
close to the intermediate ground states during the evolution,
but requires more steps to achieve convergence.

The second parameter is $\delta t$, which affects the ASP algorithm
by influencing the accuracy of the time evolution operators
used to approximate the Hamiltonian with the Trotter product formula.
This error can be reduced with higher-order schemes,
but this comes at the cost of longer algorithms per time step.

\begin{figure}[!hbpt]
    \centering
    \includegraphics[width=1\linewidth]{figures/E_4-10-T5.pdf}
    \includegraphics[width=1\linewidth]{figures/GSO_4-10-T5.pdf}
    \caption{\footnotesize
    Ground-state energy $E_A(t)$ (top) and overlap with the true ground state,  $\omega(t)$ (bottom),
    versus adiabatic time $t$ for the
    Schwinger model with a theta term.
    The exact results, represented by the solid line,
    are obtained through exact diagonalization of the adiabatic Hamiltonian
    at each value of $t$.
    Various discretizations are employed,
    as explained in the main text.}
    \label{fig:asp_energy}
\end{figure}



In Fig.~\ref{fig:asp_energy}, we present the time evolution
of the ground-state energy, $E_A(t)$,
using different discretizations with $T=5$ and $M=10$.
The exact values are computed by direct diagonalization of the intermediate Hamiltonian
at every time step. As shown, all the discretizations exhibit good agreement with
the exact results even with as few as $M=10$ steps.
However, we observe that the C2 and C1
discretizations yield better agreement, indicating that it may be beneficial to
perform more precise steps towards the end of the evolution, where the Hamiltonian
is closer to the target one.

The overlap $\omega(t)$ between the evolved
state $\ket{\psi_A(t)}$ and the exact ground state
$\ket{\psi_0(t)}$ of the adiabatic Hamiltonian
obtained via full diagonalization at each time $t$, is
$\omega(t)=|\braket{\psi_A(t)|\psi_0(t)}|^2$.
The plot reveals that $\omega$ is always within $10\%$
of the exact ground state,
indicating that the evolution successfully
keeps the state close to the intermediate ground
state throughout the process.
Notably, we find that discretization C1 offers
the most precision throughout the adiabatic
evolution but loses precision towards the end,
whereas C2 yields the best approximation at the end of the evolution.

\begin{figure}[!htpb]
    \centering
    \includegraphics[width=1\linewidth]{figures/convergence_T50_N4.pdf}
    \includegraphics[width=1\linewidth]{figures/convergence_T50_N4_step.pdf}
    \caption{\footnotesize
    Relative error of the ground state energy for the same simulation parameters as Fig.~\ref{fig:asp_energy} for the L1 and L2 discretizations. The top figure shows the error as a function of the total number of steps ($T/\delta T$) for a fixed set of total times $T$. The lower panel shows the error just as a function of $\delta T$ for the same set of values for $T$.}
    \label{fig:asp_convergence}
\end{figure}

We investigate the dependence of the total adiabatic time evolution $T$
on the variation of the number of steps $M$.
The top panel of Fig.~\ref{fig:asp_convergence} displays the relative error
for the ground state energy $E_A(T)$ obtained at the end of the
adiabatic evolution as a function of $M$.
The different colors correspond to distinct adiabatic time evolutions $T$,
with linear discretization orders of L1 or L2.
As expected, a larger $T$ results in smaller errors.
However, we also observe that there is a threshold value of $M$ for a
given $T$ at which the relative error is minimized and after which increasing the number of steps indefinitely
does not further reduce the relative error. This is fixed by $T$ and is unaffected by the order of the trotterization.

In the bottom panel of Fig.~\ref{fig:asp_convergence} the difference between first-order and second-order trotterization is apparent, as the two classes have different slopes. This indicates that using second-order trotterization allows smaller values of $\delta t$, as expected, as long as the total adiabatic time $T$ is large enough for the algorithm to be in the scaling regime.

Overall, ASP is a reliable method for preparing the ground state of our 
Hamiltonian, however, the required algorithm is limited by the bounds
imposed by the adiabatic theorem, which then requires longer evolution
times and hence more steps and longer quantum algorithms.

\subsection{Noise Model for Adiabatic State Preparation}
\label{subsec:asp_noise}

One major challenge of ASP is the length of the algorithm resulting
from the adiabatic time evolution discretization.
On noisy intermediate-scale quantum (NISQ) machines, such algorithms can become prohibitively
expensive due to the presence of many CNOT gates with relatively low precision.
To investigate the impact of CNOT gate errors on ASP,
we conducted a series of simulations using Qiskit (using
discretization L2)
where we varied the error rate of the CNOT gates.
Fig.~\ref{fig:asp_noise} shows the results of this study for
$T=5$ and $M=10$ where
we varied the error rate of CNOT gates from $10^{-2}$ to $0$, i.e~ noiseless.
The goal is to determine what error rate would allow the ASP algorithm
to produce a good estimate of the ground state of the $N=4$ Schwinger model.
The plot compares the resulting ground-state energy to the exact results,
and it is clear that, at current error rates 
ranging from $10^{-2}$ to $10^{-3}$~\cite{Kandala2021}, 
the noise significantly hinders the algorithm's
ability to produce a high-quality state.

\begin{figure}[h!]
    \centering
    \includegraphics[width=1\linewidth]{figures/energy-noise.pdf}
    \caption{\footnotesize
    Ground state energy from ASP with L2 discretization with varying CNOT
    error rate. The parameters chosen are the same as in Fig.~\ref{fig:asp_energy}
    with $T=5$ and $M=10$.}
    \label{fig:asp_noise}
\end{figure}
Fig.~\ref{fig:asp_noise_time} illustrates how the ground-state energy,
calculated with L2 discretization at the end of the
adiabatic evolution $T=5$ (with $M=10$)
depends on the error rate of the CNOT gates.
In the case of a large CNOT error rate,
the wave function tends towards a maximally disordered state,
resulting in an energy close to $0$.
The figure also demonstrates that extrapolating the error rates
from the current accessible ones (indicated by a band)
to regions where the results become independent on them is not currently possible.
It will be very difficult to perform an extrapolation
unless error rates improve by 1--2 orders of magnitude.

\begin{figure}[hbt!]
    \centering
    \includegraphics[width=1\linewidth]{figures/noise-scan.pdf}
%    \captionsetup{justification=justified}
    \caption{\footnotesize
    Energy of the ground state from ASP at the end of adiabatic
    evolution, $T=5$, with varying CNOT error rate.
    The band indicates state-of-the-art CNOT error gates~\cite{IBM2023}.}
    \label{fig:asp_noise_time}
\end{figure}

To summarize, the limits of current quantum hardware further limit the
applicability of ASP to our system, making a case for the need of shorter
and more efficient algorithms.

%There is insufficient information to perform the extrapolation.
%However, improving the error rates by two orders of magnitude
%would be sufficient to enable this procedure.


%This effect presents a secondary limit for high error rates,
%making extrapolations to a vanishing error rate impractical.
%This is further illustrated in Fig.~\ref{fig:asp_noise_time}.

