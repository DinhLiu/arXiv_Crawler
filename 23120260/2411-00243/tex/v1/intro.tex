\section{Introduction}
\label{sec:intro}

Despite the enormous success of the Standard Model (SM) of particle physics, many unanswered
questions remain, such as what constitutes dark matter (and dark energy),
or the observed amount of matter-antimatter asymmetry of the Universe.
It was established many years ago, thanks to the work of Sakharov~\cite{Sakharov:1967dj},
that the fundamental interaction and mechanism at the origin of the baryon asymmetry in the Universe
must satisfy three conditions, known as the Sakharov conditions.
An important aspect of these conditions is that the fundamental interaction should
break CP symmetry. While the SM contains, due to quark-flavor mixing, a direct source
of CP violation, the strength of this violation is not sufficient to explain the
observed asymmetry~\cite{Gavela:1993ts,Huet:1994jb}.
The search for physics beyond the Standard Model (BSM) thus includes a search for
new sources of CP violation. A prototypical example of a CP-violating interaction
that has several phenomenological implications (e.g.~axions, electric dipole moments, 
the baryon asymmetry of the universe)
is the $\theta$ term of the strong interactions described by quantum chromodynamics (QCD).

QCD is a theory asymptotically free at short distances, where it is amenable to a perturbative treatment,
but at low energy becomes nonperturbative, and to this day the only way to perform calculations
of QCD at low energy with systematically improvable uncertainties
is to regulate the theory on a four-dimensional lattice (lattice QCD) and solve it numerically.
Lattice QCD (LQCD) is a mature field with a well-defined selection of problems that it can address,
but also with clear obstacles that nowadays seem insurmountable. One of these
difficulties is the impossibility to solve theories that possess a complex Euclidean action,
because it prevents the use of stochastic methods to properly sample the field space.
A classic example of complex action in Euclidean space is QCD with a $\theta$ term.
To circumvent this problem, it is common to expand the theory in powers of the
parameter $\theta$, which is known to be smaller than about 
$10^{-10}$~\cite{Abel:2020gbr,Dragos:2019oxn},
and treat the CP-violating interaction as a perturbation of QCD.\ 
While calculations of the neutron electric dipole moment have been done this way
~\cite{Dragos:2019oxn,Alexandrou:2020mds,Bhattacharya:2021lol,Liang:2023jfj}, 
confirming the smallness of the $\theta$ parameter, 
ideally we would like to
have a computational framework that is able to deal with a generic complex action,
or even better that does not need the theory to be rotated to Euclidean space,
rather allowing real-time simulations of the theory prepared in Minkowski space.
The impossibility of simulating complex actions is just one of the aspects of what nowadays
goes under the very general definition of ``sign problem''.\

A solution, at least in theory, is represented by the Hamiltonian formalism, which deals
directly with real-time systems, but it cannot be simulated using state-of-the-art supercomputers,
due to the large number of degrees of freedom stemming from the regulated infinite-dimensional
Hilbert space. The simulation of a theory using the Hamiltonian formalism seems to be a problem
perfectly suited for a quantum computer, and even though the number of qubits currently available
is still far too low to directly simulate QCD in $3+1$ dimensions,
it is possible to use lower-dimensional field theories that share certain properties with QCD
to test new ideas and algorithms.
It is particularly important to understand the scaling of the algorithms with increasing
size of the system and towards the continuum limit.

In this work we study quantum electrodynamics (QED) in $1+1$ dimensions,
also known as the Schwinger model~\cite{Schwinger:1962tp},
with the addition of a $\theta$ term.
The Schwinger model shares many interesting properties with QCD,
such as confinement of fermions and the spontaneous breaking of the $\mathrm{U}(1)$ symmetry
with a corresponding chiral condensate, so it is an ideal toy model to test
an algorithmic or computational paradigm.

\begin{comment}
    The field of quantum computing for field theories is making significant advances\cite{jordan_quantum_2012,jordan_quantum_2014, marcos_two_dimensional_2014,wiese_towards_2014,garcia-alvarez_fermion-fermion_2015,mezzacapo_non-abelian_2015,martinez2016,yang_analog_2016,muschik_u1_2017,klco_quantum-classical_2018,lamm_simulation_2018,alexandru_sigma_2019,gustafson_quantum_2019,jordan_quantum_2019,klco_digitization_2019,avkhadiev_accelerating_2020,davoudi_towards_2020,klco_minimally-entangled_2020,klco_su2_2020,lamm_parton_2020,surace_lattice_2020,atas_su2_2021,ciavarella_trailhead_2021,gustafson_benchmarking_2021,mazzola_gauge_2021,rahman_su2_2021,yamamoto_quantum_2021,andrade_engineering_2022,asaduzzaman_quantum_2022,chakraborty_classically_2022,ciavarella_conceptual_2022,ciavarella_preparation_2022,de_jong_quantum_2022,halimeh_tuning_2022,honda_classically_2022,jensen_dynamical_2022,mildenberger_probing_2022,nguyen_digital_2022,rahman_self-mitigating_2022,shen_simulating_2022,xie_variational_2022,avkhadiev_strategies_2023,ciavarella_quantum_2023,farrell_preparations_2023,ikeda_detecting_2023,mueller_quantum_2023,nagano_quench_2023,pomarico_dynamical_2023,sakamoto_end--end_2023,zhang_observation_2023,angelides_first-order_2024,meth_simulating_2024,oshima_twirling_2024,popov_variational_2024,schuster_studying_2024}. 
    
\end{comment}
The field of quantum computing for field theories, like the Schwinger model~\cite{Chakraborty:2020uhf,Farrell:2024fit,Desaules:2022ibp,Bauer:2023qgm,Farrell:2023fgd,Pomarico:2023png,Florio:2023dke,Xie:2022jgj,Nguyen:2021hyk,deJong:2021wsd,Honda:2021ovk,Honda:2021aum,Banuls:2019bmf,Kokail:2018eiw,Klco:2018kyo,Lu:2018pjk,Martinez:2016yna,Kuhn:2014rha,Hauke:2013jga}, is making significant advances. A possible approach is to rely on discretizing the spatial coordinates, allowing the form of the discretized Hamiltonians of the system to be determined using standard techniques currently employed in lattice QCD. We have used the Schwinger model with a $\theta$ term to test several quantum algorithms for quantum state preparation with particular emphasis on the scaling
with the size of the problem and the number of qubits.
In Sec.~\ref{sec:model} we introduce the discretization of the Schwinger model on a quantum system,
while in Sec.~\ref{sec:asp} we study the first algorithm, the adiabatic state evolution.
In Sec.~\ref{sec:qaoa} we study the quantum approximate optimization algorithm (QAOA)
and in Sec.~\ref{sec:rodeo} the newly proposed Rodeo algorithm~\cite{Choi:2020pdg}.
In Sec.~\ref{sec:conclusion} we present our summary and conclusions.

%The Schwinger model \cite{Schwinger:1962gauge}
%has been of interest as a toy model for Quantum Chromodynamcs (QCD) as they
%share significant properties such as confinement.
