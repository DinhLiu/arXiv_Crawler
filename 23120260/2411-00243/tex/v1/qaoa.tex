\section{Quantum Approximate Optimization Algorithm}
\label{sec:qaoa}
The quantum approximate optimization algorithm (QAOA)~\cite{farhi2019quantum} is a method that relies on the variational principle to solve optimization problems. State preparation can be cast to such a problem by giving a parametrized ansatz for the ground state wave function and then optimizing its parameters variationally. One of the most prominent benefits it offers is the relatively short length of the algorithms and the small number of parameters it requires.

Similar to ASP, the starting point is a trivially solvable Hamiltonian $H_0$ with eigenstate $\ket{\psi_0}$. The QAOA ansatz for the ground state of the target Hamiltonian is
\begin{equation}
    \label{eq:ansatz}
    \ket{\psi_M( \vec{\gamma}, \vec{\beta})} = \left(\prod_{k=0}^{M-1}e^{-i\beta_{M-k} H_0}e^{-i\gamma_{M-k} H}\right)\ket{\psi_0},
\end{equation}
where the $2M$ real coefficients $\vec{\beta}, \vec{\gamma}$ parametrize the wave function. From the variational principle, we know that given the parameters $\vec{\gamma}^*$ and $\vec{\beta}^*$, the expectation value of the Hamiltonian operator is
\begin{equation}
    \bra{\psi_M( \vec{\gamma}^*, \vec{\beta}^*)}H
    \ket{\psi_M( \vec{\gamma}^*, \vec{\beta}^*)} = E_0^V \geq E_0,
    \label{eq:qaoa}
\end{equation}
where $E_0$ is the true ground state of the system. This means that we can use a minimization algorithm; in our case, we used simulated annealing~\cite{kirkpatrick1983optimization}, because it is suitable for the multiple local minima of the problem. The minimization is performed classically and not on quantum hardware, though in the future a hybrid classical-quantum algorithm could be feasible as well.

As opposed to ASP, the length of the QAOA is chosen as a parameter instead of having to find the optimal number of Trotter steps. The precision of the results then depends on the quality of the optimal solution found by the minimizer, not on the length of the algorithm.

\subsection{QAOA Results}
We applied the QAOA to the case of the Schwinger model with the same parameters as Sec.~\ref{sec:asp}, so $(m,m_0,w,J,\theta) = (0,0.5,0.5,0.5,0)$ and $(1,0.5,0.5,0.5,\pi/4)$. Table~\ref{table:qaoa} displays the results for the energy expectation value and ground-state overlap for the ground state, obtained using the second-order Trotter formula, with two and three QAOA steps.

For the given ansatz, the total number of CNOT gates is given by $M(8(N-1) + (N-1)(N-2))$ for the first-order trotterization and twice as many for second-order. This is the same quadratic scaling as ASP, considering that the dominant term is given by $H_{ZZ}$. However, the total number of CNOT gates is much reduced because the number of steps $M$ is fixed beforehand to a very small number. We used $M=2,3$, which is orders of magnitude smaller than those used in ASP.\

\begin{table}[!hbtp]\footnotesize
    \centering
    \setlength\tabcolsep{5pt} % Spread columns a little more
    \begin{tabular}{ccccccc}
        \makecell{\\Method} & \makecell{\\$N$} & \makecell{\\$(\theta,m)$} & \makecell{\\$M$} & \makecell{CNOT/\\qubit} & \makecell{Rel.\\Err. $E_0$} & \makecell{GS\\Overlap} \\\hline 
        QAOA   & $4$ & $(0,0)$        & $2$ & $30$       & $0.0029$        & $0.9798$   \\
        QAOA   & $4$ & $(0,0)$        & $3$ & $45$       & $0.0012$        & $0.9928$   \\
        QAOA   & $4$ & $(\pi/4,0.1)$  & $2$ & $30$       & $0.0023$        & $0.9855$   \\
        QAOA   & $4$ & $(\pi/4,0.1)$  & $3$ & $45$       & $0.0021$        & $0.9871$   \\\hline
        QAOA   & $8$ & $(0,0)$        & $2$ & $49$       & $0.0021$        & $0.9679$   \\
        QAOA   & $8$ & $(0,0)$        & $3$ & $73.5$     & $0.0016$        & $0.9856$   \\
        QAOA   & $8$ & $(\pi/4,0.1)$  & $2$ & $49$       & $0.0077$        & $0.9732$   \\
        QAOA   & $8$ & $(\pi/4,0.1)$  & $3$ & $73.5$     & $0.0067$        & $0.9789$   \\
    \end{tabular}
    \caption{\footnotesize Results for the QAOA method for varying system sizes and parameters with first-order Trotter formula. Here $N$, $\theta$ and $m$ are the parameters of the simulation, and $M$ is the number of QAOA steps. The remaining columns give the CNOT gates per qubit, the relative error of the ground state energy measurement, and the ground state overlap.}~\label{table:qaoa}
\end{table}

The results show that QAOA is highly efficient in preparing the ground state and estimating its energy, using significantly fewer CNOT gates compared to ASP. Note that the values in Tab.~\ref{table:qaoa} refer to the total length of the algorithm, while in Tab.~\ref{tab:cnot} the count was per step.\ Specifically, for the $M=10$ case of ASP, the second-order approximation results in $90$ CNOT gates per qubit for $N=4$, leading to less accurate results when compared to QAOA, which in turn requires only $48$ CNOT gates with 3 steps. 

The downside of QAOA, however, is the minimization process, which requires an accurate determination of the optimal parameters. Although classical machines can simulate the process with exact exponentiation for small systems, these become problematic for larger systems. Moreover, it is uncertain whether the minimization process can be performed efficiently in the hybrid computation scenario with NISQ machines. Furthermore, the results reported in the table are the best results out of five runs of the optimizer, as the stochasticity of the simulated annealing method leads to slightly different results. In principle, the variational system of Eq.~\ref{eq:qaoa} can have several local minima, and any algorithm could get stuck in one of them instead of the global minimum which corresponds to the true ground state. However, the qualitative conclusions drawn from comparing QAOA with ASP remain unchanged, as the former is orders of magnitude better for comparable algorithm lengths. In summary, the results indicate that once a set of optimal coefficients $\vec{\gamma}^*$, and $\vec{\beta}^*$ is determined, QAOA is more effective than ASP in preparing the ground state.

\subsection{Blocked QAOA}
\label{sec:blocked-qaoa}
To reduce the number of CNOT gates per qubit, an option is to employ custom optimized 2-qubit gates, which can be tailored for any unitary operation on two qubits, as proposed in~\cite{vatan2004optimal, shende2004minimal}. However, because of nonlocality of the Hamiltonian in Eq.~(\ref{eq:hamiltonian}) owing to the presence of the $H_{ZZ}$ term, this task is not straightforward. Nevertheless, the QAOA algorithm can still be used since it relies on a relatively general ansatz. To handle the nonlocal term, we introduce a modified Hamiltonian, denoted as ``blocked'' or $H_B$, which retains only the diagonal and nearest-neighbor terms of the full Hamiltonian. Thus, defining $H_B = H_\pm + H_Z + H_{ZZ}^l$ this can be represented by a single 2-qubit gate, 
where $H_{ZZ}^l$ is the local part of $H_{ZZ}$. We then modify the QAOA ansatz as follows:
\begin{align}
    \label{eq:ansatz-block}
    \ket{\psi_M( \vec{\gamma}, \vec{\beta})} &= e^{-i\beta_{M} H_0}e^{-i\gamma_{M}H} \\ \nonumber
     & \qquad \times \left(\prod_{k=1}^{M-1}e^{-i\beta_{M-k} H_0}e^{-i\gamma_{M-k} H_B}\right)\ket{\psi_0}.
\end{align}
In the above equation the first $M-1$ unitary applications involve $H_B$, and only one application of the full Hamiltonian is applied on the final step. The aim is to encode the nearest-neighbor interactions using the blocked Hamiltonian, while the last step should adjust for nonlocal effects. This approach can be implemented only when $H_{ZZ}$, which comprises nonlocal terms, is not the dominant term of the Hamiltonian. Thus, we are restricted to cases where $J$ is not large. The results for the blocked approach in the $N=4,8$ case are presented in table~\ref{tab:qaoa-block}.

\begin{table}[!hbtp]\footnotesize
    \centering
    \setlength\tabcolsep{5pt} % Spread columns a little more
    \begin{tabular}{ccccccc}
        \makecell{\\Method} & \makecell{\\$N$} & \makecell{\\$(\theta,m)$}   & \makecell{\\$M$} & \makecell{CNOT/\\qubit}  & \makecell{Rel.\\Err. $E_0$} & \makecell{GS\\Overlap} \\\hline 
        QAOA   & $4$ & $(0,0)$        & $2$ & $27$              & $0.0026$        & $0.9853$   \\
        QAOA   & $4$ & $(0,0)$        & $3$ & $39$              & $0.0022$        & $0.9922$   \\
        QAOA   & $4$ & $(\pi/4,0.1)$  & $2$ & $30$              & $0.0019$        & $0.9887$   \\
        QAOA   & $4$ & $(\pi/4,0.1)$  & $3$ & $45$              & $0.0015$        & $0.9941$   \\\hline
        QAOA   & $8$ & $(0,0)$        & $2$ & $38.5$            & $0.0028$        & $0.9553$   \\
        QAOA   & $8$ & $(0,0)$        & $3$ & $52.5$            & $0.0024$        & $0.9632$   \\
        QAOA   & $8$ & $(\pi/4,0.1)$  & $2$ & $38.5$            & $0.0023$        & $0.9763$   \\
        QAOA   & $8$ & $(\pi/4,0.1)$  & $3$ & $52.5$            & $0.0034$        & $0.9711$   \\
    \end{tabular}
    \caption{\footnotesize Blocked QAOA results for $N=4$. Here $N$, $\theta$ and $m$ are the parameters of the simulation, $M$ is the number of QAOA steps. The remaining columns give the CNOT gates per qubit, the relative error of the ground state energy measurement, and the ground state overlap.}~\label{tab:qaoa-block}
\end{table}

The decrease in the number of CNOT gates is more noticeable for larger values of $N$. An exact expression for this type of blocking would be: $8M(N-1) + (N-1)(N-2)$. One can note that there is no significant difference between the results with the full and blocked QAOA ansatz. This implies that the non-local part of the Hamiltonian can be encoded efficiently just in the last step.
Consequently, a possible advantage of the blocking procedure is the potential to 
scale the system size while maintaining the optimal parameters $\vec{\gamma}$ or 
$\vec{\beta}$ fixed as $N$ changes. 
This approach is not perfect as the non-local part of the Hamiltonian 
changes with increasing system size. 
However, it produces good ans\"atze for 
larger systems without requiring costly optimization procedures. 
Table~\ref{table:qaoa_scaling} illustrates the results for different 
system sizes using the same QAOA parameters obtained through simulated 
annealing on the $N=4$ system for the $(\theta,m)=(0,0)$ case with three steps, 
two of which are blocked while the last one is executed with the full Hamiltonian.

\begin{table}[!htbp]\footnotesize
    \centering
    \setlength\tabcolsep{5pt} % Spread columns a little more
    \begin{tabular}{ccccc}
        \makecell{\\$N $} & \makecell{CNOT/\\qubit} & \makecell{Rel.\\Err. $E_0$} & \makecell{GS\\Overlap} \\\hline
        $4*$ & $39$            & $0.0003$        & $0.9995$    \\
        $6 $ & $46.6$          & $0.0022$        & $0.9839$    \\
        $8 $ & $52.5$          & $0.0021$        & $0.9722$    \\
        $10$ & $57.6$          & $0.0022$        & $0.9599$    \\
        $12$ & $62.3$          & $0.0020$        & $0.9479$    \\
    \end{tabular}
    \caption{\footnotesize QAOA blocked with $M=3$ steps. The results for the $N=4$ are obtained after parameter optimization; the results for $N \in \{6,8,10,12\}$ have been computed using the same optimal parameters for $N=4$. }~\label{table:qaoa_scaling}
\end{table}

The data presented indicates that blocked QAOA can serve as a very cheap starting point for subsequent optimizations or alternative state-preparation algorithms, particularly when dealing with large values of $N$.
