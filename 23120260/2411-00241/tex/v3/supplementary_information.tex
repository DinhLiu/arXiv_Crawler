\documentclass[letterpaper, 10 pt, conference]{ieeeconf} 
\IEEEoverridecommandlockouts
\overrideIEEEmargins                                      
%%%%%%%%%%%%%%%%%%%%%%%%%%%%%%%%% Packages %%%%%%%%%%%%%%%%%%%%%%%%%%%%%%%%%%%%%%%%%%%%

\usepackage[utf8]{inputenc}
\usepackage[T1]{fontenc}
\usepackage{comment}
\usepackage{graphicx} % for picture formats people actually use
\usepackage{dblfloatfix} 
\graphicspath{ {Figures/} }
\usepackage{amsmath, amsfonts, amssymb, bm}
\usepackage[usenames,dvipsnames]{xcolor}
\usepackage{array}
\usepackage{setspace}
\newcolumntype{L}{>{\centering\arraybackslash}m{0.45in}}
\newcolumntype{W}{>{\centering\arraybackslash}m{0.8in}}
\usepackage{threeparttable}
\usepackage{bm}

%% These were originally in Gina's template but I'm not sure why
%\addtolength{\topmargin}{+0.2cm}
%\addtolength{\textheight}{-.2cm}


\usepackage{float}
\usepackage{caption}
\usepackage{subcaption}
\usepackage{accents}
\usepackage{hyperref}
\usepackage{cite}

\usepackage[labelfont=bf, font=small]{caption}

\newtheorem{theorem}{Theorem}

\let\labelindent\relax
\usepackage{enumitem}

% Geometric Mechanics Circ notation
\DeclareSymbolFont{symbolsC}{U}{txsyc}{m}{n}
\DeclareMathSymbol{\circleleft}{\mathrel}{symbolsC}{146}
\DeclareMathSymbol{\circleright}{\mathrel}{symbolsC}{145}
\DeclareRobustCommand{\groupderiv}[1]{\accentset{\scriptstyle\circ}{#1}}
\DeclareRobustCommand{\rightgroupderiv}[1]{\accentset{\scriptscriptstyle{\circleright}}{#1}}

\DeclareRobustCommand{\rightundercirc}[1]{\underaccent{\scriptscriptstyle{\circleright}}{#1}}

% Matrix notation shorthands with formatting:
\DeclareRobustCommand{\inv}[1]{{#1}^{\text{-}1}}
\newcommand{\tpose}[1]{{#1}^T}
\newcommand{\vect}[1]{\bm{#1}}
\newcommand{\Adjoint}[1]{\textrm{Ad}_{#1}}
\newcommand{\Adjointinv}[1]{\textrm{Ad}_{#1}^{\textrm{-}1}}
\newcommand{\Adjointinvtpose}[1]{\textrm{Ad}_{#1}^{\textrm{-}T}}

\newcommand{\billcomment}[1]{\textcolor{red}{#1}}
\newcommand{\todo}[1]{\textcolor{OliveGreen}{#1}}

\newcommand{\conv}{\textbf{conv}}

% ============================= END SETUP ===================================



% \title{\LARGE \bf Fast Evaluation of Continuum Arms Against Target Arm Geometry and Loads}
\title{\LARGE \bf Supplementary Information For "Fast Evaluation of Continuum Arms Against Target Arm Geometry and Loads"}
% \title{\textcolor{blue}{Interpretable and Fast Evaluation of Continuum Arm Task-Competence}

% \textcolor{blue}{Interpretable and Fast Comparison of Task-Competence Between Antagonistic and Regular Continuum Arms}
% } % This is just worse right, because "meausring and comparing" is just "evaluating" but wordier?
% What about this one?
%\title{\LARGE \bf Interpretable and Fast Evaluation of Continuum Arm Task-Competence}
%\title{\LARGE \bf Model-based Evaluation and Comparison of Continuum Arm Task-competence}

% \author{Bill Fan$^{1}$, Jacob Roulier$^{2}$, and Gina Olson$^{2}$% 
% \thanks{$^{1}$ Olin College of Engineering, Needham, MA, USA.}
% \thanks{$^{2}$Department of Mechanical and Industrial Engineering, University of Massachusetts, Amherst, MA, USA.}%
% }

\begin{document}

\maketitle
\thispagestyle{empty}
\pagestyle{empty}

\section{IN-COORDINATE MODEL FOR PLANAR ARM}
Given the following 2D rigid body transformation from $g_a$ to $g_b$ as $g_{ab} \in SE(2)$:

\begin{equation}
    g_{ab} =
    \begin{bmatrix}
       \mathbf{R}_{ab} & \mathbf{t}_{ab} \\
       \mathbf{0} & 1
    \end{bmatrix}
\end{equation}

its adjoint action matrix is:
\begin{equation}
    \textrm{Ad}_{ab} =
    \begin{bmatrix}
        \mathbf{R} & -[1]_{\times} \mathbf{t} \\
        \mathbf{0} & 1
    \end{bmatrix}
\end{equation}

where

\begin{equation}
    [1]_\times = 
    \begin{bmatrix}
        0 & -1 \\
        1 & 0
    \end{bmatrix}
\end{equation}

The left lifted action, which is equivalent to the Jacobian of the left-action $\Delta g \circ g$ evaluated at $g$:

\begin{equation}
    T_eL_{g_{ab}} = 
    \begin{bmatrix}
        \mathbf{R}_{ab} & 0 \\
        0 & 0
    \end{bmatrix}
\end{equation}

For a twist vector $\rightgroupderiv{g} = \begin{bmatrix} \lambda & \gamma & \kappa \end{bmatrix}^T$, the analytic form of the exponential map in $SE(2)$ is: 

\begin{equation}
    \exp_M(\rightgroupderiv{g}) = 
    \begin{bmatrix}
        \exp_M(\kappa) & \mathbf{V}(\kappa) \mathbf{l} \\
        \mathbf{0} & 1
    \end{bmatrix}
\end{equation}

where $\exp_M(\kappa)$ is the 2D rotation matrix:

\begin{equation}
    \exp_M(\kappa) =
    \begin{bmatrix}
        \cos(\kappa) & -\sin(\kappa) \\
        \sin(\kappa) & \cos(\kappa)
    \end{bmatrix}
\end{equation}

and the matrix $V(\kappa)$ is:

\begin{equation}
    \mathbf{V}(\kappa) = \frac{\sin(\kappa)}{\kappa}\mathbf{I} + \frac{1 - \cos \kappa}{\kappa}[1]_\times
\end{equation}

and $\mathbf{l}$ is the vector of linear velocities $\mathbf{l} = \begin{bmatrix} \lambda & \gamma \end{bmatrix}^T$

\section{PROOF OF ATTAINABILITY LEMMAS}
We will now prove the following two properties used previously in our simplification of the attainability problem

\textbf{Lemma 1:}
\begin{enumerate}
    \item{Each $\mathbf{a}_i[\mathcal{P}]$ is convex.}
    \item{The boundary of each $\mathbf{a}_i[\mathcal{P}]$ is the image of the pressure space boundary $\partial \mathcal{P}$, i.e., $\partial \mathbf{a}_i[\mathcal{P}] \subset \mathbf{a}_i[\partial \mathcal{P}]$ }
\end{enumerate}

First we will lay out some more notation. Assuming ${}_o\rightgroupderiv{g}$ is fixed from the given task, we will omit the shape terms $\epsilon$ and $\kappa$ and simply refer to the generic reaction wrench equation:

\begin{multline}
    \mathbf{a}(\mathbf{p}) = \begin{bmatrix}
        1 & 1 & \dots & 1 \\
        0 & 0 & \dots & 0 \\
        r_1 & r_2 & \dots & r_m
    \end{bmatrix}
    \begin{bmatrix}
        f_A(p_A) \\
        f_B(p_B) \\
        \vdots \\
        f_\mu(p_\mu)
    \end{bmatrix}
    +\\
    \begin{bmatrix}
        0 & 0 & \dots & 0 \\
        0 & 0 & \dots & 0 \\
        1 & 1 & \dots & 1
    \end{bmatrix}
    \begin{bmatrix}
        \tau_A(p_A) \\
        \tau_B(p_B) \\
        \vdots \\
        \tau_\mu(p_\mu)
    \end{bmatrix}
    +c
\end{multline}

where $c$ is a constant from the shear term $1\textrm{e}5\begin{bmatrix}{}_{A}\gamma_i & {}_{B}\gamma_i &  \dots & {}_{\mu}\gamma_i \end{bmatrix}^\textrm{T}$ and can practically be ignored for the rest of this proof.

\begin{figure}
    \centering
    \includegraphics[width=\linewidth]{figures/SI/boundary_mapping_explanation.png}
    \caption{Visualization of the preservation of boundaries within the internal reaction function. The mapping $\mathbf{f}$ is a homeomorphism and thus preserves the box structure of $\mathcal{P}$, while the rank-deficient linear map $\mathbf{a}$ "squashes" the cube but does not turn it "inside out".}
    \label{fig:enter-label}
\end{figure}

Let us denote: 
\begin{equation}
    \mathbf{x}(\mathbf{p}) = \begin{bmatrix}
        1 & 1 & \dots & 1 \\
        0 & 0 & \dots & 0 \\
        r_1 & r_2 & \dots & r_m
    \end{bmatrix}
    \begin{bmatrix}
        f_A(p_A) \\
        f_B(p_B) \\
        \vdots \\
        f_\mu(p_\mu)
    \end{bmatrix}
    = \mathbf{U} \mathbf{f}(\mathbf{p})
\end{equation}

and 
\begin{equation}
    \mathbf{y}(\mathbf{p}) =
    \begin{bmatrix}
        0 & 0 & \dots & 0 \\
        0 & 0 & \dots & 0 \\
        1 & 1 & \dots & 1
    \end{bmatrix}
    \begin{bmatrix}
        \tau_A(p_A) \\
        \tau_B(p_B) \\
        \vdots \\
        \tau_\mu(p_\mu)
    \end{bmatrix}
     = \mathbf{V} \bm{\tau}(\mathbf{p})
\end{equation}

Note that although $\mathbf{x}[\mathcal{P}]$ and $\mathbf{y}[\mathcal{P}]$ are both subsets of $\mathbb{R}^3$, since their second coordinate is always zero (that is left to the constant shear reaction term) we will investigate them as if they are elements of $\mathbb{R}^2$. Thus, when we refer to their interiors or boundaries we will actually mean their \textit{relative interiors} and \textit{relative boundaries}. For an overview of the distinction see \cite{boyd_convex_2004}.

\textbf{Claim 1:} $\mathbf{f}[\mathcal{P}]$ and $\bm{\tau}[\mathcal{P}]$ are convex, and their boundaries are the image of $\partial \mathcal{P}$, ie. $\partial \mathbf{f}[\mathcal{P}] = \mathbf{f}[\partial \mathcal{P}]$. 

\textbf{Proof of claim 1:} We will prove these properties for $\mathbf{f}[\mathcal{P}]$, and a proof for $\bm{\tau}$ is analogous. Since each $f_\alpha(p): \mathbb{R} \rightarrow \mathbb{R}$ are continuous and monotonic over a compact subset of $\mathbb{R}$, they are homeomorphisms. Since $\mathbf{f}(\mathbf{p})$ is just a concatenation of these individual functions, it is not only also a homeomorphism, but it stretches each dimension completely independently of each other. The fact that $\mathbf{f}$ is a homeomorphism implies it preserves the connectedness of $\mathcal{P}$, and sends boundaries to boundaries, as both properties are topological invariants \cite{munkres_topology_2000}. Moreover, because $\mathbf{f}$ only affects each dimension independently, and because $\mathcal{P}$ is an M-dimensional box, therefore $\mathbf{f}[\mathcal{P}]$ is also an M-dimensional box, and therefore also convex. $\hfill \blacksquare$



\textbf{Claim 2:} $\mathbf{x}[\mathcal{P}]$ is convex and $\partial \mathbf{x}[\mathcal{P}] \subset \mathbf{x}[\partial \mathcal{P}]$.

\textbf{Proof of claim 2:} Convexity follows by nature of $\mathbf{x}[\mathcal{P}]$ being a linear transformation of $\mathbf{f}[\mathcal{P}]$. The boundary property is proven for any $C^1$ continuous function with non-vanishing Jacobian in Buck "Advanced Calculus" as a corollary of Theorem 4 in Section 8 \cite{buck_advanced_2003}.  $\hfill \blacksquare$

\textbf{Claim 3:} $\mathbf{y}[\mathcal{P}]$ is convex and $\partial \mathbf{y}[\mathcal{P}] \subset \mathbf{y}[\partial \mathcal{P}]$.

\textbf{Proof of claim 3}: Same as the proof for $\mathbf{x}[\mathcal{P}]$. $\hfill \blacksquare$

We will now prove Lemma 1.

\textbf{Proof of Lemma 1}: Since $\mathbf{a}(\mathbf{p}) = \mathbf{x}(\mathbf{p}) + \mathbf{y}(\mathbf{p}) + c$ is again $C^1$ continuous with nonvanishing Jacobian, by the same result from Buck therefore $\partial \mathbf{a}[\mathcal{P}] \subset \mathbf{a}[\partial \mathcal{P}]$. Further, since $\mathbf{a}(\mathbf{p})$ is a positive weighted sum of convex functions, which is a convexity preserving operation, $\mathbf{a}[\mathcal{P}]$ is also convex. Since any convex set is path-connected by definition, $\mathbf{a}[\mathcal{P}]$ is also path-connected. $\hfill \blacksquare$


\section{UNATTAINABILITY METRICS FORMULATION}
In the event that a requirement wrench sequence $\mathbf{w}^*$ is not attainable - ie. not every $\mathbf{w}^*_i$ is enclosed by its respective $\mathbf{a}_i[\mathbf{E}]$, and not every $\Delta \mathbf{w}^*_i$ is enclosed by its respective $\Delta \mathbf{a}_i[\mathbf{E}]$ - it is useful to measure the degree to which it is unattainable.

We define the \textbf{absolute unattainability} as the sum over $i$ of the minimum distance from each $\mathbf{w}^*_i$ to $\mathbf{a}_i[\mathbf{E}]$. Since $\mathbf{a}_i[\mathbf{E}]$ is convex, this can be formulated as a convex quadratic program. For each $\mathbf{a}_i[\mathbf{E}]$, collect its vertices into a column-matrix $\mathbf{H}$. Thus we can formulate the following problem:

\begin{equation}
\begin{aligned}
    u_{a_i} = \quad \min_{\mathbf{x}} \quad & \lVert \mathbf{x} - \mathbf{w}^*_i \rVert \\
    \textrm{s.t.} \quad & \mathbf{x} = \mathbf{H} \mathbf{s}\\
    & \mathbf{1}^\textrm{T} \mathbf{s} \leq 1
\end{aligned}
\end{equation}

The vector $\mathbf{s}$ is a vector of weights for ensuring $\mathbf{x}$ is a convex sum of the convex hull vertices, thus ensuring $\mathbf{x}$ is an element of the convex set $\mathbf{a}_i[\mathbf{E}]$. In practice it is easier to perform a substitution and directly optimize over $\mathbf{w}$:

\begin{equation}
\begin{aligned}
    u_{a_i} = \quad \min_{\mathbf{s}} \quad & \lVert \mathbf{H} \mathbf{s} - \mathbf{w}^*_i \rVert \\
    \textrm{s.t.} \quad & \mathbf{1}^\textrm{T} \mathbf{s} \leq 1
\end{aligned}
\end{equation}

The total absolute unattainability of a task is therefore the sum $\sum_{k=1}^N u_{a_i}$. To measure the \textbf{relative unattainability}, the same problem can be solved but summing the minimum distances between each $\Delta \mathbf{w}_i^*$ and their corresponding $\Delta \mathbf{a}_i[\mathbf{E}]$. These problems are all implemented using Matlab's fmincon solver. 

\section{BELLOWS ACTUATOR CHARACTERIZATION}
\begin{figure}
    \centering
    \includegraphics[width=\linewidth]{figures/SI/dataset2.pdf}
    \caption{Example actuator characterization dataset}
    \label{fig:enter-label}
\end{figure}

\begin{figure}
    \centering
    \includegraphics[width=\linewidth]{figures/SI/fit_accuracy.pdf}
    \caption{Comparison of experiment data (circles) with fitted actuator model (solid). Vertical line is the maximum allowed pressure in simulation.}
\end{figure}
Bellow actuators were characterized on an Instron in the setup shown in Fig. \ref{fig:model_explanation}C, where four actuators were connected together to reduce extraneous effects such as buckling or shearing. Once the actuators were mounted, an automated Instron routine deformed the actuators to different levels of strain and held the strain constant for one minute at a time. During this time the bellows were manually actuated across a range of pressures using a syringe pump, and the corresponding forces were measured with a pressure sensor connected to an Arduino and logged using PySerial. 

Afterwards, the time of the pressure data was aligned to the Instron strain and force first by manually shifting the pressures. Then, the pressure data was smoothed with a Gaussian window, and then linearly interpolated at the Instron's timestamps. This enabled us to experimentally estimate the mapping of $\hat{f}(\epsilon, p)$, which we fitted with a poly33 model from Matlab's curve fitting toolbox.


\bibliographystyle{ieeetr}
\bibliography{zotero,GinaRef_14Feb2019}

\end{document}
