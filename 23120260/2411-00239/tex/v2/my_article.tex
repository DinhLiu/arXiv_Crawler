\documentclass[lettersize,journal]{IEEEtran}
\usepackage{amsmath,amsfonts}
\usepackage{algorithmic}
\usepackage{algorithm}
\usepackage{array}
\usepackage{bm}
\usepackage[caption=false,font=normalsize,labelfont=sf,textfont=sf]{subfig}
\usepackage{textcomp}
\usepackage{stfloats}
\usepackage{url}
\usepackage{verbatim}
\usepackage{graphicx}
\usepackage{cite}



\usepackage{color}
% 以上 1行命令实现字体加颜色的宏包  https://zhuanlan.zhihu.com/p/138818311 

\usepackage{multirow}  % 表格合并多行
\usepackage{multicol}  % 表格合并多列
\usepackage{booktabs}  % 表格粗线条 \toprule、\midrule、\bottomrule给表格画横线
\usepackage{makecell}  % 表格内部换行
\usepackage[table,xcdraw]{xcolor}
% 以上 4行命令用于实现更加高质量的表格

% \usepackage{caption}
% \usepackage{subcaption} % 子图格式
% \captionsetup[subfigure]{font=normalfont}
% \usepackage{caption}


% https://blog.wordvice.com/recommended-verbs-for-research-writing/
% https://www.editage.com/all-about-publication/research/impressive-verbs-to-use-in-your-research-paper
% https://paperpal.com/blog/academic-writing-guides/language-grammar/verbs-academic-writing


% \usepackage{hyperref}  
% 以上 \usepackage{hyperref} 命令可以引入参考文献的字体框

\usepackage[colorlinks=true,linkcolor=blue,citecolor=blue,urlcolor=blue,]{hyperref} 
\usepackage[doipre={doi:~}]{uri}
% 以上 2行命令实现所有参考文献引用变成蓝色, 这里的 linkcolor之前是black

% 这行代码是用于LaTeX文档的,它指定了单词的连字符(hyphenation)规则。
\hyphenation{op-tical net-works semi-conduc-tor IEEE-Xplore}
% updated with editorial comments 8/9/2021s
\begin{document}

% 模板参考介绍:https://blog.csdn.net/weixin_41198651/article/details/123428311 
\title{Aquatic-GS: A Hybrid 3D Representation for Underwater Scenes}

%Jiajun: Semantic-guided? Semantically 是副词, 什么是blind-spot framework? 感觉缺个动词,blind-spot是名词,framework也是名词,写在一起有点奇怪

% \author{Shaohua Liu, Junzhe Lu, Hongkun Dou, Jiajun Li~\IEEEmembership{Member,~IEEE} and Yue Deng,~\IEEEmembership{Senoir Member,~IEEE}     % <-this % stops a space
% \thanks{S. Liu, J. Lu, H. Dou, J. Li and Y. Deng are with the School of Astronautics, Beihang University, Beijing 100191, China.}% <-this % stops a space
% \thanks{S. Liu is also with Shenyuan Honors College, Beihang University, Beijing 100191, China.}%


% *-* 【记得选择这个】

\author{Shaohua Liu, Junzhe Lu, Zuoya Gu, Jiajun Li~\IEEEmembership{Member,~IEEE} and Yue Deng,~\IEEEmembership{Senoir Member,~IEEE}     % <-this % stops a space

\thanks{Shaohua Liu is with the School of Astronautics and the Shenyuan Honors College, Beihang University, Beijing 100191, China (e-mail: liushaohua@buaa.edu.cn).}

\thanks{Junzhe Lu, Zuoya Gu and Jiajun Li are with the School of Astronautics, Beihang University, Beijing 100191, China (e-mail: junzhelu.bh@gmail.com; zuoyagu@buaa.edu.cn; jiajunli@buaa.edu.cn).}

\thanks{Yue Deng is with the School of Artificial Intelligence, Beihang University, Beijing 100191, China (e-mail: ydeng@buaa.edu.cn).}%

\thanks{Corresponding author: Yue Deng}

}

% The paper headers
\markboth{Journal of \LaTeX\ Class Files,~Vol.~14, No.~8, August~2021}%
% \markboth{IEEE TRANSACTIONS ON IMAGE PROCESSING}%
{Liu \MakeLowercase{\textit{et al.}}: Aquatic-GS: A Hybrid 3D Representation for Underwater Scenes}

\IEEEpubid{0000--0000/00\$00.00~\copyright~2021 IEEE}
% Remember, if you use this you must call \IEEEpubidadjcol in the second
% column for its text to clear the IEEEpubid mark.

\maketitle

% 家军老师建议汇总:
% 1,要注意逻辑的连贯性,不能让读者读的时候有一种一跳一跳的感觉;
% 2,要注意不能服务上下文的话,就不要说,不要总是抛出新的概念而不去讲或者不去用,徒增文章复杂性;
% 3,Related works要和本文的工作努力的结合起来,不要抛出个概念就不管了,最好在对应章节结束的时候,回扣到文章内容上;
% 4,最后的Conclusion部分,不能和摘要一样。比如说摘要是说 为了xx 我们做了xxx,那么结论应该是,我们做了xx 实现了xx效果,最后还可以做一些未来引申或者展望


% Jiajun: abstract太长,太细了,看不懂。Abstract是给大同行看的,尽量用缜密的逻辑把文章故事讲清楚,涉及的RFA和GSE具体细节是为逻辑服务的,不要尝试在abstract里就想让读者明白你做了什么
\begin{abstract}
% 【立足点,我们做的不仅是水下含水场景的场景,还做的是无水场景的重建,因为真正无人机导航需要的,就是原本的无水结构信息】
% 【注意:一定要反复强调,把水下3D表征和 复原+结构+水体 画等号!】
% 【水下处理这块,我们主要解决的问题,是水下NeRF缓慢的问题,引入一个更快的表达方式,同时更有效的水体建模【非均匀】,来复合正式场景】
% 【*-*】注意一定要把水下NeRF的方法的特点说出来:体渲染框架中的集成
% 【*-*】思考一下 相关工作这块,或者说对比的seathru方法这块,是不是应该从大道理上讲的细节(明确)一点,不然总给人一种模糊的感觉,会不会让审稿人把握不好我们方法的路子
% 【*-* 讲了渲染水下图像后,是不是?】这里应该加一点,就是有了 渲染了需相应的水下场景拿来干什么? 或者说,通过UIF模型结合,可以实现2个事情,第1是渲染水下场景,第二是用于后面的优化【感觉不用说了,因为我们前面一直在讲水下场景的3D表征,现在讲到渲染水下场景,就可以了】【先不加了】

% 3D representation of underwater scenes is a valuable yet challenging task due to the distance-dependent and wavelength-selective attenuation and scattering effects. Existing underwater 3D representation methods struggle to effectively represent the objects and the water medium simultaneously. 

Representing underwater 3D scenes is a valuable yet complex task, as attenuation and scattering effects during underwater imaging significantly couple the information of the objects and the water. This coupling presents a significant challenge for existing methods in effectively representing both the objects and the water medium simultaneously. To address this challenge, we propose Aquatic-GS, a hybrid 3D representation approach for underwater scenes that effectively represents both the objects and the water medium. Specifically, we construct a Neural Water Field (NWF) to implicitly model the water parameters, while extending the latest 3D Gaussian Splatting (3DGS) to model the objects explicitly. Both components are integrated through a physics-based underwater image formation model to represent complex underwater scenes. Moreover, to construct more precise scene geometry and details, we design a Depth-Guided Optimization (DGO) mechanism that uses a pseudo-depth map as auxiliary guidance. After optimization, Aquatic-GS enables the rendering of novel underwater viewpoints and supports restoring the true appearance of underwater scenes, as if the water medium were absent. Extensive experiments on both simulated and real-world datasets demonstrate that Aquatic-GS surpasses state-of-the-art underwater 3D representation methods, achieving better rendering quality and real-time rendering performance with a 410$\times$ increase in speed. Furthermore, regarding underwater image restoration, Aquatic-GS outperforms representative dewatering methods in color correction, detail recovery, and stability. Our models, code, and datasets can be accessed at \href{https://aquaticgs.github.io}{https://aquaticgs.github.io}.


\end{abstract}
\begin{IEEEkeywords}
Underwater 3D Representation, 3D Gaussian Splatting, Implicit Modeling, Novel View Synthesis, and Underwater Image Restoration.
\end{IEEEkeywords}

% ---------------------------------------------------------------------------
\section{Introduction}


% 家军老师建议汇总:
% 1,要注意逻辑的连贯性,不能让读者读的时候有一种一跳一跳的感觉;
% 2,要注意不能服务上下文的话,就不要说,不要总是抛出新的概念而不去讲或者不去用,徒增文章复杂性;
% 3,Related works要和本文的工作努力的结合起来,不要抛出个概念就不管了,最好在对应章节结束的时候,回扣到文章内容上;
% 4,最后的Conclusion部分,不能和摘要一样。比如说摘要是说 为了xx 我们做了xxx,那么结论应该是,我们做了xx 实现了xx效果,最后还可以做一些未来引申或者展望
% Jiajun: Introduce sonar image and the importance of sonar image despeckling
% Jiajun: writing must be easy-to-read


\begin{figure}[!t]
\centering
\includegraphics[width=1.0\columnwidth]{images/fig1.pdf}
\caption{(a) Underwater imaging environment and the hybrid representation strategy employed by Aquatic-GS. (b) Scene information learned by Aquatic-GS, including water parameters, the true appearance, and the geometry of the underwater scene. 'Atten.' and 'Coeffs' are abbreviations for Attenuation and Coefficients, respectively. (c) Rendering of an underwater image by Aquatic-GS using a physics-based underwater imaging model.}
\label{fig_imageformation}
\end{figure}
% 【这里要强调一下水体参数分布的空间非均匀性吗】—【不强调,只展示】

\IEEEPARstart{T}{he} 3D representation of underwater scenes plays a vital role in a wide range of applications, including autonomous underwater vehicles (AUVs)~\cite{wang2023real}, marine ecosystem studies~\cite{yuval2024releasing}, and underwater virtual reality systems~\cite{johnson2017high, llorach2023experience}. An effective underwater 3D representation should encompass both objects (i.e., the water-free scene) and the water medium. The objects provide critical information about the scene's appearance and geometry, while the water medium conveys essential water parameters. However, achieving this goal remains difficult for vision-based 3D representation methods due to the impact of distance-dependent and wavelength-selective attenuation and scattering inherent in the underwater imaging process~\cite{boittiaux2024sucre}. As illustrated in Fig.~\ref{fig_imageformation}(a), attenuation occurs when light reflected from objects is absorbed by the water as it travels toward the camera, with red light being attenuated more than blue and green, leading to a color cast in captured images. Scattering, particularly backscattering, happens when underwater ambient light is scattered toward the camera by particles suspended in the water, leading to a hazy appearance and low contrast in underwater images. Moreover, these two water effects become more pronounced as the distance between the object and the camera increases, leading to a significant coupling of the information from both the objects and the water medium. Such coupling complicates the modeling of both the objects and the water medium, hindering the capture of the true appearance, geometry, and water parameters in underwater scenes~\cite{boittiaux2024sucre}.

\IEEEpubidadjcol  % used this to clear the IEEEpubid mark

Recently, advanced 3D scene representation methods, such as Neural Radiance Field (NeRF)~\cite{mildenhall2021nerf} and 3D Gaussian Splatting (3DGS)~\cite{kerbl20233d}, have received significant attention. NeRF employs an implicit modeling strategy, typically leveraging neural networks to efficiently encode complex scenes, and utilizes volume rendering to produce photo-realistic images. In contrast, 3DGS explicitly models the scene with a set of learnable 3D Gaussian primitives and employs a tile-based rasterization pipeline, resulting in superior rendering quality, reduced training cost, and faster rendering speed compared to NeRF. However, in the context of underwater scene representation, both NeRF and 3DGS struggle to effectively represent both the objects and the water medium simultaneously. For instance, some NeRF-based studies~\cite{zhang2023beyond, levy2023seathru, tang2024neural} extend their volumetric rendering framework to accommodate the water medium; however, their implicit modeling strategies for objects often lead to blurred details, noisy geometry, and significant rendering costs. Conversely, while the explicit modeling strategy employed by 3DGS adeptly represents opaque objects, it struggles with the abundant translucent water medium~\cite{fei20243d}, leading to inevitable artifacts and inaccurate geometry when applied to underwater scenes. 
% Furthermore, due to fundamental differences in principles, the volumetric rendering framework in~\cite{zhang2023beyond, levy2023seathru, tang2024neural} is incompatible with the rasterization mechanism of 3DGS, hindering the integration of these methods in underwater scenes~\cite{lee2024deblurring}. 
Thus, in underwater scene representation, how to effectively represent both objects and the water medium simultaneously remains a significant challenge.

% Worse still,

% 注意 这里讲不透明的物体表面,要思考一下是否要说的这么明确,应该加上文献的
% 如果不讲 3DGS主要是建模不透明物体的表面的话,也可以说,3DGS在半透明物体的表征方面能力较差。
% 考虑到这些属性实际上受太阳角度、观察视角等因素影响并非是全场景均匀的,我们将它们建模为逐像素视线方向的函数来更好表征水体非均匀特性。
% 【注意】水体属性空间非均匀的考虑这块,应该加上文献【加文献】 —— 【*-* 别直接在这里提了,直接就是会受多种影响】
% Specifically, considering the spatial non-uniformity of water properties in real environments~\cite{bekerman2020unveiling, akkaynak2018revised, nakath2021situ}, we designed a Neural Water Field (NWF) to implicitly model the distribution of water parameters.
% Given the multitude of complex factors influencing water properties in real environments, we designed a Neural Water Field (NWF) to implicitly model the distribution of water parameters.

% To address this challenge, we propose a hybrid 3D representation approach for underwater scenes, called Aquatic-GS, which implicitly models water properties and explicitly represents underwater objects (i.e., water-free scenes), integrating both through a physics-based underwater image formation (UIF) model to effectively represent underwater scenes (see Fig.~\ref{fig_imageformation}).

% , which implicitly models water properties and explicitly represents underwater objects (i.e., water-free scenes), integrating both through a physics-based underwater image formation (UIF) model to effectively represent underwater scenes (see Fig.~\ref{fig_imageformation}).
% 
% In this work, we propose a novel 3D representation approach for underwater scenes, called Aquatic-GS. As shown in Fig.~\ref{fig_imageformation}, Aquatic-GS employs a hybrid representation manner to decouple the water-free scene from the water effects in underwater images.
% 

To address this challenge, we propose a hybrid 3D representation approach called Aquatic-GS, which combines the advantages of explicit representation and implicit modeling to effectively capture the characteristics of both the objects and the water medium in underwater scenes, as shown in Fig.~\ref{fig_imageformation}.
Specifically, considering the spatial non-uniformity of water properties in real environments~\cite{bekerman2020unveiling, akkaynak2018revised, nakath2021situ}, we designed a Neural Water Field (NWF) to model the distributions of the water parameters implicitly. Simultaneously, motivated by the efficiency of 3DGS in representing objects, we utilize 3D Gaussians to explicitly capture the true appearance and geometry of the scene. Both components are integrated through a physics-based underwater image formation (UIF) model to represent the underwater scene. Moreover, to tackle the issues caused by water effects in accurately representing the scene, we introduce a Depth-Guided Optimization (DGO) mechanism. The DGO mechanism employs pseudo-depth maps generated by the latest monocular depth estimators like DepthAnything~\cite{yang2024depth} to guide Aquatic-GS in achieving a more precise representation of the scene’s geometry and distant details. Once optimized, Aquatic-GS learns the true appearance, geometry, and distributions of water parameters in underwater scenes. This not only enables real-time rendering of novel underwater viewpoints but also supports restoring the underwater scene as if the water medium were absent. 

We evaluated Aquatic-GS's performance in underwater novel view synthesis (NVS) and underwater image restoration (UIR) on three real-world underwater datasets and our simulated dataset. For the NVS task, Aquatic-GS outperforms state-of-the-art (SOTA) NeRF-based underwater representation methods, achieving superior rendering quality, reduced training time, and a 410$\times$ increase in rendering speed for real-time performance. For the UIR task, Aquatic-GS also surpasses representative dewatering methods in color correction, detail recovery, and stability.

% 注意,这里因为已经前面提到了 underwater objects,反映的是谁在场景的真实外观和几何,现在这里又提了一下 water-free scene。 那么具体如何复原,就在methods讲就可以了。

% scene’s appearance and geometry

% To address this challenge, we propose a hybrid 3D representation approach called Aquatic-GS, which effectively represents underwater 3D scenes by combining the advantages of implicit modeling and explicit representation to accommodate the characteristics of both objects and the water medium.


Our contributions can be summarized as follows:
\begin{itemize}
\item{We propose Aquatic-GS, a novel 3D representation method for underwater scenes that implicitly models water parameters and explicitly models the scene’s appearance and geometry. This hybrid setup effectively represents both the water medium and the objects, ensuring a comprehensive depiction of complex underwater scenes.}
\item{We extend the latest 3DGS to underwater environments and introduce a Depth-Guided Optimization mechanism to tackle the obstacles posed by distance-dependent water effects, achieving better geometry details.}
\item{Extensive experiments on underwater novel view synthesis and image restoration tasks demonstrate the effectiveness of Aquatic-GS for underwater scene representation.}
\end{itemize}





The rest of this paper is structured as follows: Section~\ref{sec:related} covers the related work. In Section~\ref{sec:method}, we provide a detailed description of the proposed Aquatic-GS. Section~\ref{sec:experiment} presents extensive experimental results. Finally, Section~\ref{sec:conclusion} concludes the paper.

\section{Related work}\label{sec:related}


\subsection{NeRF-Based Underwater Representation Methods}

% 删除大尺度 turki2022mega
% 删除 scatternerf ramazzina2023scatternerf
% 删除 wild martin2021nerf, 
NeRF is a volumetric scene representation method that utilizes a deep neural network to encode the radiance field. Its implicit representation strategy has the advantage of efficiently compressing complex scenes. NeRF typically queries the scene's density and color at any given point through a neural network and employs volumetric rendering to produce remarkably photorealistic images. Furthermore, it has been widely applied to various types of scenes, including unbounded~\cite{barron2022mip, muller2022instant}, in-the-wild~\cite{yang2023cross}, large-scale~\cite{xu2023grid}, and scattering environments~\cite{zhang2023beyond, levy2023seathru, tang2024neural, ramazzina2023scatternerf}.

For the 3D representation of underwater scenes, NeRF-based methods typically extend their volumetric rendering framework to accommodate the water medium. NeuralSea~\cite{zhang2023beyond} incorporating different light-transmitting physics for the water medium and the objects to represent underwater 3D scenes. SeaThru-NeRF~\cite{levy2023seathru} assigns colors and densities for objects and water at each spatial point, utilizing a scattering-aware volumetric rendering framework to simulate attenuation and scattering, ultimately supporting the rendering of new viewpoints for underwater scenes and water-free scenes. Tang et al.~\cite{tang2024neural} treat water as a semi-transparent object with uniform density and color throughout the space, and jointly optimize its parameters with those of the scene objects. However, the implicit modeling of objects in existing methods often results in blurred details, noisy geometry, and considerable computational overhead, including lengthy training times and slow rendering speeds. In contrast, our Aquatic-GS introduces a hybrid modeling strategy. Unlike these methods that consider the water medium from a volumetric rendering perspective, we view the water medium from the perspective of underwater imaging models. We leverage the advantages of implicit modeling to learn the spatial distribution of water parameters, and we extend the advanced 3DGS to explicitly represent the objects, thereby enabling more accurate geometry, efficient training, and real-time rendering.


% However, the implicit modeling strategies of these methods for objects often result in blurred details and noisy geometry. These methods also face significant computational overhead, lengthy training times, and slow rendering speeds, hindering real-time rendering and interaction. In contrast, our Aquatic-GS explicitly represents the underwater objects, enabling more accurate geometry, efficient training, and real-time rendering for practical applications.

% face significant computational overhead, lengthy training times, and slow rendering speeds, hindering real-time rendering and interaction. In contrast, Aquatic-GS adapts advanced 3DGS techniques to underwater scenes, effectively representing underwater objects while enabling efficient training and real-time rendering for practical applications.

% However, these volumetric methods face significant computational overhead, lengthy training time, and slow rendering speed due to the ray-marching strategy, hindering real-time rendering and interaction. In contrast, Aquatic-GS extends advanced 3DGS techniques to underwater scenes, effectively representing complex environments while enabling efficient training and real-time rendering for practical applications.


\subsection{3D Gaussian Splatting}

3DGS~\cite{kerbl20233d} represents a significant breakthrough in 3D scene representation and has attracted considerable attention for its superior high-fidelity rendering quality and real-time rendering capabilities compared to NeRF. Subsequent research has extended 3DGS to various challenging scenarios, including unconstrained photo collections~\cite{dahmani2024swag}, sparse views~\cite{zhu2025fsgs,xiong2023sparsegs, li2024dngaussian, paliwal2024coherentgs}, blurry image collections~\cite{chen2024deblur, lee2024deblurring}, large-scale scenes~\cite{lin2024vastgaussian}, and dynamic scenes~\cite{wu20244d}. Several approaches introduce edge~\cite{gong2024eggs}, frequency~\cite{zhang2024fregs}, and effective rank~\cite{hyung2024effective} regularization to address over-reconstruction issues, while others~\cite{zhang2024pixel, ye2024absgs, fang2024mini} focus on improving over-reconstructed Gaussian identification for finer-grained representation. Methods like SparseGS~\cite{xiong2023sparsegs} and DNGaussian~\cite{li2024dngaussian} use monocular depth maps for depth supervision, mitigating overfitting in sparse-view scenarios. Other methods incorporate optical flow~\cite{paliwal2024coherentgs} and normal priors~\cite{turkulainen2024dn} to further enhance 3DGS's geometric representation. Additionally, Chen et al.\cite{chen2024deblur} and Lee et al.\cite{lee2024deblurring} model the formation of motion and defocus blur, enabling the reconstruction of sharp radiance fields from blurry image collections. Despite these advancements, 3DGS's explicit modeling strategy limits its effectiveness in underwater scene representation and restoration~\cite{li2024watersplatting, zhang2024recgs}. In this work, we extend 3DGS to underwater environments by integrating a neural water field and a physics-based underwater image formation model. Our Aquatic-GS accurately captures the scene's true appearance, geometry, and water parameters, enabling novel underwater viewpoint synthesis and supporting reliable scene restoration.


% 哪些被拿出来的额外文献 
% wild: xu2024wild
% large-scale: liu2024citygaussian
% dynamic: yang2024deformable
% segmentation: zhou2024feature , and 3D scene segmentation~\cite{ye2023gaussian}
% generation: chen2024text , and 3D content generation~\cite{zhou2025dreamscene360}
% 

% Despite these advancements, 3DGS is constrained by its focus on handling atmospheric imaging environments and its weak ability to represent semi-transparent objects, which limits its effectiveness in underwater scene representation and restoration~\cite{fei20243d, li2024watersplatting, zhang2024recgs}. In this work, we extend 3DGS to underwater environments. Our Aquatic-GS utilizes 3DGS to represent water-free scenes while integrating a neural water field and a physics-based underwater image formation model to account for water effects. This integration not only facilitates the rendering of novel underwater viewpoints but also focuses Aquatic-GS on accurately capturing the scene’s true appearance and geometry, thus supporting reliable underwater scene restoration.


% ------------------------- 【 下面的内容算是对 一些水下3DGS 应用的说明 暂不打算考虑 】 --------------
 
% Recently, some studies~\cite{qu2024z,zhang2024recgs,li2024watersplatting} have adapted 3DGS for underwater environments. Z-Splat~\cite{qu2024z} introduced a 3DGS algorithm incorporating sonar data for coarse geometric reconstruction of underwater objects, while RecGS~\cite{zhang2024recgs} employed 3DGS to remove water caustics in shallow-water seabed areas. Notably, these methods did not consider the effects of water absorption and scattering. 

% In a concurrent parallel study, Li \textit{et al.}~\cite{li2024watersplatting} modified the rasterization pipeline of 3DGS to consider medium effects between adjacent Gaussians and render underwater scenes. 

% In contrast, our approach, Aquatic-GS, utilizes 3DGS to represent water-free scenes while integrating a neural water field and a physics-based underwater image formation model to effectively simulate water effects. This integration focuses Aquatic-GS on accurately capturing the true appearance and geometry of the scene, facilitating reliable underwater scene restoration.




% 【*-* 这里后面应该考虑一下,新增关于额外的深度引导机制的优势???】

% This distinction emphasizes our unique integration of techniques to address the complexities of underwater imaging.



% 感悟:要多读论文,实在是读论文读的太少了
\subsection{Underwater Image Restoration}

The goal of underwater image restoration is to correct the color cast and low contrast caused by water, obtaining the true appearance of underwater scenes. Existing underwater image restoration methods can be broadly classified into four categories: visual prior-based, data-driven, physics-based, and NeRF-based approaches. 


% 放弃插入的文献:
% visual-based: 拉普拉斯先验 zhuang2022underwater
% Data-driven: 合成水下数据集 li2020underwater GAN的方法 yan2023hybrur
% 3D-based: skinner2017automatic bryson2013colour

% Visual prior-based methods~\cite{ancuti2012enhancing, zhang2022underwater, zhuang2022underwater, zhou2024pixel} often rely on histogram adjustment, image fusion, and Retinex theory to directly adjust pixel values and improve the quality of underwater images. However, they tend to overlook the underwater imaging mechanism, making them prone to over-enhancement and color artifacts. Data-driven approaches~\cite{badran2023daut, li2020underwater, li2019underwater, xie2024uveb, wang2023domain, yan2023hybrur,liu2022twin} leverage deep learning to remove water effects and restore underwater images. Due to the lack of water-free reference images, researchers either synthesize underwater images~\cite{badran2023daut, li2020underwater} or generate pseudo ground truth~\cite{li2019underwater, xie2024uveb} for supervised learning. Some methods also employ GANs~\cite{wang2023domain, yan2023hybrur} or contrastive learning~\cite{liu2022twin} to map underwater image distributions to in-air image distributions. However, the domain gap between synthetic and real-world data, along with variable water effects, limits the generalizability of these methods to real underwater scenes.

% Visual prior-based methods~\cite{ancuti2012enhancing, zhang2022underwater, zhou2024pixel} typically adjust pixel values to enhance underwater images but often overlook the underwater imaging mechanism, making them prone to over-enhancement and color artifacts. Data-driven approaches use supervised learning with synthetic datasets~\cite{badran2023daut, li2019underwater, xie2024uveb}, GANs~\cite{wang2023domain}, or contrastive learning~\cite{liu2022twin} to map underwater images to in-air distributions. However, the domain gap between training and testing data limits their generalizability to real scenes. 
% 
Visual prior-based methods~\cite{ancuti2012enhancing, zhang2022underwater, zhou2024pixel} typically adjust pixel values to enhance underwater images but often overlook the underwater imaging mechanism. Data-driven approaches use supervised learning with synthetic datasets~\cite{badran2023daut, li2019underwater, xie2024uveb}, GANs~\cite{wang2023domain}, or contrastive learning~\cite{liu2022twin} to map underwater images to in-air distributions.
% 
% 删去了基于 Visual-based 和 data-driven 方法的 缺点
% 
Physics-based methodss~\cite{drews2016underwater, peng2017underwater, akkaynak2019sea, berman2020underwater, fu2022unsupervised, nakath2021situ, boittiaux2024sucre} consider underwater image formation models to estimate parameters and reverse the degradation process. These methods often incorporate priors, such as underwater dark channel~\cite{drews2016underwater}, and haze-lines~\cite{berman2020underwater}, to facilitate parameter estimation, while some~\cite{nakath2021situ, boittiaux2024sucre} leverage multi-view observations to enhance parameter estimation by constructing 3D structures. For instance, SUCRe~\cite{boittiaux2024sucre} employs the 3D structure to track color changes of a point from different viewpoints, resulting in more accurate parameter estimation. However, most rely on simplified image formation models that assume water parameters are globally uniform within each channel, which may lead to instability in practical applications~\cite{bekerman2020unveiling, li2019underwater}.
% 
% Physics-based methods~\cite{akkaynak2019sea, drews2016underwater, peng2017underwater, berman2020underwater, fu2022unsupervised, bryson2013colour, skinner2017automatic, boittiaux2024sucre} take into account physical models of underwater image formation, estimate parameters of these models, and recover the true scene appearance by reversing the degradation process. These methods often introduce additional prior information to facilitate parameter estimation, including underwater dark channel prior~\cite{drews2016underwater}, histogram distribution prior~\cite{li2016underwater}, blurriness prior~\cite{peng2017underwater}, haze-lines prior~\cite{berman2020underwater}, and homology prior~\cite{fu2022unsupervised}. 
% For instance, Sea-thru~\cite{akkaynak2019sea} utilizes the dark channel prior~\cite{he2010single} and known depth information to estimate water parameters. Berman et al.~\cite{berman2020underwater} employed a haze-line prior to automatically identify the most suitable water parameters from a collection of known water types. 
% Some approaches~\cite{bryson2013colour, skinner2017automatic, boittiaux2024sucre} also utilize multi-view observations of the scene to construct a 3D structure, enhancing the estimation of imaging model parameters. 
% For instance, SUCRe~\cite{boittiaux2024sucre} employs this 3D structure to track color changes of a point from different viewpoints, resulting in more accurate parameter estimation.
% However, most physics-based methods rely on simplified image formation models that assume water parameters are globally uniform within each channel,  which may lead to instability in practical applications~\cite{bekerman2020unveiling, li2019underwater}. 
% 
% 
NeRF-based methods~\cite{sethuraman2023waternerf, levy2023seathru, zhang2023beyond, tang2024neural}, like SeaThru-NeRF~\cite{levy2023seathru}, typically extend volumetric rendering framework to accommodate the water medium and learn the water-free scenes. However, their reliance on implicit modeling results in blurred details, high training costs, and considerably slow rendering speed.
% 
In contrast, our Aquatic-GS extends the latest 3DGS to learn the water-free scene, while integrating a neural water field to learn the distribution of water parameters. This integration better characterizes complex underwater environments, decouples water effects from the true appearance of underwater scenes, and enables reliable underwater image restoration with lower training costs and real-time rendering performance.



\begin{figure*}[!t]
\centering
\includegraphics[width=7in]{images/Fig2.pdf}
\caption{Pipeline of Aquatic-GS. For a given viewpoint within a bounded viewing frustum, Aquatic-GS uses the Neural Water Field to obtain the underwater ambient light $\bm{A}$, attenuation coefficients $\bm{\beta}^D$, and backscattering coefficients $\bm{\beta}^B$ for the current viewpoint, while utilizing 3D Gaussian Splatting to render the water-free image $\bm{J}$ along with the corresponding depth $\bm{D}$ and distance maps $\bm{R}$. These outputs are then unified through a physics-based underwater image formation (UIF) model to render the corresponding underwater image $\bm{I}$. During the optimization, along with the reconstruction loss between $\bm{I}$ and $\hat{\bm{I}}$, the Depth-Guided Optimization mechanism, which includes four specifically designed loss functions, leverages a pseudo-depth map $\hat{\bm{D}}$ to guide Aquatic-GS in producing more precise scene representation.}
\label{fig_Framework}
\end{figure*}



\section{methodology}~\label{sec:method}

\subsection{Underwater Image Formation Model}~\label{sec:UIF_Model}   

% Due to the water effects, images captured underwater do not reflect the true appearance of objects, accompanied by various forms of degradation, such as color cast and low contrast. 
According to the physics-based underwater image formation (UIF) model proposed by Akkaynak et al.~\cite{akkaynak2018revised}, the relationship between the captured underwater image and the true appearance of the scene can be expressed as follows:
\begin{equation}
\label{uif}
\bm{I}= \bm{J} e^{-\bm{\beta}^D \bm{R}} + \bm{A} (1 - e^{-\bm{\beta}^B \bm{R}}),
\end{equation}
where $\bm{R} \in {\mathbb{R}}^{h \times w \times 1}$ represents the distance between the objects and the camera corresponding to each pixel in the captured image $\bm{I} \in {\mathbb{R}}^{h \times w \times 3}$, with $h$, $w$ and $3$ denoting the height, width, and channels of the image, respectively. $\bm{J} \in {\mathbb{R}}^{h \times w \times 3}$, the water-free image, represents the image of objects that the camera would capture in the absence of the water medium, reflecting the true appearance of the underwater scene. $\bm{A} \in {\mathbb{R}}^{h \times w \times 3}$ denotes the underwater ambient light, which can be considered as the backscatter water color at infinity. $\bm{\beta}^D \in {\mathbb{R}}^{h \times w \times 3}$ and $\bm{\beta}^B \in {\mathbb{R}}^{h \times w \times 3}$ represent the attenuation coefficients and backscatter coefficients of the water medium, respectively, with three channels modeling the wavelength selectivity of these effects.

Typically, the water parameters $\bm{A}$, $\bm{\beta}^D$ and $\bm{\beta}^B$ are assumed to be uniform within each channel in the scene, while being different across channels, simplifying the UIF model to nine unknowns~\cite{akkaynak2019sea, boittiaux2024sucre}. However, some studies suggest that this simplified version may not fully account for factors such as solar directionality~\cite{bekerman2020unveiling}, viewing angles~\cite{akkaynak2018revised}, and local water composition~\cite{nakath2021situ}, among others, which can lead to instabilities and visually unpleasing results~\cite{li2019underwater}. Therefore, in our work, we relax the assumption of uniformity within each channel and attempt to learn the spatially non-uniform distributions of these parameters across each channel.


\subsection{Overview of Aquatic-GS}   

% 【有界视场的一些同类做法】Given that MiDaS outputs inverse relative depth, we cap the depth values at a maximum of 20 meters. This aligns with the understanding that scene radiance in underwater environments is predominantly affected by backscatter beyond this range.

% we utilize the neural water field to output the water parameters $\bm{A}$, $\bm{\beta}^D$, and $\bm{\beta}^B$ associated with each pixel's viewing direction.

The pipeline of Aquatic-GS is illustrated in Fig.~\ref{fig_Framework}. The distance-dependent water effects cause exponential decay of object information in underwater images as distance increases. Therefore, we set a bounded viewing frustum with a maximum visible distance of $r_{max}$ for each camera to focus on the scene representation within the visible range. For a given viewpoint, we utilize the neural water field to output the water parameters $\bm{A}$, $\bm{\beta}^D$, and $\bm{\beta}^B$. Simultaneously, we employ 3DGS to render the water-free image $\bm{J}$ and corresponding geometric information of the scene, such as depth map $\bm{D}$ and distance map $\bm{R}$. 
% Notably, the depth map represents the distance along the z-axis in camera space, while the distance map represents the line-of-sight distance from the camera center to the object.
As shown in Fig.~\ref{fig_Framework}(c), these components are unified through the physics-based UIF model described in Eq.\eqref{uif}, which computes the water effects and applies them to the water-free image, enabling the rendering of the corresponding underwater image $\bm{I}$. During the optimization, in addition to computing the reconstruction loss between $\bm{I}$ and the ground truth underwater image $\hat{\bm{I}}$, we design a depth-guided optimization mechanism that uses a pseudo-depth map $\hat{\bm{D}}$ to guide Aquatic-GS in generating more accurate underwater scene representation.

% Fig.~\ref{fig_Framework}
% thereby facilitating the decoupling of water effects from the water-free scene.
% After optimization, thanks to its hybrid representation manner, Aquatic-GS can not only render underwater scenes from novel viewpoints but can also directly render the true appearance of underwater scenes, enabling restoration of underwater environments.

In subsequent sections, we will elaborate on the detailed configurations of the neural water field (Section~\ref{sec:NWF}) and the 3DGS (Section~\ref{sec:3DGS}). The depth-guided optimization mechanism will be discussed in Section~\ref{sec:DGO}, and the implementation details will be provided in Section~\ref{sec:DETAILS}.

\subsection{Neural Water Field}~\label{sec:NWF}  
% 考虑到太阳角度、观察的角度会对水下成像模型中的水体参数存在影响,因此,对于一个特定范围内的水下场景,水体参数的波动变化主要与观察视角相关,因此,我们把水体参数在空间中的非均匀分布定义为与相机像素viewing direction相关的函数。构建一个神经水体场来学习水体参数,例如环境光、吸收系数、散射系数的关于视角的分布情况。具体而言,我们使用散射系数来

In real scenarios, the water effects are influenced by the solar directionality and viewing angles~\cite{bekerman2020unveiling, akkaynak2018revised}. Thus, we define the distributions of the water parameters within each channel as functions of the viewing direction and design a neural water field to implicitly learn these distributions. As shown in Fig.~\ref{fig_Framework}, for a given camera, the viewing direction corresponding to the $p$-th pixel is denoted as $\bm{d}_p$, and the corresponding underwater ambient light $\bm{A}_p \in {\mathbb{R}}^3$, attenuation coefficient $\bm{\beta}^D_p \in {\mathbb{R}}^3$, and backscatter coefficient $\bm{\beta}^B_p \in {\mathbb{R}}^3$ for this pixel can be queried from the neural water field. For implementation, we use spherical harmonics (SH) to model the underwater ambient light and a shallow multilayer perceptron (MLP) to model the attenuation and backscatter coefficients. Below, we describe these two modeling strategies in detail. 

% ----------------------------------------------------------------------------
% [提出未显示的参考文献] 球谐函数实践 yu2021plenoctrees 环境光照明 ramamoorthi2001efficient
\subsubsection{SH for Underwater Ambient Light}
Spherical harmonics have been widely applied for modeling low-frequency variations in environmental lightings, such as the sky's illumination distribution~\cite{habel2008efficient}. In this work, we explore their application in underwater scenes. We use learnable SH coefficients to construct a weighted sum of SH basis functions, approximating the complex distribution of underwater ambient light relative to the viewing direction.

Following the standard practice~\cite{fridovich2022plenoxels}, the underwater ambient light $\bm{A}_p$ for a given viewing direction $\bm{d}_p$ can be represented as: 
\begin{equation}
\label{SH}
\bm{A}_p=\sum_{l=0}^{l_{max}} \sum_{m=-l}^{l} \bm{k}_l^m Y_l^m (\bm{d}_p),
\end{equation}
where $\{Y_l^m(\cdot)\}_{l:0 \leq l \leq l_{max}}^{m: -l \leq m \leq l}$ denotes the set of SH basis functions, and $\{\bm{k}_l^m\}_{l:0 \leq l \leq l_{max}}^{m: -l \leq m \leq l}$ represents the corresponding learnable SH coefficients. Each $\bm{k}_l^m \in {\mathbb{R}}^3$ corresponds to the RGB components of the ambient light. The parameter $l$ indicates the degree of the SH basis function. Notably, each SH basis function defines a distribution on the sphere, with higher $l$ corresponding to more complex distribution and greater variation.  The parameter $l_{max}$ represents the highest degree of SH basis functions used in the approximation, with a larger $l_{max}$ incorporating more SH bases, thus improving the capacity to model complex distributions. For computational efficiency, we set $l_{max}=3$.

% ----------------------------------------------------------------------------
\subsubsection{MLP for Attenuation and Backscatter Coefficients}
Considering the additional influence of the local composition of the water~\cite{nakath2021situ}, we employ a shallow MLP to model the complex distributions of $\bm{\beta}^D$ and $\bm{\beta}^B$. 
% We employ a shallow MLP to model the complex distributions of $\bm{\beta}^D$ and $\bm{\beta}^B$. 
First, for a given viewing direction $\bm{d}_p$, we convert it into a Cartesian unit vector and apply the positional encoding strategy~\cite{mildenhall2021nerf} to obtain a high-dimensional embedding, denoted as $\bm{d}'_p \in {\mathbb{R}}^{27}$. The corresponding $\bm{\beta}^D_p$ and $\bm{\beta}^B_p$ are then computed as follows: 
\begin{equation}
\label{MLP}
(\bm{\beta}^D_p, \bm{\beta}^B_p) = \sigma( F_2( \sigma( F_1(\bm{d}') ) ) ),
\end{equation}
where $F_1(\cdot)$ and $F_2(\cdot)$ represent linear layers with output feature dimensions of 128 and 6, respectively, and $\sigma(\cdot)$ is the Softplus activation function.

By inputting all pixels' viewing directions into the neural water field, we can query the corresponding $\bm{A}$, $\bm{\beta}^D$, and $\bm{\beta}^B$ for the current viewpoint.


% ----------------------------------------------------------------------------
\subsection{3D Gaussian Splatting}~\label{sec:3DGS}      
% 
3D Gaussian Splatting employs a collection of 3D Gaussians to represent scenes explicitly. The spatial distribution of each Gaussian is modeled by the equation:
\begin{equation}
\label{3DGS}
G(\bm{x})=e^{- \frac{1}{2} (\bm{x}-\bm{\mu})^T {\bm{\Sigma}}^{-1} (\bm{x}-\bm{\mu})},
\end{equation}
where $\bm{x}$ is a position in world space, $\bm{\mu}$ is the mean position of the Gaussian, and $\bm{\Sigma}$ is the covariance matrix. In practice, $\bm{\Sigma}$ is computed using a scaling matrix $\bm{S}$ and a rotation matrix $\bm{R}$, defined as $\bm{\Sigma}=\bm{R}\bm{S}\bm{S}^T\bm{R}^T$, ensuring positive semi-definiteness. Each Gaussian also includes a set of learnable spherical harmonics coefficients to model view-dependent color and an opacity attribute $o$ that determines its contribution during the blending process.

For efficient rendering, 3DGS utilizes a tile-based differentiable rasterization pipeline. Initially, each 3D Gaussian $G(\bm{x})$ is projected onto the image plane as a 2D Gaussian $G'(\bm{x})$~\cite{zwicker2001ewa}. These are then depth-sorted based on their z-axis coordinates in camera space. The pixel color on the image plane is computed using $\alpha$-blending:
\begin{equation}
\label{alpha_blending}
c(\bm{x}') = \sum_{i=1}^{N_g} c_i \alpha_i \prod_{j=1}^{i-1}(1-\alpha_j), \quad \alpha_i = o_i G'(\bm{x}'), 
\end{equation}
where $\bm{x}'$ is the pixel position in image space, $c_i$ and $\alpha_i$ are the color and opacity of the $i$-th sorted Gaussian at that position, and ${N_g}$ represents the number of Gaussians involved in the blending. The differentiable rasterization process enables the end-to-end optimization of Gaussian parameters, such as $\bm{\mu}$, $\bm{R}$, $\bm{S}$, ${o}$, and SH coefficients, by calculating and backpropagating the gradients of the reconstruction loss (i.e., L1 loss and D-SSIM) between the rendered image and the ground truth.

% Initially, 3D Gaussians are derived from a sparse point cloud obtained via Structure from Motion (SfM)~\cite{schonberger2016structure}. Throughout the optimization phase, an Adaptive Density Control (ADC) strategy regulates the growth and pruning of Gaussians to refine scene details. Additionally, a periodic opacity reset is employed to mitigate the influence of floaters near the camera on the accurate reconstruction of the scene’s geometry.
 
In our work, 3DGS is employed to model the underwater objects, which correspond to the underlying water-free scene. For a given viewpoint, the RGB image rendered by 3DGS corresponds to the water-free image (i.e., the true appearance of the underwater scene) and is denoted as $\bm{J}$.
% Each pixel's color in $\bm{J}$ is computed using the $\alpha$-blending formula defined in Eq.\eqref{alpha_blending}. 
Alongside $\bm{J}$, we extend the 3DGS to render the accumulated opacity $\bm{o}$, defined as follows: 
\begin{equation}
\label{alpha_opacity}
 \bm{o}(\bm{x}')= \sum_{i=1}^{N_g} \alpha_i \prod_{j=1}^{i-1}(1-\alpha_j),
\end{equation}
Additionally, the corresponding depth map $\bm{D}$ and distance map $\bm{R}$, essential for rendering underwater images, are rendered as follows:
\begin{equation}
\label{alpha_depth}
\bm{D}(\bm{x}')= \begin{cases}
\frac{\sum_{i=1}^{N_g} d_i \alpha_i \prod_{j=1}^{i-1}(1-\alpha_j)}{ \bm{o}(\bm{x}')}, &{\text{if}} \ \bm{o}(\bm{x}') > 0 \\
r_{max}, &{\text{if}} \ \bm{o}(\bm{x}') = 0 
\end{cases}
\end{equation}
% and
\begin{equation}
\label{alpha_distance}
\bm{R}(\bm{x}')= \begin{cases}
\frac{\sum_{i=1}^{N_g} r_i \alpha_i \prod_{j=1}^{i-1}(1-\alpha_j)}{ \bm{o}(\bm{x}')}, &{\text{if}} \ \bm{o}(\bm{x}') > 0 \\
r_{max}, &{\text{if}} \ \bm{o}(\bm{x}') = 0 
\end{cases}
\end{equation}
Here, $d_i$ is the distance from the $i$-th Gaussian to the camera along the z-axis in camera space, while $r_i$ is the Euclidean distance between the $i$-th Gaussian and the camera. We compute $\bm{D}$ and $\bm{R}$ under two conditions: First, if $\bm{o}(\bm{x}') = 0$, this indicates that no Gaussians contribute to the blending at pixel $\bm{x}'$ within the current viewing frustum. In this case, we set both $\bm{D}(\bm{x}')$ and $\bm{R}(\bm{x}')$ to the frustum's radius, $r_{max}$. Second, if $\bm{o}(\bm{x}') > 0$, we normalize the $\alpha$-blended depth and distance using $\bm{o}(\bm{x}')$.

% Once we have obtained the water-free image $\bm{J}$, the distance map $\bm{R}$, and the water parameters $\bm{A}$, $\bm{\beta}^D$, and $\bm{\beta}^B$ generated by the NWF, we can utilize the UIF model described in Eq.\eqref{uif} to render the corresponding underwater image $\bm{I}$, enabling the end-to-end optimization of both NWF and 3DGS.
Once we have obtained the water-free image $\bm{J}$, the distance map $\bm{R}$, and the water parameters $\bm{A}$, $\bm{\beta}^D$, and $\bm{\beta}^B$ generated by the NWF, we can utilize the UIF model described in Eq.\eqref{uif} to synthesize the corresponding underwater image $\bm{I}$, enabling the end-to-end optimization of Aquatic-GS.

\subsection{Depth-Guided Optimization Mechanism}~\label{sec:DGO}   
% 【注意 这里说的是 water-free scene 是 scene 这样子的话 后面就要用images 】
% 仅仅计算渲染图像真值之间的默认重构损失,并不能

Solely using the reconstruction loss does not ensure that Aquatic-GS can effectively decouple the water-free scene from the water effects in underwater images. As shown in Fig.~\ref{fig_distant}, the learned water-free scene using only the reconstruction loss exhibits two main issues: first, the appearance remains hazy, especially in distant areas. Second, in distant, low-contrast areas of the underwater images, the corresponding water-free areas show severe blurring and loss of detail.


\begin{figure}[!t]
\centering
\includegraphics[width=1.0\columnwidth]{images/Fig4.pdf}
\caption{Visual comparison of the water-free scene learned with reconstruction loss alone versus those learned using our DGO mechanism. The former displays a haze, especially in distant areas, as highlighted by the red and green boxes. The introduction of our DGO mechanism effectively reduces haze and restores more geometry details in these challenging regions.}
\label{fig_distant}
\end{figure}

These issues primarily stem from the distance-dependent nature of water effects. On one hand, these effects weaken the efficacy of multi-view reconstruction loss in enforcing geometric constraints on Aquatic-GS, especially with increasing distance. Rather than focusing on opaque surfaces, the distribution of 3D Gaussians tends towards a semi-transparent arrangement in space, resulting in a hazy appearance and inaccurate geometry. On the other hand, water effects cause distant areas in underwater images to display significantly weaker texture details and lower contrast than in reality, making it challenging for Aquatic-GS to accurately represent these regions.

To overcome these issues, we have developed a Depth-Guided Optimization (DGO) mechanism, which leverages the pseudo-depth map estimated by DepthAnything~\cite{yang2024depth} to guide Aquatic-GS in accurately representing scene geometry and distant details. This mechanism addresses the issues from four specific aspects: transmittance regularization, minimization of depth variance, coarse-grained depth supervision, and patch-wise frequency domain supervision. Each aspect is implemented through a specifically designed loss function detailed in the following sections.

\subsubsection{Depth-Guided Transmittance Regularization}~\label{sec:DGT}   
Ideally, high-opacity Gaussians should be concentrated on object surfaces, while semi-transparent Gaussians should be minimized~\cite{fang2024mini}. And, for each pixel, the transmittance of Gaussians involved in $\alpha$-blending tends towards 0 or 1, avoiding intermediate values~\cite{reiser2024binary}. To achieve this, we apply transmittance regularization to suppress semi-transparent Gaussians.
% and eliminate residual backscattering effects.

First, inspired by~\cite{rebain2022lolnerf}, we introduce a metric $t(\cdot)$ to assess the presence of semi-transparent Gaussians during the $\alpha$-blending process for each pixel as follows:
\begin{equation}
\label{dgo_t}
t(\bm{x}')= \frac{1}{{N_g}} \sum_{i=1}^{N_g} \{ - log ( e^{-10| T_i|} + e^{-10|1 - T_i|} )   \}, 
\end{equation} 
where $\bm{x}'$ represents the coordinates of any given pixel, and $T_i=\prod_{j=1}^{i-1} (1-\alpha_j)$ is the transmittance of $i$-th Gaussian in the pixel' $\alpha$-blending process. A smaller value of $t(\bm{x}')$ indicates that the transmittance values of all the Gaussians contributing to the pixel are closer to either 0 or 1, suggesting fewer semi-transparent Gaussians. We further define a depth-guided transmittance regularization term $L_{dgt}$, calculated as:
\begin{equation}
\label{dgo_dgt}
L_{dgt} = \frac{1}{{HW}} \cdot \sum_{\bm{x}' \in \bm{\Omega} }  \gamma(\bm{x}') t(\bm{x}'),
\end{equation} 
where the entire image domain $\bm{\Omega}$ is segmented into a near region $\bm{\Omega}_{n} = \{ (x,y) | \hat{\bm{D}}(x,y) > \tau_1 \}$ and a far region $\bm{\Omega}_{f} = \{ (x,y)) | \hat{\bm{D}}(x,y) \leq \tau_1 \}$ based on the pseudo-depth map $\hat{\bm{D}}$. The weight $\gamma(\bm{x}')$ is set as $\gamma_n=1$ for $\bm{\Omega}_{n}$ and $\gamma_f=10$ for $\bm{\Omega}_{f}$, with $\gamma_f$ being larger to enhance the suppression of semi-transparent Gaussians in the distant regions.

\subsubsection{Depth Variance Minimization Regularization}~\label{sec:DGMV} 
For any given pixel, the normalized depth, $\bm{D}(\bm{x}')$, computed using Eq.\eqref{alpha_depth}, represents the weighted average depth of all Gaussians participating in the $\alpha$-blending at that pixel. The weight, ${w_i}$, for the $i$-th Gaussian is defined as: ${w_i}=\frac{\alpha_i \prod_{j=1}^{i-1}(1-\alpha_j)}{\sum_{i=1}^{N_g} \alpha_i \prod_{j=1}^{i-1}(1-\alpha_j)}$. Similarly, the variance of depths, $\bm{V}_D(\bm{x}')$, for all Gaussians involved in the $\alpha$-blending process for that pixel can be computed as:
\begin{equation}
\label{dgo_dgmv_var}
\bm{V}_D(\bm{x}')=\sum_{i=1}^{N_g}  {w_i} ( d_i - \bm{D}(\bm{x}'))^2.
\end{equation}

Typically, a smaller value of $\bm{V}_D(\bm{x}')$ implies that the Gaussians contributing to the current pixel are more concentrated. Consequently, we propose a depth variance minimization regularization term, $L_{dvm}$, to encourage a tighter concentration of Gaussians on object surfaces by minimizing the depth variance for each pixel: 
\begin{align} 
L_{dvm} &= \frac{1}{HW} \cdot \sum_{\bm{x}' \in \bm{\Omega}} \eta(\bm{x}') \bm{V}_D(\bm{x}'),
\label{dgo_dgv} 
\end{align} 
where $\bm{\Omega} = \bm{\Omega}_{n} \cup \bm{\Omega}_{f}$ and the weights $\eta(\bm{x}')$ are defined as $\eta_n$ within $\bm{\Omega}_{n} \setminus \bm{\Omega}_{e}$, $\eta_f$ within $\bm{\Omega}_{f} \setminus \bm{\Omega}_{e}$, and $\eta_e$ within ${\bm{\Omega}}_{e}$. The edge region, $\bm{\Omega}_e$, is determined by the pseudo-depth map $\hat{\bm{D}}$. We assign the weights $\eta_n=1$, $\eta_f=0.1$, and $\eta_e=0.001$ to reflect the varying concentration needs across the image.

This depth-guided weighting strategy is motivated by the observation that pixels corresponding to distant regions involve a wider spatial distribution of 3D Gaussians due to perspective projection. To avoid excessive compression of the depth field across this wider spatial distribution, smaller weights are applied. Moreover, in regions identified as edges in the pseudo-depth map, where Gaussians exhibit varied depths, the minimal weight is assigned to prevent excessive smoothing and retain geometric details.

% -------------------------------------------
% 逆深度相关性损失

\subsubsection{Inverse Depth Correlation Loss}~\label{sec:IDC}   
Due to the scale ambiguity inherent in the pseudo-depth maps, which encode normalized relative disparity rather than absolute depth values, conventional loss functions like L1 loss do not work well for depth supervision~\cite{paliwal2024coherentgs}. Recent work by Xiong et al.~\cite{xiong2023sparsegs} demonstrated that normalized cross-correlation can effectively measure the similarity between two maps, regardless of discrepancies in their absolute value ranges. Inspired by this, we propose an inverse depth correlation loss function, $L_{idc}$, to facilitate effective depth supervision. We first convert our depth map to its inverse form $\bm{D}' = 1/(\bm{D}+1)$, and then compute the normalized cross-correlation between $\bm{D}'$ and  $\hat{\bm{D}}$. The loss function $L_{idc}$ is defined as follows:
\begin{equation}
\label{dgo_idc}
L_{idc} =1 - {Cov(\bm{D}', \hat{\bm{D}})}/{\sqrt{Var(\bm{D}')Var(\hat{\bm{D}})}}, 
\end{equation}
where $Cov(\cdot, \cdot)$ and $Var(\cdot)$ denote covariance and variance, respectively. This approach not only addresses the issue of scale ambiguity but also enhances the geometric fidelity by aligning $\bm{D}'$ and  $\hat{\bm{D}}$ more closely, providing coarse-grained geometric supervision.

% Although the estimated pseudo-depth map corresponds directly to disparity information rather than true depth measurements and presents scale ambiguity relative to the actual scene, it still reflects the overall geometric structure of the scene. Therefore, we design an inverse depth correlation loss function, $L_{idc}$, to incorporate coarse-grained geometric supervision into the optimization of 3DGS. We first compute the inverse depth map as $\bm{D}' = 1/(\bm{D}+1)$. Drawing on the methodology outlined in~\cite{xiong2023sparsegs}, we calculate the Pearson correlation coefficient between $\bm{D}'$ and  $\hat{\bm{D}}$,  defining $L_{idc}$ as follows:
% \begin{equation}
% \label{dgo_idc}
% L_{idc} =1 - {Cov(\bm{D}', \hat{\bm{D}})}/{\sqrt{Var(\bm{D}')Var(\hat{\bm{D}})}}.
% \end{equation}



% -------------------------------------------
% 深度指导的分块频域损失 Depth-Guided Patch Frequency Loss
% 弃用的相关文献-多模态图像融合 xiao2024fafusion
% 
\subsubsection{Depth-Guided Patch Frequency Loss}~\label{sec:DPF}   
Frequency domain analysis, adept at capturing subtle details and texture variations, has been widely applied across tasks like image generation~\cite{jiang2021focal}, super-resolution~\cite{korkmaz2024training}, and image fusion~\cite{liu2024mm}. In this study, we propose a depth-guided patch frequency loss $L_{dpf}$, which utilizes frequency domain insights to enhance texture detail perception in distant, low-contrast areas of underwater images, thereby enabling Aquatic-GS to model these challenging areas more effectively.

We begin by partitioning the rendered underwater image $\bm{I}$, the ground truth underwater image $\hat{\bm{I}}$, and the pseudo-depth map $\hat{\bm{D}}$ into patches of $k \times k$ pixels using the same grid pattern. This results in $K$ patch triplets, denoted as  $\{ \bm{I}_i, \hat{\bm{I}}_i, \hat{\bm{D}}_i \}_{i=1}^{K}$, where $K=\lfloor {\frac{h}{k}} \rfloor \times \lfloor {\frac{w}{k}} \rfloor$. We then apply the Discrete Fourier Transform (DFT) to the patches from $\bm{I}$ and $\hat{\bm{I}}$. For the $i$-th patch from $\bm{I}$, denoted as $\bm{I}_i$, the DFT is computed as follows:
\begin{equation}
\label{dgo_dft}
\bm{F}_i(u,v) = \sum_{x=0}^{k-1} \sum_{y=0}^{k-1} \bm{I}_i(x,y) \cdot e^{-2\pi ( \frac{ux}{k} +  \frac{vy}{k} )},
\end{equation}
where $(x,y)$ are the spatial coordinates, $\bm{I}_i(x,y)$ represents the pixel value, $(u,v)$ are the frequency coordinates, and $\bm{F}_i(u,v)$ is the computed complex frequency value. We calculate the magnitude of $\bm{F}_i$, $|\bm{F}_i|$. A similar process is performed for each patch of $\hat{\bm{I}}$. We also calculate the average depth in each patch based on $\hat{\bm{D}}$, denoted $\hat{D}_i^{avg}$. The $L_{dpf}$ loss is then computed as:
\begin{equation}
\label{dgo_DPF}
L_{dpf} = \frac{1}{K}  \sum_{i=1}^K \psi_i \| |\bm{F}_i| -  |\hat{\bm{F}}_i| \|_1,
\end{equation}
where $\psi_i$ is a depth-dependent weight, and $\| |\bm{F}_i| -  |\hat{\bm{F}}_i|   \|_1$ represents the L1-norm between the magnitudes of $\bm{F}_i$ and $\hat{\bm{F}}_i$. Since $\hat{\bm{D}}$ corresponds to disparity, $\psi_i$ is defined as $\psi_i = 1 - \hat{D}_i^{avg}$, ensuring that patches corresponding to distant areas are assigned greater weight and attention. The final $L_{dpf}$ loss encourages Aquatic-GS to focus on enhancing texture detail representation in these challenging distant areas.


\subsection{Implementation Details}~\label{sec:DETAILS}   
Given $M$ observed images of an underwater scene, we first employ COLMAP\cite{schonberger2016structure} to calibrate the extrinsic and intrinsic camera matrices. This process also yields a sparse point cloud, which serves as the initialization for 3D Gaussians. The positions of the cameras in world space are denoted as $\{ \bm{O}_i \}_{i=1}^M$, and the point cloud is denoted by $\{ \bm{P}_j \}_{j=1}^{N_p}$, where $N_p$ is the number of points. We then determine the maximum visible distance in the scene as $r_{max} = \lambda \cdot \max_{i,j} \|\bm{O}_i - \bm{P}_j \|$, with $\lambda$ being a scale factor empirically set to 2.

Because the surfaces in underwater scenes can often be considered Lambertian reflectors~\cite{murez2015photometric}, we limit the highest order of SHs used in 3DGS to zero to enable view-independent color rendering. The final loss function used in the optimization is expressed as follows:
% \begin{equation}
% \begin{aligned} 
% L_{final} = &(1-\lambda_{ssim})L_1 + \lambda_{ssim} L_{D\text{-}SSIM} + \lambda_{dgt} L_{dgt}\\
% &+ \lambda_{dvm} L_{dvm} + \lambda_{idc} L_{idc} + \lambda_{dpf} L_{dpf},
% \end{aligned}
% \label{total_loss}
% \end{equation}
\begin{equation}
\begin{aligned} 
L_{final} = &L_{recon} + \lambda_{dgt} L_{dgt} + \lambda_{dvm} L_{dvm}\\
&+ \lambda_{idc} L_{idc} + \lambda_{dpf} L_{dpf},
\end{aligned}
\label{total_loss}
\end{equation}
where $L_{recon}=(1-\lambda_{ssim})L_1 + \lambda_{ssim} L_{D\text{-}SSIM}$ denotes the reconstruction loss between the rendered and ground truth underwater images, and $L_{D\text{-}SSIM}$ and $\lambda_{ssim}$ denote the D-SSIM loss and its weight, respectively. Additionally, $\lambda_{dgt}$, $\lambda_{dvm}$, $\lambda_{idc}$, and $\lambda_{dpf}$ are the weights assigned to the respective loss functions within our DGO mechanism. After optimization, for a given viewpoint, Aquatic-GS not only renders the underwater image but also supports the rendering of the corresponding water-free image through the 3DGS branch, achieving underwater image restoration.


\section{Experiments}~\label{sec:experiment}
In this section, we demonstrate the effectiveness of our proposed Aquatic-GS through extensive experiments. First, we describe the experimental configurations, including the datasets, implementation details, and evaluation metrics, in Section ~\ref{sec:config}. Then, in Section~\ref{sec:performance_NVS} and Section~\ref{sec:performance_UIR}, we compare the performance of Aquatic-GS with state-of-the-art methods for underwater novel view synthesis and underwater image restoration, respectively. Finally, in Section~\ref{sec:ablation} we conduct ablation studies to further analyze the effectiveness of the proposed Aquatic-GS.


\subsection{Experimental Configurations}~\label{sec:config}
\subsubsection{Datasets}
We employed four distinct datasets to assess the performance of our Aquatic-GS, comprising three datasets from actual underwater environments and one simulated dataset.


% Cura\c{c}ao
{\bf{SeathruNeRF}~\cite{levy2023seathru}:} This dataset comprises four real underwater scenes: Curacao, Panama, Japanese Gardens, and IUI3, each reflecting diverse aquatic and imaging conditions. The scene image counts are 21, 18, 20, and 29, respectively. Following~\cite{levy2023seathru}, these images were downsampled to an average resolution of 900$\times$1400 pixels.
% We utilized the publicly available COLMAP dataset for distortion correction, acquiring PINHOLE camera intrinsics, camera poses, and the undistorted images essential for 3DGS.

{\bf{Seathru}~\cite{akkaynak2019sea}:} We employed the horizontally-viewed D3 and D5 scenes from this dataset, where each scene consists of 68 and 43 images, respectively.
% captured under consistent exposure settings and natural lighting. 
Color charts with known patterns are distributed throughout these scenes, serving as ground truth to evaluate Aquatic-GS' effectiveness in underwater image color correction.

{\bf{In-the-Wild Underwater (IWU) Dataset}:}
We selected two in-the-wild scenes, Coral and Car, from raw, unstructured footage available online to assess Aquatic-GS's performance in general real-world scenarios. 
% The Coral scene displays a vast coral formation, while the Car scene showcases a model submerged at Cancun's MUSA. 
For each scene, we sampled frames from the videos, yielding 38 and 54 images, respectively.

{\bf{Our Simulated Dataset}:} 
We constructed three simulated underwater scenes based on Blender, each showcasing distinct challenges: S1 exhibits detailed textures and varied colors; S2, a large-scale environment, displays significant low contrast at distant regions; and S3 shows pronounced greenish color distortions. Each scene contains 50 images with a resolution of 720$\times$1280. More construction details can be found in the supplementary materials.

For real-world scenes, following~\cite{akkaynak2019sea}, we applied white balancing and used COLMAP to obtain camera intrinsics and poses. For simulated scenes, these parameters were exported from Blender and converted to COLMAP format. As our focus is on static underwater scenes, we manually annotated masks to exclude dynamic elements, such as divers in the D5 scene and fish in the Coral scene, to avoid interference.

% First, we sourced three 3D underwater models online, adjusted their textures for a water-free appearance, and used Blender to render clear images and their corresponding depth and distance maps. Each scene produced 50 clear images with a resolution of 720$\times$1280.
% % Forward-looking camera paths simulated the movement of divers or underwater vehicles. 
% We then utilized the simplified version of Eq.\eqref{uif} to simulate the underwater images. For the S1 scene, $\bm{\beta}^D=[3.3, 2.9, 2.5]$, $\bm{\beta}^B=[2.0, 1.9, 1.8]$, and $\bm{A}=[0.10, 0.55, 0.78]$. For the S2 scene, $\bm{\beta}^D=[1.3, 2.4, 4.0]$, $\bm{\beta}^B=[2.0, 1.9, 1.8]$, and $\bm{A}=[0.10, 0.65, 0.41]$. For the S3 scene, $\bm{\beta}^D=[5.5, 4.8, 4.2]$, $\bm{\beta}^B=[6.2, 5.8, 5.5]$, and $\bm{A}=[0.23, 0.38, 0.49]$. 

% ----------------------------------------------------------------------
\begin{table*}[]
\caption{Quantitative evaluations of the underwater NVS task on real-world and simulated datasets. For each dataset, the best metric is highlighted in \textbf{bold} and the second-best in \underline{underlined}. Efficiency metrics are also included.}
\label{tab_nvs}
\renewcommand{\arraystretch}{1.6}  
\setlength{\tabcolsep}{2.6pt}
\centering


\begin{tabular}{l|ccc|ccc|ccc|ccc|c|c}
\hline
\multicolumn{1}{c|}{\multirow{2}{*}{Methods}} & \multicolumn{3}{c|}{Seathru}                                & \multicolumn{3}{c|}{SeathruNeRF}                            & \multicolumn{3}{c|}{IWU}                                    & \multicolumn{3}{c|}{Simulated}                              & \multirow{2}{*}{\begin{tabular}[c]{@{}c@{}}Speed \\ (FPS)\end{tabular}} & \multirow{2}{*}{\begin{tabular}[c]{@{}c@{}}Avg. \\ Time\end{tabular}} \\ \cline{2-13}
\multicolumn{1}{c|}{}                         & PSNR $\uparrow$    & SSIM $\uparrow$   & LPIPS $\downarrow$ & PSNR $\uparrow$    & SSIM $\uparrow$   & LPIPS $\downarrow$ & PSNR $\uparrow$    & SSIM $\uparrow$   & LPIPS $\downarrow$ & PSNR $\uparrow$    & SSIM $\uparrow$   & LPIPS $\downarrow$ &                                                                         &                                                                             \\ \hline
Mip-NeRF 360~\cite{barron2022mip}             & 21.707             & 0.681             & 0.369              & 27.030             & 0.868             & 0.230              & \underline{25.765} & 0.821             & 0.240              & 26.721             & 0.788             & 0.322              & 0.13                                                                    & 12.8h                                                                       \\
InstantNGP~\cite{muller2022instant}           & 20.847             & 0.695             & 0.376              & 20.817             & 0.729             & 0.381              & 20.582             & 0.779             & 0.240              & 25.490             & 0.855             & 0.213              & 0.44                                                                    & 4.7h                                                                        \\
NeuralSea~\cite{zhang2023beyond}              & -                  & -                 & -                  & 25.412             & 0.709             & 0.461              & -                  & -                 & -                  & 28.340             & 0.751             & 0.371              & 0.05                                                                    & 65.8h                                                                       \\
SeaThru-NeRF~\cite{levy2023seathru}           & 21.267             & 0.657             & 0.409              & \underline{27.493} & \underline{0.874} & 0.226              & 25.119             & 0.804             & 0.268              & 30.475             & 0.853             & 0.222              & 0.13                                                                    & 13.5h                                                                       \\
3DGS~\cite{kerbl20233d}                       & \underline{21.764} & \underline{0.757} & \underline{0.280}  & 24.295             & 0.857             & \underline{0.222}  & 24.527             & \underline{0.872} & \underline{0.146}  & \underline{38.243} & \underline{0.974} & \underline{0.050}  & 134.03                                                                  & 0.4h                                                                        \\
Aquatic-GS                                    & \textbf{23.193}    & \textbf{0.782}    & \textbf{0.235}     & \textbf{28.125}    & \textbf{0.894}    & \textbf{0.176}     & \textbf{26.664}    & \textbf{0.885}    & \textbf{0.136}     & \textbf{40.008}    & \textbf{0.981}    & \textbf{0.031}     & 53.64                                                                   & 1.0h                                                                        \\ \hline
\end{tabular}


\end{table*}  




% ----------------------------------------------------------------------

% \begin{figure*}[!t]
% \centering
% \includegraphics[width=2.0\columnwidth]{images/nvs.pdf}
% \caption{Visual results.}
% \label{fig_nvs}
% \end{figure*}


\begin{figure*}[!t]
\centering
\includegraphics[width=2.0\columnwidth]{images/nvs.pdf}
\caption{Qualitative comparison in the underwater NVS task. Alongside the rendered underwater images, corresponding rendered depth maps are also displayed to assess the geometric representation capabilities of different approaches. Challenging regions are highlighted within red boxes. The pseudo-depth maps serve as benchmarks for evaluating geometric fidelity. Aquatic-GS demonstrates superior rendering quality, effectively reducing blur and artifacts, while reconstructing more details, particularly in distant regions. Regarding geometric representation, our Aquatic-GS generates more reasonable depth maps that are smooth with clear edges and minimal influence from floaters.
}
\label{fig_nvs}
\end{figure*}

% ----------------------------------------------------------------------




\subsubsection{Implementation Details}

Our implementation is based on the official PyTorch version of 3DGS~\cite{kerbl20233d}. We extended its CUDA-based differentiable rasterization pipeline to support the rendering of depth maps, distance maps, and accumulated opacity maps, as well as computations and gradient backpropagation for $L_{dgt}$ and $L_{dvm}$. During training, the weights in Eq.\eqref{total_loss} were $\lambda_{ssim}=0.2$, $\lambda_{dgt}=0.0001$, $\lambda_{dvm}=0.001$, $\lambda_{idc}=0.1$, and $\lambda_{dpf}=0.02$. The patch size $k$ used to compute $L_{dpf}$ is set to 256. For the NWF, we set the learning rates for zero-order SH coefficients at $2.5\times10^{-3}$ and higher-order SH coefficients at $1.25\times10^{-4}$. The MLP within NWF uses an exponential decay learning rate schedule, decreasing from $2\times10^{-3}$ to $2\times10^{-5}$. All other optimization parameters followed the original 3DGS settings. Inspired by ~\cite{zhang2024pixel, ye2024absgs}, we accumulated the per-pixel absolute gradient norms of $\bm{\mu}$ for finer densification. For the NVS task, one in every eight images was selected for testing, with the rest for training. For the UIR task, all images were utilized for training. Each scene underwent 30,000 iterations on a single NVIDIA RTX 3090 GPU.

\subsubsection{Evaluation Metrics}
For the underwater NVS task, we use PSNR, SSIM, and LPIPS metrics to evaluate performance by comparing rendered underwater images of test viewpoints with static ground truth images. We also report the average training time  (Avg. Time) and rendering speed (Frames Per Second, FPS) on an RTX 3090 GPU at a resolution of 720$\times$1280. 

% 关闭一些文献: ancuti2017color, li2021underwater,
For the UIR task, where the water-free reference image is hard to obtain, color charts provide reliable benchmarks for evaluating color correction. Following recent research~\cite{peng2023u, wang2023domain, bekerman2020unveiling}, we use CIEDE2000 ($\Delta E_{00}$)\cite{gaurav2005ciede2000} and average angular error $\bar{\psi}$ (in degrees)\cite{boittiaux2024sucre} to measure color fidelity. For simulated scenes with clean images, PSNR, SSIM, and LPIPS assess overall restoration quality. Moreover, visual inspection remains essential alongside quantitative metrics.











% %%%%%%%%%%%%% 【这里介绍基于像素的方法时候,可以提出他们难以有效解决 distance 相关的现象】
\subsection{Performance of Underwater Novel View Synthesis}~\label{sec:performance_NVS}
% 【注意】白平衡的事情,由于数据集那里已经提到,在这里不再进行说明。  For real-world scenes, the input to all approaches is the same set of white-balanced images. 
% 【对比方法使用的代码,这里不再强调了】For both methods, we use the publicly released code from the authors.

We compared our method with five state-of-the-art approaches, including general scene 3D representation methods Mip-NeRF 360~\cite{barron2022mip}, InstantNGP~\cite{muller2022instant}, and the original 3DGS~\cite{kerbl20233d}, along with two NeRF-based underwater methods: NeuralSea~\cite{zhang2023beyond} and SeaThru-NeRF~\cite{levy2023seathru}. Due to NeuralSea's specific scenario requirements, we evaluated it exclusively on the SeaThru-NeRF dataset and our simulated dataset.

The quantitative results are summarized in Table~\ref{tab_nvs}. Our method consistently demonstrated superior performance across all metrics and datasets, achieving the highest PSNR and SSIM while exhibiting the lowest LPIPS. Notably, the original 3DGS outperformed the NeRF-based approaches overall. On the SeaThru dataset, our method surpassed the second-best 3DGS by 1.429 dB in PSNR. In the SeaThru-NeRF dataset, we improved PSNR by 0.632 dB and SSIM by 0.02 compared to SeaThru-NeRF, while reducing LPIPS by 0.046. In the IWU dataset, Mip-NeRF 360 outperformed 3DGS in PSNR but was still 0.899 dB lower than our method.  On the simulated dataset, our method surpassed 3DGS by 1.765 dB in PSNR. The advantages of our approach over the original 3DGS stem from the distinct modeling of water and objects, as well as the DGO mechanism, demonstrating a more effective scene representation.

Visual comparison results are shown in Fig.~\ref{fig_nvs}. Mip-NeRF 360 struggles to reconstruct distant details, resulting in blurriness. Conversely, SeaThru-NeRF recovers more details in such areas; however, the substantial noise present in the depth maps indicates numerous floaters, suggesting difficulties in learning the true scene geometry. The original 3DGS also struggles to reconstruct details in distant regions significantly affected by water. Its explicit modeling strategy fails to effectively model the water medium, causing numerous floaters to match the training views and resulting in artifacts and incorrect geometry when rendering from test viewpoints. These floaters are closer to the camera than the actual scene objects, shown as bright yellow regions in the depth maps. In contrast, our method achieves superior visual quality with reduced blurriness and artifacts, effectively reconstructing more texture details, particularly in challenging, distant regions. Moreover, it generates a more reasonable depth map, characterized by overall smoothness, sharp edges, and fewer floaters, aligning better with underwater object distribution, demonstrating a more effective scene geometry representation.

In terms of efficiency, we achieve 410$\times$ faster rendering and reduce training time to 7\% of SeaThru-NeRF. Despite the neural water field and DGO mechanism extending training time and slightly reducing speed compared to 3DGS, we maintain real-time rendering at 720$\times$1280 resolution at 53.64 FPS, supporting broad real-time applications.



% ----------------------------------------------------------------------
\begin{table*}[]
\caption{Quantitative evaluations of the UIR task were conducted on three real-world scenes (D3, D5, and Curacao) with color charts and our simulated dataset. Metrics on real-world scenes assess algorithms' color correction capabilities, while those on the simulated dataset evaluate overall underwater image restoration performance. For each scene or dataset, the best metric is highlighted in \textbf{bold} and the second-best in \underline{underlined}.}
\label{tab_uir}
\renewcommand{\arraystretch}{1.6}  
\setlength{\tabcolsep}{7.8pt}
\centering


\begin{tabular}{cl|cc|cc|cc|ccc}
\hline
\multicolumn{2}{c|}{\multirow{2}{*}{Method}}                                                                         & \multicolumn{2}{c|}{D3 Scene}                            & \multicolumn{2}{c|}{D5 Scene}                            & \multicolumn{2}{c|}{Curasao Scene}                       & \multicolumn{3}{c}{Simulated Dataset}                       \\ \cline{3-11} 
\multicolumn{2}{c|}{}                                                                                                & $\Delta E_{00}$ $\downarrow$ & $\bar{\psi}$ $\downarrow$ & $\Delta E_{00}$ $\downarrow$ & $\bar{\psi}$ $\downarrow$ & $\Delta E_{00}$ $\downarrow$ & $\bar{\psi}$ $\downarrow$ & PSNR $\uparrow$    & SSIM $\uparrow$   & LPIPS $\downarrow$ \\ \hline
\multirow{2}{*}{\begin{tabular}[c]{@{}c@{}}Visual\\ Prior-based\end{tabular}} & Fusion~\cite{ancuti2012enhancing}    & 22.94                        & 22.72                     & 24.06                        & 25.60                     & 30.22                        & 26.61                     & 18.152             & 0.873             & 0.097              \\
                                                                              & MLLE~\cite{zhang2022underwater}      & 23.34                        & \underline{21.50}         & 29.97                        & 24.98                     & 23.82                        & 25.20                     & 16.895             & 0.729             & 0.174              \\ \hline
\multirow{4}{*}{Data-driven}                                                  & Waternet~\cite{li2019underwater}     & 21.63                        & 23.39                     & 25.37                        & 24.86                     & 24.86                        & 25.67                     & \underline{20.795} & \underline{0.924} & 0.096              \\
                                                                              & TACL~\cite{liu2022twin}              & 21.59                        & 23.70                     & 25.54                        & 26.62                     & 25.34                        & 26.13                     & 20.219             & 0.909             & 0.123              \\
                                                                              & TUDA~\cite{wang2023domain}           & 24.71                        & 24.52                     & 22.11                        & 25.30                     & 29.00                        & 27.18                     & 19.351             & 0.904             & 0.089              \\
                                                                              & UVEB~\cite{xie2024uveb}              & \underline{19.98}            & 22.18                     & 34.69                        & 26.13                     & \underline{21.03}            & 25.29                     & 19.158             & 0.855             & 0.186              \\ \hline
\multirow{4}{*}{Physics-based}                                                & Seathru~\cite{akkaynak2019sea}       & 23.84                        & 23.16                     & 21.75                        & 23.98                     & 28.43                        & 25.78                     & 19.199             & 0.914             & \underline{0.075}  \\
                                                                              & Hazeline~\cite{berman2020underwater} & 22.49                        & \textbf{20.01}            & 25.11                        & 24.04                     & 30.90                        & 24.63                     & 14.259             & 0.715             & 0.236              \\
                                                                              & USUIR~\cite{fu2022unsupervised}      & 25.08                        & 23.47                     & 23.32                        & 26.29                     & 31.89                        & 27.09                     & 18.902             & 0.883             & 0.124              \\
                                                                              & SUCRe~\cite{boittiaux2024sucre}      & 24.99                        & 23.24                     & \textbf{16.50}               & \underline{21.90}         & 30.58                        & 25.90                     & 20.140             & 0.918             & 0.087              \\ \hline
\multirow{2}{*}{NeRF-based}                                                   & NeuralSea~\cite{zhang2023beyond}     & -                            & -                         & -                            & -                         & 28.17                        & 29.28                     & 11.236             & 0.422             & 0.486              \\
                                                                              & SeaThru-NeRF~\cite{levy2023seathru}  & 20.24                        & 23.57                     & 34.71                        & 26.98                     & 21.36                        & \underline{23.50}         & 15.597             & 0.654             & 0.399              \\ \hline
3DGS-based                                                                    & Aquatic-GS (ours)                    & \textbf{19.95}               & 21.95                     & \underline{20.16}            & \textbf{18.95}            & \textbf{17.73}               & \textbf{21.55}            & \textbf{29.201}    & \textbf{0.970}    & \textbf{0.042}     \\ \hline
\end{tabular}

\end{table*}   



\begin{figure}[!t]
\centering
\includegraphics[width=1.0\columnwidth]{images/Hue_red.pdf}
\caption{Hue values vs. distance. We tracked the hue values of the restored (a) red and (b) yellow color patches at different observation distances within the Curacao scene. The first row illustrates the situation in the original underwater images. The results demonstrate that our Aquatic-GS outperforms other approaches in color correction and maintaining stable hue values.}
\label{fig_hue}
\end{figure}



\subsection{Performance of Underwater Image Restoration}~\label{sec:performance_UIR}
% 【2个关键的事情:颜色校正和细节保留】
% 【先不讲Seathru算法是SUCRe复现的了】boittiaux2024sucre
% 【白平衡的事情要不也不要讲了, 篇幅不够,容易引起歧义,参考的是~\cite{bekerman2020unveiling}】
% 【关于Multi-view的事情,不说了】NeuralSea, SeaThru-NeRF, and SUCRe are similar to ours in that they operate based on multi-view images of the underwater scene, while most of the other methods are implemented based on single underwater images.
% 【评价的是无水场景外观【暗示不考虑水】】
% 【introduction可以不讲跨视角一致性,可以讲,我们能够在不同的深度,均一致的恢复!—— 这个要体现出是相比其他方法的优势!——思考 可靠怎么说,比较 SeathruNeRF并没有实现这种优势,所以可以不用归结于使了3D结构】
% 【过度增强亮度的表述】UIEM over-enhances the brightness that results
% 【要不要放一个S2的图片?强调一下我们可以更好的恢复远处的色彩。】 

We compared our method with 12 representative underwater image restoration methods, categorized as visual prior-based (Fusion~\cite{ancuti2012enhancing}, MLLE~\cite{zhang2022underwater}), data-driven (Waternet~\cite{li2019underwater}, TACL~\cite{liu2022twin}, TUDA~\cite{wang2023domain}, UVEB~\cite{xie2024uveb}), physics-based (Seathru~\cite{akkaynak2019sea}, Hazeline~\cite{berman2020underwater}, USUIR~\cite{fu2022unsupervised}, SUCRe~\cite{boittiaux2024sucre}), and NeRF-based (NeuralSea~\cite{zhang2023beyond}, SeaThru-NeRF~\cite{levy2023seathru}) approaches. The latter three leverage multi-view information of the underwater scene for 3D reconstruction, while the rest are primarily implemented based on single images. 
% Due to the unavailability of the original SeaThru implementation, we employed Boittiaux et al.'s latest reproduction. For other methods, we used publicly available codes and models provided by the original authors.
  
In terms of color correction accuracy, we analyzed the color differences between restored and expected color patches in three real-world scenes (D3, D5, and Curacao). Table~\ref{tab_uir} shows that our method generally outperforms others in $\Delta E_{00}$ and $\bar{\psi}$ across all scenes. Notably, in the Curacao scene, our approach achieved the lowest $\Delta E_{00}$ and $\bar{\psi}$ values, reducing $\Delta E_{00}$ by 3.3 compared to UVEB and lowering $\bar{\psi}$ by 1.95 degrees relative to SeaThru-NeRF. Moreover, we tracked the hue shifts of the restored red and yellow color patches in the Curacao scene at different observation distances (Fig.~\ref{fig_hue}). Most methods, except UVEB, NeuralSea, and SeaThru-NeRF, oversaturated close-range colors and failed to correct greenish tints at longer distances. While UVEB and SeaThru-NeRF effectively corrected nearby color casts, they struggled with stronger color casts as the patches were farther away. Conversely, our approach consistently delivered effective restoration across various distances, maintaining stable hues and outperforming competitors in color fidelity.


\begin{figure*}[!t]
\centering
\includegraphics[width=2.0\columnwidth]{images/dewater.pdf}
\caption{Visual comparison in the UIR task. We present visual comparison results among our method and the latest methods from visual prior-based, data-driven, physics-based, and NeRF-based approaches. For the simulated dataset, the corresponding water-free ground truth is displayed. Challenging regions are highlighted with red boxes. Our method effectively restores the true appearance of distant areas, where water effects are pronounced and underwater image contrast is low, achieving more realistic colors and finer texture details. Even in the S3 scene, where the color cast is particularly pronounced, our method yields results that closely match the ground truth.}
\label{fig_uir}
\end{figure*}


For overall performance in underwater image restoration, Table~\ref{tab_uir} shows that our approach outperforms others on the simulated dataset, achieving superior PSNR, SSIM, and LPIPS values—8.406 dB, 0.046, and 0.033 better than the next best methods, respectively—highlighting its effectiveness. Fig.~\ref{fig_uir} provides a visual comparison with state-of-the-art methods. In real-world scenes, MLLE struggles with enhancing distant regions due to a lack of underwater physics consideration. UVEB addresses color deviations partially but leaves a reddish tint and severe backscatter in distant areas, likely due to a domain gap between training and testing images. SUCRe corrects colors and recovers details in distant regions but fails in areas with poorly reconstructed 3D models. SeaThru-NeRF recovers distant appearances more accurately, yet its performance is unstable, with the D5 scene unrecovered, distant regions in the Curacao scene dark, and D3 scene textures blurry. In contrast, our method stably restores scenes with realistic colors and textures, especially in distant regions. In the simulated S3 scene, MLEE corrects color but over-enhances contrast, UVEB and SeaThru-NeRF show limited restoration, and SUCRe removes backscatter but introduces color artifacts. Unlike these, our method effectively mitigates severe water effects, producing results closer to the ground truth. Overall, our method outperforms others in terms of visual fidelity, color correction, detail recovery, and stability.


% 【SUCRe说明】In other words, our method fully exploits the virtual dynamic range augmentation offered by multiple observations of the color charts in different images. Additionally, this ensures color consistency across elements of the scene at different distances from the sensor. Figure 4 illustrates that, when texturing a 3D mesh, restoring underwater images using our approach leads to significant improvements, including finer details as well as plausible and coherent colors.

% 【性能介绍参考】it can be observed that our method not only obtains a pleasing visual perception but also achieves quite good color restoration accuracy. Such results further demonstrate the superiority of the proposed method in underwater image color correction.


\subsection{Ablation Studies}~\label{sec:ablation}

To demonstrate the effectiveness of our Aquatic-GS, we conducted a series of ablation studies. These studies systematically evaluated the effectiveness of modeling water parameters with non-uniform distributions, as well as the contribution of each loss function component in the DGO mechanism. All ablation experiments were performed on the SeathruNeRF dataset, with experimental settings kept consistent across all tests.



% ----------------------------------------------------------------------
\begin{table}
\caption{Ablation Study on Modeling Non-uniform Distributions of Water Parameters.}
\label{tab_NWF_setting}
\renewcommand{\arraystretch}{1.6}  
\setlength{\tabcolsep}{2.5pt}
\centering
        
\begin{tabular}{c|cccc}
\hline
Distribution Setting                         & $\Delta E_{00}$ $\downarrow$ & $\Delta E_{00}$ std $\downarrow$ & $\bar{\psi}$ $\downarrow$  & $\bar{\psi}$ std  $\downarrow$\\ \hline
Uniform Distribution     & 19.92           & 4.03                & 21.55          & 2.25             \\
Non-uniform Distribution & \textbf{17.73}  & \textbf{0.64}       & \textbf{21.55} & \textbf{0.97}    \\ \hline
\end{tabular}

\end{table}

\begin{figure}[!t]
\centering
\includegraphics[width=1.0\columnwidth]{images/Uniform.pdf}
\caption{Visual comparison of the learned water-free images with different settings of water parameter distribution. The red box highlights a region in the lower left of the view, while the green box emphasizes a region in the upper right. Our non-uniform distribution setting faithfully restores the scene colors, avoiding both over-enhancement and under-restoration.}
\label{fig_HWF}
\end{figure}

\subsubsection{Ablation Study on Modeling Non-uniform Distributions of Water Parameters} 

To validate the need for modeling water parameters as non-uniform distributions, we create a configuration where water parameters are uniformly distributed within each channel. Table~\ref{tab_NWF_setting} quantitatively evaluates color correction performance in the Curacao scene under different water parameter distribution settings. We report the standard deviations of $\Delta E_{00}$ and $\bar{\psi}$ across all restored color charts to assess restoration stability. With uniform water parameter distributions, $\Delta E_{00}$ increased by 2.19, indicating reduced color correction accuracy, while the standard deviations of $\Delta E_{00}$ and $\bar{\psi}$ rose by 3.39 and 1.28, reflecting greater instability.

Fig.~\ref{fig_HWF} shows the learned water-free images under different settings. With the uniform distribution setting, the restored underwater image shows a reddish tint in the lower left view due to over-restoration, while the upper right view appears too dark, indicating insufficient recovery of attenuated light. This suggests that uniformly distributed water parameters are excessive for the lower left and insufficient for the upper right, failing to capture complex spatial variations of water parameters. In contrast, our non-uniform distribution setting provides more reliable restoration, avoiding both over- and under-restoration, demonstrating its effectiveness in complex real-world scenarios.


% ----------------------------------------------------------------------
\begin{table}
\caption{
Ablation Study on the Individual Components of the DGO Mechanism.
}
\label{tab_Ablation_DGO}
\renewcommand{\arraystretch}{1.6}  
\setlength{\tabcolsep}{2pt}
\centering
\begin{tabular}{c|c|c|c|c|ccc}
\hline
Comparative Model & $L_{dgt}$  & $L_{dvm}$  & $L_{idc}$  & $L_{dpf}$  & PSNR $\uparrow$ & SSIM $\uparrow$ & LPIPS $\downarrow$ \\ \hline
Base              &            &            &            &            & 26.695          & 0.879          & 0.205          \\
-                 & \checkmark &            &            &            & 27.092          & 0.880          & 0.205          \\
-                 & \checkmark & \checkmark &            &            & 27.823          & 0.890          & 0.191          \\
-                 & \checkmark & \checkmark & \checkmark &            & 27.906          & 0.893          & 0.190          \\
Aquatic-GS        & \checkmark & \checkmark & \checkmark & \checkmark & \textbf{28.125} & \textbf{0.894} & \textbf{0.176} \\ \hline
\end{tabular}
\end{table}



\subsubsection{Ablation Study on The Individual Components of The DGO Mechanism}

To validate the effectiveness of each loss function component in the proposed depth-guided optimization mechanism, we designate the Aquatic-GS without the DGO mechanism as the Base model. We then incrementally incorporate $L_{dgt}$, $L_{dvm}$, $L_{idc}$, and $L_{dpf}$ to establish various comparison models, as shown in Table~\ref{tab_Ablation_DGO}. This table presents quantitative metrics for different combinations of the DGO mechanism's components in the underwater NVS task, evaluating their performance in representing underwater scenes. The results indicate that each loss function component contributes to performance improvements to varying degrees. Notably, $L_{dgt}$ and $L_{idc}$ significantly enhance the PSNR, while $L_{dvm}$ effectively improves PSNR, SSIM, and LPIPS. The inclusion of $L_{dpf}$ substantially boosts PSNR and reduces LPIPS, resulting in the best performance.

Compared with the Base model, our Aquatic-GS with the DGO mechanism achieves a PSNR improvement of 1.43 dB, an SSIM increase of 0.015, and a further LPIPS reduction of 0.029. As illustrated in Fig.~\ref{fig_distant}, the DGO mechanism effectively facilitates the decoupling of water effects from the water-free scene, leading to a reduction in haze and blur while restoring more geometry details, particularly in challenging distant regions.


\section{Conclusion}~\label{sec:conclusion}

In this paper, we present Aquatic-GS, a novel 3D representation method for underwater scenes. This method employs a hybrid representation strategy that implicitly models the water medium while explicitly representing the objects. Specifically, the neural water field learns the non-uniform distributions of water parameters, while the extended 3DGS accurately captures the true appearance and geometry of underwater scenes. Both components are unified using a physics-based underwater imaging model to generate high-quality underwater images. Our depth-guided optimization mechanism overcomes the negative effects of water, ensuring accurate scene geometry, especially for distant details. Our Aquatic-GS effectively renders novel underwater viewpoints and faithfully restores underwater scenes. Extensive experiments on real-world and simulated datasets demonstrate that Aquatic-GS surpasses SOTA NeRF-based underwater representation methods with higher rendering quality and 410$\times$ faster real-time rendering performance. Furthermore, it outperforms representative dewatering approaches in color correction, detail recovery, and stability.

% More settings can be seen in the supplementary materials.
% For further implementation details, please refer to our supplementary
% More qualitative results are in Supplementary. (Best viewed in color.)

% \section*{Acknowledgments}
% This should be a simple paragraph before the References to thank those individuals and institutions who have supported your work on this article. Thank these people for their support and help: Hongkun Dou, Yiming Hou, Yuxuan Zhang, Xingyu Jiang, Bohan Li, Yiming Qi


% ------------------------------------------References------------------------------------------------
% argument is your BibTeX string definitions and bibliography database(s)
\bibliography{IEEEabrv,my_article}   
\bibliographystyle{IEEEtran}

% % ------------------------Biography----后面可以直接把这里全解除注释--------------------------------
% \section{Biography Section}
% \begin{IEEEbiography}[{\includegraphics[width=1in,height=1.25in,clip,keepaspectratio]{kenan.png}}]{Tomas Liu}
% He was born in 1997.5, and as an B.S.xxx
% Sonar Image Denoising Book: https://www.intechopen.com/chapters/18875
% \end{IEEEbiography}

% \vspace{11pt}

% \bf{If you will not include a photo:}\vspace{-33pt}
% \begin{IEEEbiographynophoto}{Tomas Liu}
% He was born in 1997.5, and as an B.S.xxx
% \end{IEEEbiographynophoto}

\vfill
\end{document}


