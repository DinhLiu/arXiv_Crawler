%%%%%%%%%%%%%%%%%%%%%%%%%%%%%%%%%%%%%%%%%%%%%%%%%%%%%%%%%%%%%%%%%%
% Introduction
%%%%%%%%%%%%%%%%%%%%%%%%%%%%%%%%%%%%%%%%%%%%%%%%%%%%%%%%%%%%%%%%%%
\section{Introduction}\label{sec:introduction}
Lithium-ion (Li-ion) batteries are among the most widely used energy storage solutions today, powering everything from consumer electronics to electric vehicles (EVs), even resulting in the 2019 Nobel Price in Chemistry \citep{fernholm_nobel_2019}. Their popularity stems from their high energy density, long lifespan, and low self-discharge rate, which make them both efficient and durable \citep{li_30_2018}. 

However, ensuring safety, reliability, and efficiency of Li-ion batteries over time requires sophisticated battery management systems (BMS) that monitor, control, and optimize battery performance. Accurate prediction of either the state of health (SOH) or state of charge (SOC) are essential to prevent unexpected failures and extend battery life. 

Traditional BMS often rely on equivalent circuit models (ECM) \citep{liu_comparative_2014} as well as electrochemical models (EM) \citep{elmahallawy_comprehensive_2022}, but these are limited by their complexity and sensitivity to varying operational conditions. In recent years, deep learning models have emerged as powerful tools for health prediction in Li-ion batteries due to their ability to learn complex, non-linear relationships directly from data, providing more accurate, adaptive, and scalable solutions for real-time health monitoring.

We noticed that most of recent works are not considering recent advances of deep learning \citep{mazzi_lithium-ion_2024, Yao2024}. We acknowledge that some works \citep{Crocioni2020} have put their focus on deploying models on embedded devices to show that small deep learning based models can be used for real-time health monitoring of Li-ion batteries. At the same time the problem of SOH prediction is a multi-disciplinary problem that requires expertise in many different disciplines like battery technology, signal processing, and deep learning. Some works use modern transformer architectures \citep{Feng2024,Gomez2024,Zhu2024}, which have shown great success in many deep learning disciplines like natural language processing and computer vision. While these show great performance, they are not well-suited for time series data with many measurement samples due to their quadratic work complexity \citep{keles2022computationalcomplexityselfattention} and require a substantial amount of resources to train and large datasets to converge \citep{Popel_2018}.\\

In this paper we propose \modelname{}, a novel deep learning model based on Mamba state space models \citep{gu_mamba_2024,behrouz_mambamixer_2024} for predicting the SOH of Li-ion batteries. Our model is designed to handle long-range temporal dependencies in time series data and passing information between channels in multi-variate time series data. We evaluate our model NASA's real-world dataset of Li-ion battery discharge cycles \citep{saha2007nasa} and demonstrate its superior performance compared to state-of-the-art deep learning models. \\

In this sense, we summarize our main contributions of this paper as follows:
\begin{enumerate}
  \item Introducing Mamba state space models to the problem of Li-ion battery SOH prediction.
  \item Developing an anchor-based resampling scheme to resample time signals to have the same number of samples while serving as a data augmentation method.
  \item Applying a sample time-based positional encoding scheme to the input sequence to tackle sample jitter, time signals of varying length and recuperation effects of Li-ion batteries.
\end{enumerate}

We release our code on GitHub\footnote{GitHub Repo: \url{https://github.com/sascha-kirch/samba-mixer}}.