\section{Conclusion}\label{sec:conclusion}
We have presented \modelname{}, a novel approach for the prediction of the state of health of Li-ion batteries on structured state space model. We have shown that our model outperforms the state-of-the-art on the NASA battery discharge dataset~\cite{saha2007nasa}. We further introduced a novel anchor-based resampling method and a sample time and cycle time difference positional encoding to improve the performance of our model. Our results show that our model is able to predict the state of health of Li-ion batteries with high accuracy and robustness, capable to extract information from multi-variate time series data and to model recuperation effects.

% Limitations
\subsection{Limitations}\label{subsec:limitations}
Even though our model outperforms the state-of-the-art on the NASA battery discharge dataset, we acknowledge that we evaluated our model only on a single dataset; the NASA battery discharge dataset from~\cite{saha2007nasa}. This dataset only contains batteries of the same chemistry and we selected only constant discharge cycles for our experiments. Future work should evaluate our model on different datasets and different battery chemistries to further validate the generalization capabilities of our method.

% Future Line of Work
\subsection{Future Work}\label{subsec:future_work}
In future work, we plan to evaluate our model on different datasets and different battery chemistries to further validate the generalization capabilities of our model. We also plan to investigate the impact of different discharge profiles on the performance of our model. Furthermore, we plan to investigate the impact of different hyperparameters on the performance of our model and to further optimize our model for better performance. Finally, we plan to investigate different model architectures and different state space models to further improve the performance of our model.